\chapter{The Standard Model, Higgs Boson and New Scalar Particles }

\section{The Standard Model}

\section{The Higgs Boson}

\section{New Scalar Particles}

There are many deficiencies of the Standard Model (SM), such as the hierarchy problem,
flavor problem, dark matter problem, cosmological constant problem, electroweak symmetry breaking problem, CP violation problem, baryogenesis problem, etc.
The presence of a hidden sector, defined here to mean extra states that
have no SM gauge charge but are charged under some other exotic gauge symmetry, does not necessarily solve any of the problems above. However, in order to identify whether the SM Higgs sector is complete,  the searches of additional heavy scalars are performed.  They would prove the presence of beyond-the-SM (BSM) physics in the form of a non-minimal Higgs sector~\cite{Robens:2015gla}. The existence of sibling Higgs boson, denoted X, is motivated in many BSM scenarios, so the research in the full mass range accessible at colliders  remains one of the main objectives of the experimental community. This  road  needs  to  be continued within the full mass range that is accessible to current and future experiments.
\newline
\paragraph{Higgs Singlet Extension}
The simplest extension of the SM Higgs sector consist in an additional singlet which is neutral under all quantum number of the SM gauge groups.
A complex $SU(2)_L$ doublet,  denoted $\Phi$, is added by an additional real scalar $S$ which is a singlet under all SM gauge groups. 
The most general gauge-invariant and renormalisable scalar Lagrangian is,


\begin{equation}
 \mathcal{L}_s = (D_{\mu} \Phi )^{\dagger}  \: D_{\mu} \Phi  +  \partial^{\mu}S   \partial_{\mu}S -V(\Phi, S)     \end{equation}
where $V(\Phi, S) $ is the scalar potential,  

\begin{equation}
 V(\Phi, S)= -m^2   \Phi ^{\dagger}\Phi -\mu^2 S^2 +\lambda_1 (\Phi ^{\dagger}\Phi)^2 +\lambda_2 S^4 + \lambda_3 \Phi ^{\dagger}\Phi^2 S^2  \end{equation}
Here, $Z_2$ ($S \rightarrow -S$) symmetry is imposed which forbids additional terms in the potential.
The scalar potential $V(\Phi, S)$ is bounded from below if the following conditions are fulfilled,

\begin{equation}
 4 \lambda_1  \lambda_2 - \lambda_3^2 >0    \end{equation}

\begin{equation}
 \lambda_1 , \lambda_2  >0    \end{equation}

where if the first condition is fulfilled, the extremum is a local minimum. The
second condition (5), guarantees that the potential is bounded from below for large field values.
The Higgs fields, $\Phi$ and $S$, have non-zero vacuum expectation, denoted by $v$ and $x$, respectively.
Following the unitary-gauge prescription, the the Higgs fields is given by,
\newline
$$
{\mathcal H} \equiv \left(
\begin{array}{c}
0  \\
\frac{\tilde{h}+v }{\sqrt{2}}  \\
\end{array}
\right)
, \qquad 
S \equiv \frac{h'+x }{\sqrt{2}}
$$
Expansion around the minimum leads to the squared mass matrix
\newline
$$
{\mathcal M}^2 = \left(
\begin{array}{cc}
2 \lambda_1^2 v^2 & \lambda_3 vx  \\
\lambda_3 vx & 2 \lambda_1^2 x^2 \\

\end{array}
\right)
$$
with the mass eigenvalues

\begin{equation}
 m_h^2=  \lambda_1 v^2 + \lambda_2 x^2 -\sqrt{(\lambda_1 v^2 - \lambda_2 x^2)^2 +\lambda_3 (xv)^2 } \qquad,  \end{equation}

\begin{equation}
 m_H^2=  \lambda_1 v^2 + \lambda_2 x^2 +\sqrt{(\lambda_1 v^2 - \lambda_2 x^2)^2 +\lambda_3 (xv)^2 } \qquad,  \end{equation}

where $h$ and $H$ are the scalar fields of definite masses $m_h$ and $m_H$ respectively, with $m_h^2 < m_H^2$ .
The gauge and mass eigenstates are related via the mixing matrix
\newline
$$
\left(
\begin{array}{c}
h   \\
H \\
\end{array}
\right)
=
\left(
\begin{array}{cc}
\cos \alpha & -\sin \alpha   \\
\sin \alpha & \cos \alpha \\
\end{array}
\right)
\;
\left(
\begin{array}{c}
\tilde{h}   \\
h' \\
\end{array}
\right)
$$
\newline
where the mixing angle $ - \frac{\pi}{2} \leq \alpha \leq  \frac{\pi}{2} $ is given by,
\newline
\begin{equation}
\sin 2 \alpha= \frac{\lambda_3 xv}{\sqrt{(\lambda_1 v^2 - \lambda_2 x^2)^2 +\lambda_3 (xv)^2 } } \; , \end{equation}


\begin{equation}
\cos 2 \alpha= \frac{\lambda_2 x^2 - \lambda_1 v^2}{\sqrt{(\lambda_1 v^2 - \lambda_2 x^2)^2 +\lambda_3 (xv)^2 } } \; .  \end{equation}
\newline
By the  mixing matrix it is clear that the light (heavy) Higgs couplings to SM particles are now
suppressed by $\cos \alpha $ ( $\sin \alpha $).
The heavy Higgs is a new version of the SM Higgs with rescaled couplings
to the matter contents and to the gauge fields of the SM. In fact, the only novel channel
with respect to the light Higgs case is $H \rightarrow hh$. The partial decay width $\Gamma$ is given by \cite{Schabinger:2005ei},
\newline
\newline
 \begin{equation}
\Gamma_{ H \rightarrow hh} =  \frac{|\mu'|^2}{8 \pi m_H } \, \sqrt{1- \frac{4m_h^2}{m_H^2}}  \; , \end{equation}
\newline
where the coupling strength $\mu'$ is,
\newline
\begin{equation}
 \mu' =  - \frac{\sin 2 \alpha}{2vx} \, (\sin \alpha v + \cos \alpha x) \, (m_h^2 + \frac{m_H^2}{2})  \; . \end{equation}
\newline
In collider phenomenology, is important:
\begin{itemize}
\item the suppression of the production cross section of the two Higgs states induced by the mixing
\item the suppression of the Higgs decay modes to SM particles,
\end{itemize}
For the high mass  scenario, i.e. the case where the heavy Higgs boson is identified with the discovered Higgs state at $\sim$125 GeV, $|\sin \alpha| $= 1 corresponds to the complete decoupling of the second Higgs boson and therefore the SM-like scenario.

