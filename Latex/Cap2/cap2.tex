\chapter{The CMS experiment at LHC}

\section{The Large Hadorn Collider}
The Large Hadron Collider (LHC)~\cite{Pettersson:291782}  at CERN,  on 2008, is the largest and most powerful hadron collider ever built. Installed in the
underground tunnel which housed the Large Electron Positron Collider (LEP),
the leptonic accelerator in operation until 2nd November 2000, the LHC accelerator has
the shape of a circle with a length of about 27 km and is located at a depht varying
between 50 m to 175 m, straddling the Franco-Swiss border near Geneva. It is designed
to collide two 7 TeV counter-circulating beams of protons resulting in a center-of-mass
energy of 14 TeV, or two beams of heavy ions, in particular lead nuclei at an energy of
2.76 TeV/nucleon in the center-of-mass frame.
The transition from a leptonic collider to a hadronic collider entailed the following
advantages: first, it has been possible to build a machine that having the same size of the
previous one (and therefore accommodated in the same LEP tunnel, substantially reduc-
ing the cost and time of construction), could reach a higher energy in the center-of-mass
frame. This is due to the much lower amount of energy loss through synchrotron radiation
emitted by the accelerated particles, that is proportional to the fourth power of the ratio
E/m between their energy and their mass. Secondly, the composite structure of protons
compared to the elementary structure of electrons allows LHC to be able to access simultaneously a wider energy spectrum, despite the production of many low energies particles in a complex environment. This feature is particularly important for a machine dedicated
to the discovery of “new” physics.
Schematic description of the accelerator complex installed at CERN is shown in Fig.~\ref{lhc}
The acceleration is performed in several stages. The protons source is a Duoplasma-
tron: the protons are obtained by removing electrons from a source of hydrogen gas
and then sent to the LINAC2, a 36 m long linear accelerator which generates a pulsed
beam with an energy of 50 MeV using Radio Frequency Quadrupoles (RFQ) and focusing
quadrupole magnets. The beam is subsequently sent to the Proton Synchrotron Booster
(PSB), a circular accelerator consisting of four superimposed synchrotron rings with a
circumference of about 160 m, which increases the proton energy up to 1.4 GeV. Then,
protons are injected into the Proton Synchrotron (PS), a single synchrotron ring with a
circumference of about 600 m where the energy is increased to 25 GeV. The sequential combination of these two synchrotrons also allows to create a series of protons bunches
interspersed by 25 ns (i.e. at the frequency of 40 MHz) as required for the final correct
operation of LHC. The final proton injection stage is the Super Proton Synchrotron (SPS),
a synchrotron with a circumference of approximately 7 km where protons reach an energy
value of 450 GeV. Subsequently, protons are extracted and injected into the LHC ring
via two transmission lines, to generate two beams running in opposite directions in two
parallel pipes and which are accelerated up to the energy of interest. The beams collide
at four interaction points where the four main experiments: ALICE, ATLAS, CMS and LHCb. Two small experiments, TOTEM and LHCf, which focus on forward particles, have also been built.
The 7 TeV per-beam-energy limit on the LHC
is not determined by the electric field generated by the radiofrequency cavity but by the
magnetic field necessary to maintain the protons in orbit, given the current technology
for the superconducting magnets.
\begin{figure}
\centering
\includegraphics[scale= 0.5]{../Cap2/lhc}
\caption{Schematic description of the accelerator complex installed at CERN.}
\label{lhc}
\end{figure}
One of the most important parameter of an accelerator is the instantaneous luminosity $\mathcal{L}$. For a process having a cross section $\sigma$ and producing N particles per time unit, the
instantaneous luminosity $\mathcal{L}$ is defined by the relation
\begin{equation}
N= \mathcal{L} \sigma \end{equation}
Then the integrated luminosity $L$ can be defined as
\begin{equation}
L=\int \mathcal{L}  dt \end{equation}


\section{The Compact Muon Solenoid experiment}
The Compact Muon Solenoid (CMS)  is a general purpose detector optimized for the
proton-proton interactions analysis with the expected energy and luminosity of the LHC
particle accelerator design, identifying with precision muons, electrons and photons. It
has been designed to investigate a wide range of physics, with the search for the Higgs
boson as one of the main highlights. Search for new physics is also an important goal of the
experiment, as well as top physics and, of course, Standard Model precision measurements.
Although it has the same scientific goals as the ATLAS experiment, it uses different
technical solutions and a different magnet-system design. The CMS experiment is one
of the largest international scientific collaborations in history, involving more than 4000
people (particle physicists, engineers, technicians, students and support staff) from about
180 universities and institutes in more than 40 countries.
The experiment is placed in a cavern 100 m underground in the area called Point 5 (an old LEP access point) near the village of Cessy, in France.
The coordinate system used in CMS is a right-handed Cartesian system, having the origin
in the nominal beam collision point inside the detector. The x-axis points radially towards
the center of the LHC ring, the y-axis is directed upwards along the vertical and the z-
axis is oriented along the direction of the beams, along the counter-clockwise direction of
the ring if seen from above. The cylindrical symmetry of CMS design and the invariant
description of proton-proton physics suggest to define a new coordinate system based on
pseudo-angular coordinates, given by the triplet ($r$, $\phi$, $\eta$) where $r$ is the distance from
the $z$-axis, $\phi$ is the azimuth angle measured on the $x-y$ plane starting from the $x$-axis
and $\eta$ is the pseudorapidity, App.~\ref{psr}.\\
\newline
The CMS detector,  Fig.~\ref{cms}, is 21.6 m long, has a diameter of 15 m and a
weight of about 12, 500 tons. The constructive element that characterizes the experiment
is a solenoidal superconducting magnet which produces an internal constant magnetic
field of 3.8 T along the direction of the beams. The CMS detector is designed as a
dodecagonal base prism. The central part of the prism, named barrel, contains several
layers of detectors with cylindrical symmetry, coaxial with respect to the direction of the
beams. A set of detector disks, called endcaps, close the detector at its ends, to ensure its
tightness. 
\begin{figure}
\centering
\includegraphics[scale= 1]{../Cap2/cms}
\caption{A view of the CMS detector with its subdetectors labeled.}
\label{cms}
\end{figure}
From the inner region to the outer one, the various components of CMS are:
\begin{itemize}
\item Silicon Tracker: it is placed in the region $r$ < 1.2 m and $|\eta|$ < 2.5. It consists of
a silicon pixel vertex detector and a surrounding silicon microstrip detector, with a
total active area of about 215 m$^2$ . It is used to reconstruct charged particle tracks
and vertices;
\item Electromagnetic Calorimeter (ECAL): it is placed in the region 1.2 m $< r <$
1.8 m and $|\eta|$< 3. It consists of scintillating crystals of lead tungstate  and it is used to measure the trajectory and the energy released by photons and
electrons;
\item Hadron Calorimeter (HCAL): it is placed in the region 1.8 m $< r <$ 2.9 m and
$|\eta|$ < 5. It consists of brass layers alternated with plastic scintillators and it is used
to measure the direction and the energy released by the hadrons produced in the
interactions;
\item Superconducting Solenoidal Magnet: it is placed in the region 2.9 m $< r <$
3.8 m and  $|\eta|$< 1.5. It generates an internal uniform magnetic field of 3.8 T along
the direction of the beams, necessary to deflect the charged particles in order to
allow a measurement of their momentum through the curvature observed in the
tracking system. The magnetic field is closed with an iron yoke 21.6 m long with a
diameter of 14 m, where a residual magnetic field of 1.8 T is present, in the opposite
direction with respect to the 3.8 T field;
\item Muon System: it is placed in the region 4 m $< r < 7.4$ m and $|\eta|$ < 2.4. It consists
of Drift Tubes (DT) in the barrel region and Cathode Strip Chambers (CSC) in the
endcaps. A complementary system of Resistive Plate Chambers (RPC) is used both
in the barrel and in the endcaps. This composite tracking system for muons is used
to reconstruct tracks released by muons that pass through it. The muons chambers
are housed inside the iron structure of the return yoke that encloses the magnetic
field.
\end{itemize}

\subsection*{The Tracker}
The silicon tracker is the detector closest to the beams collision point. Its goal is
the high resolution reconstruction of the trajectories of charged particles originating
from the interaction point and the identification of the position of secondary vertices
produced by particles with a short mean life time (in particular hadrons containing the
b quark, that decay after few hundreds of $\mu$m). The events produced in the proton-
proton collisions can be very complex and track reconstruction is an involved pattern
recognition problem. Indeed, at the nominal instantaneous luminosity of operation,
an average of about 20 pile-up events overlapping to the event of interest are expected,
leading to about 1000 tracks to be reconstructed every 25 ns. In order to make the
pattern recognition easier, two requirements are fundamental:
a low occupancy detector and a large redundancy of the measured points (hits) per track.
The first requirement is achieved building a detector with high granularity 3 . The
redundancy of the hits is instead achieved having several detecting layers, and is
necessary to reduce the ambiguity on the assignment of the hits to a given track.
Nevertheless, the amount of tracker material has to be as low as possible, in order to
avoid compromising the measurement of the particle trajectory. An excessive amount
of material would indeed deteriorate the measurement, mainly because of the increased
probability of particle multiple scattering. The outer detectors such as ECAL would
be influenced by the material as well, for example because of the increased probability
for a photon to convert to an electron-positron pair in the tracker material. For this
reasons, the tracker layers are limited in number and thickness. The tracker comprises
a large silicon strip detector with a small silicon pixel detector inside it. In the central
$\eta$ region, the pixel tracker consists of three co-axial barrel layers at radii between
4.4 cm and 10.2 cm and the strip tracker consists of ten co-axial barrel layers extending
outwards to a radius of 110 cm. Both subdetectors are completed by endcaps on either
side of the barrel, each consisting of two disks in the pixel tracker, and three small
plus nine large disks in the strip tracker. The endcaps extend the acceptance of the
tracker up to $|\eta|$ < 2.5. A three-dimensional schematic view of the tracker is shown in
Fig.~\ref{trk}, while in Fig.~\ref{fig_cmstracker} a pictorial representation of a slice of the tracker is displayed,
showing the various layers of the subdetectors.
The whole tracker has a cylindrical shape with a length of 5.8 m and a diameter
of 2.5 m, with the axis aligned to the beams direction. The average number of hits
per track is 12-14, allowing high reconstruction efficiency and low rate of fake tracks.

\begin{figure}
\centering
\includegraphics[scale= 0.5]{../Cap2/trk}
\caption{Three-dimensional schematic view of the CMS silicon tracker.}
\label{trk}
\end{figure}

\begin{figure}
\centering
\includegraphics[scale= 0.3]{../Cap2/fig_cmstracker}
\caption{Pictorial view of a tracker slice in the r-z plane. Pixel modules are shown in
red, single-sided strip modules are depicted as black thin lines and strip stereo modules are
shown as blue thick lines.}
\label{fig_cmstracker}
\end{figure}

\paragraph*{The Pixel Vertex Detector} The pixel vertex detector,  is mainly used in CMS as a starting point for
the reconstruction of tracks and is essential for the reconstruction of the primary vertex
(PV) and any possible secondary vertices. It is placed in the region closest to the collision
point, where the particle flow is maximum. It covers the region $|\eta|$ < 2.5 and is composed
of a central part (barrel) and by two forward parts (endcaps). The barrel is composed of
three concentric cylindrical sectors 53 cm long, located at an average distance r of 4.4 cm,
7.3 cm and 10.2 cm. Each half-cylinder is composed of ladders and half ladders that serve
as support and cooling structure for the modules of pixels, with each ladder containing 8
modules. In total, the barrel is composed of 768 modules. Each endcap is composed of
two disks placed at a distance of 34.5 cm and 46.5 cm from the nominal beams impact
point. They cover a radius $r$ in a range between 6 cm and 15 cm in such a way that each
track included in the detector acceptance passes through at least two layers. Each disk is
divided into 24 segments, on each of which 7 modules of different sizes are mounted, for
a total of 672 modules on all the endcaps. Each module is composed of several units that
contain a highly integrated and segmented silicon sensor with a thickness of 250 $\mu$m. In
order to optimize the reconstruction of vertices and the track parameters near the vertex,
a set of rectangular pixels with a size of 150 × 100 $\mu$m 2 are used, with the 100 $\mu$m side
oriented along the $r \phi$ direction in the barrel region and along the $z$ direction in the endcap
region. The resolution in the hit reconstruction is about 10-15 $\mu$m in the barrel and
about 15 $\mu$m in the endcaps.

\paragraph*{The Silicon Microstrip Detector} In the region of the detector that is more than 20 cm far from the beam, the flow of
particles is sufficiently limited to allow the use of a silicon microstrip detector (Silicon
Strip Tracker, SST). Overall, this detector consists of 15400 units (modules), composed
of one or two sensors sticked on a support of carbon fiber, together with the readout
electronics. In case of a “doubled” sensor, the second detector is rotated with respect
to the first one in order to have strips forming an angle of 100 mrad between them.
This “stereo” combination, although of lower resolution, is preferable compared to a pixel segmentation because it has a lower number of readout channels. 
The ambiguities due to
the hit recognition are resolved in the process of reconstruction of the entire track. The
silicon microstrip tracker is 5.4 m long, extending up to a distance of 1.1 m from the axis
of the beams. It consists of a barrel and two endcaps and it is divided into four distinct
parts, TIB and TOB, and TID and TEC.


\subsection*{The Electromagnetic Calorimeter (ECAL)}
The main function of an electromagnetic calorimeter is to identify electrons and photons
and to measure accurately their energy. The electromagnetic calorimeter (Fig.~\ref{ecal_all} ) of
CMS (ECAL, Electromagnetic CALorimeter )  is a homogeneous calorimeter with
cylindrical geometry, whose elements are scintillating crystals of lead tungstate (PbWO$_4$)
with a truncated pyramidal shape. It consists of an ECAL Barrel (EB) with 61200 crystals
and two ECAL Endcaps (EE) containing 7324 crystals each one.
Crystals are grouped into 5 × 5 matrices called towers. The barrel has an inner radius of
129 cm, a length of 630 cm and it extends in the region $|\eta|$ < 1.479. Crystals in the ECAL
barrel have the following dimensions: 22 × 22 mm$^2$ at the front face, 26 × 26 mm$^2$ at the
rear face, and a length of 23 cm, corresponding to 25.8 $X_0$ . Each submodule, consisting
in a 5 × 2 crystals arrays mounted on a glass fiber structure, forms the elementar unit of
EB. The granularity of a single crystal is about 1 grade. To avoid that cracks might align with
particle trajectories, the crystal axes are tilted by 3 grade with respect to the direction from
the interaction point, both in the $\eta$ and $\phi$ direction.
Each endcap covers the region 1.479 < $|\eta|$ < 3 and is formed by two semicircular halves
of aluminum called dees. Crystals in endcaps have a length of 22 cm and frontal area
equal to 28.6 × 28.6 mm$^2$ . They are arranged in supercrystals with 5 × 5 elementary unity.
Unlike the crystals in the barrel, arranged in a $\eta - \phi $ symmetry, the endcap crystals are
arranged according to a $xy$ geometry.
Two preshower detectors are placed in front of the endcaps in order to separate the
showers produced by a primary $\gamma$ from those produced by a primary $\pi_0$ . This detector,
which covers the region 1.653 <  $|\eta|$ < 2.6, is a sampling calorimeter and it consists of two
disks of lead converters  that start the electromagnetic
shower of the incident photon/electron, alternating with two layers of silicon microstrip
detectors in which a measurement of the released energy and the identification of the
shower profile are performed. The strips are arranged orthogonally in the two planes,
according to a $xy$ configuration.
\begin{figure}
\centering
\includegraphics[scale= 0.5]{../Cap2/ecal_all}
\caption{Schematic representation of the electromagnetic calorimeter ECAL.}
\label{ecal_all}
\end{figure}
The choice of the PbWO$_4$ crystals as scintillating material for ECAL is due to several
reasons. First, the high-density the short radiation length  and the reduced Molière radius 5 (R M = 2.2 cm) allow to build a compact and
high granularity calorimeter. Furthermore, the 15 ns decay scintillation time allows to
collect about 80\% of the emitted light during the 25 ns that exist between two consecutive
beam interactions in the LHC design performance. Finally, the PbWO$_4$ crystals have a
good intrinsic radiation hardness and therefore they can operate for years in the hard
LHC environment, with a modest deterioration in performance. The main disadvantage
of these crystals is the low light yield  which makes an internal
amplification for the photodetectors necessary. This is achieved through the use of silicon
avalanche photodiodes  in barrel and single stage photomultipliers  in the endcaps, both resistant to the radiation and to the strong
magnetic field of CMS.
The energy resolution of a homogeneous calorimeter is usually expressed by the sum
in quadrature of three terms, according to the formula,
\begin{equation}
\frac{\sigma_e}{E}=\frac{a}{\sqrt{E}} + \frac{b}{E} + c \; ,
\end{equation}
The stochastic term a is the dominant term at low energies: it includes the contribution
of statistical fluctuations in the number of photoelectrons generated and collected.
The noise term b includes contributions from the electronic noise, both due to the pho-
todetector and to the preamplifier, and from pileup events.
The constant term c, dominant at high energies, takes into account several contributions:
the stability of the operating conditions (in particular of temperature and voltage), the
presence of dead material in front of the crystals and the rear leakage of the electromag-
netic shower, the longitudinal non uniformity of the crystal light yield, the intercalibration
errors and the radiation damage of the crystals.
The ECAL barrel energy resolution for electrons in beam tests has been measured to: $a=2.8\%$ GeV$^{-1/2}$, $b=12\%$ GeV, $c$ 0.3\%, where the energy E is measured in GeV.

\subsection*{The Hadronic Calorimeter (HCAL)}
The hadronic calorimeter HCAL (Hadronic CALorimeter ), together with the electromagnetic calorimeter, 
makes a complete calorimetric system for the jet energy and
direction measurement. Furthermore, thanks to its tightness, it can provide a measurement 
of the features of non-interacting particles, such as neutrinos, by measuring the
missing energy deposited in the transverse plane, $E_T^{Miss}$ or MET. The CMS hadronic calorimeter is
a hermetic sampling calorimeter that covers the region $|\eta|$ < 5. As shown in Fig.~\ref{hcal},
it is divided into four subdetectors: HB (Barrel Hadronic Calorimeter ), located in the
barrel region inside the magnet, extending up to pseudorapidities  $|\eta|$ < 1.4; HE (Endcap
Hadronic Calorimeter ), situated in the endcaps region inside the magnet, extends in the
pseudorapidity region 1.3   $<|\eta|<$  3, partially overlapping the HB coverage; HO (Outer
Hadronic Calorimeter, also called tail-catcher, placed along the inner wall of the magnetic 
field return yoke, just outside of the magnet; HF (Forward Hadronic Calorimeter ),
a sampling calorimeter consisting of quartz fibers sandwiched between iron absorbers,
consisting of two units placed in the very forward region (3  $<|\eta|<$ 5) outside the magnetic coil. 
The quartz fibers emit Cherenkov light with the passage of charged particles
and this light is detected by radiation resistant photomultipliers.
In order to maximize particle containment for a precise missing transverse energy
measurement, the amount of absorber material was maximized, reducing therefore the
amount of the active material. Since HCAL is mostly placed inside the magnetic coil,
a non-magnetic material like brass was chosen as absorber. HB and HE are therefore
made with brass absorber layers interleaved with plastic scintillators (wavelength shifters,
WLS) coupled to transparent optical fibers, which transmit the light to the HPD (Hybrid
Photodiodes) photodetectors.
\begin{figure}
\centering
\includegraphics[scale= 0.4]{../Cap2/hcal}
\caption{A schematic $rz$ view of a quadrant of the CMS hadronic calorimeter HCAL.}
\label{hcal}
\end{figure}


\subsection*{The Solenoid Magnet}
The CMS magnet, which houses the tracker, the electromagnetic and the hadronic
calorimeters, is the biggest superconducting solenoid ever built in the world. The solenoid
achieves a magnetic field of 3.8 T in the free bore of 6 m in diameter and 12.5 m in length.
The energy stored in the magnet is about 2.6 GJ at full current. The superconductor is
made of four Niobium-Titanium layers. In case of a quench, when the magnet loses its
superconducting property, the energy is dumped to resistors within 200 ms. The magnet
return yoke of the barrel is composed with three sections along the z-axis; each one is
split into 4 layers (holding the muon chambers in the gaps). Most of the iron volume is
saturated or nearly saturated, and the field in the yoke is about the half (1.8 T) of the
field in the central volume.

\begin{figure}
\centering
\includegraphics[scale= 0.35]{../Cap2/magnet}
\caption{Arrival of the magnet in the tunnel on February 28, 2007.}
\label{hcal}
\end{figure}

\subsection*{The Muon chamber}
The CMS Muon System  is dedicated to the identification and measurement of high
$p_T$ muons, in combination with the tracker. Furthermore, it provides a time measurement
of the bunch-crossing and also works as trigger for events involving muons. Momentum
measurement, in the muon system, is determined by the muon bending angle at the exit
of the 3.8 T coil, considering the interaction point as the origin of the muon. Up to $p_T$
values of 200 GeV, the resolution of the muon system is dominated by multiple scattering
and the best resolution is rather given by the silicon tracker, , Fig~\ref{muon_res}. The system is placed outside
the magnetic coil, embedded in the return yoke, to fully exploit the 1.8 T return flux. It
consists of three independent subsystems, as shown in Fig.~\ref{muon_c}: drift tubes (DT), cathode strip chambers (CSC) and resistive plate
chambers (RPC). The DT and the CSC provide an excellent spatial resolution for the
measurement of charged particle momentum; the RPC are used for trigger issues because
of the very good timing. The active parts of the muon system are hosted into stations
which are interleaved by the iron layers of the return yoke of the magnet. 

\begin{figure}
\centering
\includegraphics[scale= 1.2]{../Cap2/muon_res}
\caption{Trend resolution for the muon systems. On the left the barrel zone. On the right the endcap.}
\label{muon_res}
\end{figure}
\begin{figure}
\centering
\includegraphics[scale= 0.4]{../Cap2/muon}
\caption{Schematic overview of the muon chambers.}
\label{muon_c}
\end{figure}


\subsection*{Trigger and Data Acquisition}
LHC can produce interactions at 40 MHz frequency, but only a small fraction of these
events can be written on disk. On one hand the speed at which data can be written
to mass storage is limited, on the other hand the vast majority of events produced is
not interesting, because it involves low transverse momentum interactions (minimum bias
events). Thus, a trigger system is needed to save interesting events at the highest possible
rate. The maximum rate of events written on disk is about 800 Hz. CMS has chosen a
two-level trigger system, consisting of a Level-1 Trigger (L1)  and a High Level Trigger
(HLT).
Level-1 trigger runs on dedicated processors, and accesses coarse level granularity information 
from calorimetry and muon system. A L1 Trigger decision has to be taken for
each bunch crossing within 3.2 $\mu$s. Its task is to reduce the data flow from 40 MHz to
about 100 kHz. The High Level Trigger is responsible for reducing the L1 output rate down to a maximum
rate of the order of 1 kHz. The HLT code runs on a farm of commercial processors and can access the full granularity information of all the subdetectors.

\section{Data recoiled and future plans}
The first proton beam circulated in the LHC on September 2008, after more than a decade of construction and installation.
An incident occurred between two magnets, causing the release of helium into the tunnel
and mechanical damage. After that, in March  2010 started the Run-I, a fruitful data taking era that lasted until
2012.  It was decided not to operate the LHC at its design parameters, and proton proton collisions
took place at a centre of mass energy of 7 TeV and 8 TeV. The amount of recoil data (in CMS) in this period is reported in Fig.~\ref{int_lumi_cumulative_pp_1}.
At the end of 2012, LHC operations halted for two-year due to long shutdown (LS1).
In 2015 with centre-of-mass energy of 13 TeV the proton proton collision restarted (Run-II) and in the 2016,  LHC was ready to deliver a large dataset to the experiments.
The data collected in 2016 are used in the high mass analysis that is the subjetc of this thesis. Overall, the data correspond to 35.9 fb$^{-1}$ that are data validated for the physics analyses, including the  dead times of the experiment. The 2016 LHC operations can be regrouped into several time-periods, 
labelled with a letter from A to H.
\begin{figure}
\centering
\includegraphics[scale= 0.4]{../Cap2/int_lumi_cumulative_pp_1}
\caption{Run-I ans Run-II integrated luminosity.}
\label{int_lumi_cumulative_pp_1}
\end{figure}
In the 2017 and 2018 the data collection has been continued and the total integrated luminosity for RUN-II is around $\sim$150 fb$^{-1}$. 
These are a very large amount of data thas has been achived from high energy physics collider.
In October 2018 the proton collision has been stopped, Fig.~\ref{beam}, and all operation (ion collision) will  halt in 2019 for a second long shutdown of $\sim$2.5 year (LS2) for the machine and experiments upgrade. The Run-III will start in 2021 with a energy of 14 TeV.
After that, the third long shutdown (LS3) starting in 2024 will conclude the Phase I of LHC
operations that started back in 2008.
The high-luminosity LHC (HL-LHC), or Phase II, staring in $\sim$2016
will represent an unprecedented way to study very rare phenomena at the LHC. The
machine is expected to deliver, during a decade of operations, an integrated luminosity
of about 3000 fb$^{-1}$, Fig.~\ref{lhcplan}.
\begin{figure}
\centering
\includegraphics[scale= 0.2]{../Cap2/beam}
\caption{Last proton-proton beam dump at the end of Run-II.}
\label{beam}
\end{figure}
\begin{figure}
\centering
\includegraphics[scale= 0.5]{../Cap2/lhcplan}
\caption{Schedule of LHC and HL-LHC operations.}
\label{lhcplan}
\end{figure}





