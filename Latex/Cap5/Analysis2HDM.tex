\newpage
\section{2HDM and MSSM interpretations}\label{sec:2HDM}
In this section the interpretation of this analysis in a Two-Higgs-doublet model (2HDM) and in some scenarios of the Minimal Supersymmetric extension to the Standard Model (MSSM) is described. \\


\subsection{Introduction to 2HDM and MSSM}

The 2HDM is a well motivated extension of the SM. It contains two Higgs doublets, from which a total of five Higgs bosons are predicted: Two CP-even bosons $h$ and $H$, a CP-odd boson $A$ and two charged bosons $H^\pm$. In most theories, $h$ exhibits the features of the SM Higgs boson, while $H$ is a CP-even Higgs boson at a higher mass. In this study, limits are calculated on the production cross section of the Higgs boson $H$ multiplied with the branching fraction of the decay into two $W$ bosons.\newline
The 2HDM comprises many free parameters. Two of these are of particular interest:
\begin{itemize}
\item $\tan\beta$: The ratio $\frac{v_u}{v_d}$ of the vacuum expectation values of the two Higgs doublets.
\item $\alpha$: The mixing angle of the two scalar Higgs bosons $h$ and $H$.
\end{itemize}
The quantity $\cos(\beta-\alpha)$ is also of interest, as the coupling of the heavy scalar Higgs boson $H$ to two vector bosons is proportional to this factor. In the decoupling limit, which occurs at $\cos(\beta-\alpha)=0$, all couplings become SM-like.\newline
A 2HDM of type-2 is considered in this analysis. Here up-type quarks couple to one doublet, while down-type quarks and leptons couple to the other doublet. The MSSM is a type-2 2HDM. On tree level only two parameters are left free. By convention, these parameters are chosen to be $\tan\beta$ and $m_{A}$, the mass of the pseudoscalar Higgs boson. The exclusion limits can be set in a two-dimensional plane as a function of these two parameters. Due to higher order diagrams additional free parameters occur. Benchmark scenarios are then used in order to constrain these parameters. Here two MSSM scenarios are used: The $m_{h}^{mod+}$ scenario and the hMSSM scenario \cite{Carena:2013ytb}.\newline
The analysis follows the same steps as described in sections \ref{sec:OF} and \ref{sec:SF}.

\subsection{Statistical inference}

The necessary model predictions for these scenarios are provided by the LHC Higgs Cross Section Working Group \cite{bsmhiggsxsecs}. For both MSSM scenarios the ggF cross sections have been computed with SusHi (v.1.4.1)\cite{Harlander:2012pb}. These cross sections include NLO supersymmetric QCD corrections and NNLO QCD corrections for the top quark contribution in the effective theory of a heavy top quark, as well as electroweak effects by light quarks. The masses of the Higgs bosons, their mixing, the branching fractions and the effective Yukawa couplings in the $m_{h}^{mod+}$ scenario are all calculated with FeynHiggs (v.2.10.2)\cite{Heinemeyer:1998yj, Heinemeyer:1998np, Degrassi:2002fi, Frank:2006yh, Hahn:2013ria}. For the hMSSM scenario the branching fractions are obtained from HDECAY (v.6.40)\cite{Djouadi:1997yw, Djouadi:2006bz}. The results for general 2HDM are obtained using the ggF cross sections computed with SusHi (v.1.5.0) and the branching fractions from 2HDMC (v.1.7.0)\cite{Rathsman:2011yv}. The VBF cross sections are calculated using an approximation. The BSM Higgs production cross sections for VBF, which are provided for different masses by the LHC Higgs Cross Section Working Group \cite{bsmhiggsxsecs2}, are taken and multiplied by $\cos^{2}(\beta-\alpha)$, resulting in VBF cross sections for a heavy CP-even Higgs boson.\newline
The exclusion limits obtained for the MSSM scenarios are displayed in the $m_{A}$-$\tan\beta$ plane. A fine grid is chosen in this plane, and for each point of this grid a maximum likelihood fit is performed after the $m_{A}$ and/or $\tan\beta$ dependent values of the model, such as cross sections and masses of the Higgs bosons are calculated. These fits are done using the asymptotic method. Performing a maximum likelihood fit in this manner is equivalent to a hypothesis test, where the signal hypothesis is tested against the SM-and-background hypothesis. The signal hypothesis for a combination of $m_{A}$ and $\tan\beta$ is excluded at $95\,\%$ confidence level. In the two-dimensional plane this limit is determined from interpolation between the points of the grid. The limits in the more general 2HDM are obtained in the same way, although a different parameter is chosen in place of $m_{A}$.
