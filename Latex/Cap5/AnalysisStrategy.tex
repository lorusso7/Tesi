
\textit{
%As described in Sec.~\ref{NSP}, the research of  new resonance $X$ is one of the mail goals of LHC. With a energy achieved of 13 TeV in the center of mass and the data collected in 2016, it is possible to lead searches in a vast range of mass. One of the main final state channel in in a couple of $W^+W^-$ bosons.
In this chapter the $X \to \mathrm{W^+W^-}\to2\ell2\nu$ analysis using 2016 data is reported. To increase the sensitivity of this search, the signal must be selected in the most efficient way by reducing the
presence of the backgrounds with a similar signature: the selection criteria are described in detail. I have participated at all stages of the analysis selection (signal simulation, categorization, background estimation, etc.)  and I have been responsible for the whole analysis.} 
%The ATLAS experiment has been done this kind of searches using the early 2016 statistic, 13.2 fb$^{-1}$ and the results are shown in Fig.~\ref{ATLAS-CONF-2016-074_fig}.
%With the full 2016 statistic, approximately $\sim 36$ fb$^{-1}$ is possible to investigate a wide range of masses and, if will be no evidence of high mass signal, is possible to set  tight upper limits on the possible cross section.

\section{Overview of the $X$ to $WW$ fully leptonic analysis }\label{sec:AnalysisStrategy_Intro}

The analysis strategy for the high mass search in the $X \to \mathrm{W^+W^-}\to2\ell2\nu$ final state must take into account the production modes of the new scalar, the main backgrounds, and the interference among the different processes. 

The main production mode for a Higgs-like particle over the all mass spectrum is the gluon-gluon fusion process. 
However the ratio of the VBF cross-section to the gluon-gluon fusion cross-section increases with $m_\mathrm{X}$ (see Fig.~\ref{prod}), making the VBF production mechanism more and more important.\\

Among the SM processes that have the same final state of the signal or a similar one the most important are non-resonant WW production, top pairs production, Drell-Yan. The WW production is an irreducible background and needs to be determined together with the signal in the fit. To estimate the other background processes, control regions are defined on data and compared to simulation. \\
 
The events are first divided according the flavour in the final state: 
\begin{itemize}
\item opposite-flavour final state, $e^{\pm} \mu^{\mp}$,
\item same-flavour final state, $e^+ e^-$ and  $\mu^+ \mu^-$. 
\end{itemize}
In the opposite-flavour final state four different jets categories are defined: the 0-jet, the 1-jet, the 2-jet and the VBF, Sec.~\ref{sec:OF}. 
In the same-flavour final state only the VBF category is considered. Indeed, only the VBF selection cuts are sufficiently tight to reduce the  overwhelming Z plus jets background to a manageable level, Sec.~\ref{sec:SF}. The jets categories improve the sensitivity of the analysis, because each category has different contributions from signal production modes and backgrounds.

The signal is interpreted in terms of the EWK singlet and MSSM models as described in Sec~\ref{NSP}. The Higgs boson width and lineshape is reweighted at generator level according to the parameters defined in the model. The interference effects between the signal produced via gluon-gluon fusion, the WW background also from gluon-gluon fusion, and SM Higgs boson, are expected to change the shape of the signal distribution and have been fully taken into account. 
A similar treatment is also applied for the interference between high mass signal produced via VBF, the WW plus two quarks background (emerging from the same initial state) and the SM Higgs generated with VBF production mechanism. In general, the interference becomes more and more important as the mass of $X$ increase and it is studied in detail in Sec~\ref{sec:signalModel}.
Finally, the interference between the $\mathrm{W^+W^-}\to2\ell2\nu$ and $\mathrm{ZZ}\to2\ell2\nu$ is negligible due to the different phase space characteristic of these processes.

\section{Discriminating variable}
This analysis is a shape analysis, meaning that after applying selection cuts, the events are not simply counted, but rather an histogram of a discriminating variable is filled from the data and fitted with the sum of signal and background templates, and finally the signal yield is extracted.
In principle, the variable with the best discriminating power would be the invariant mass of
the four lepton system, however, due to the presence of the neutrinos, it can not be measured.
Usually, in the the Higgs boson to $WW \to 2\ell 2\nu $, the variables used in the analysis are:
\begin{itemize}
\item the transverse mass, $m_T^H$, defined as,  
\begin{equation}
 m_T^H = \sqrt{2p_{\rm T}^{\ell\ell}\MET(1-\mathrm{cos}\Delta\phi(\ell\ell, \ptvecmiss))}
\end{equation}
where $\Delta\phi(\ell\ell, \ptvecmiss)$ is the azimuthal angle between the dilepton momentum and \ptvecmiss;
\item the di-lepton mass, $m_{\ell \ell}$.
\end{itemize}
However, both $m_T^H$ and $m_{\ell \ell}$, are not sensitive to different
signal mass hypothesis. For this reason a new variable, the visible transverse mass,  $m_T^I$, has been used.
This variable, studied for this specific analysis, is defined as the invariant mass of the four momentum resulting from the sum of the
two leptons four-momenta and the missing four-momentum: 
\begin{equation}
 m_T^I = \sqrt{ (p_\mathrm{\ell\ell} + \MET)^2 - (\vec{p}_\mathrm{\ell\ell} + \ptvecmiss)^2 \; .}
\end{equation}
The distribution of all the variables defined above can be compared in 
Fig.~\ref{fig:mt_nocuts}, where it is visible the better power of $m_T^I$ in discriminating different mass hypotheses wth respect to  $m_T^H$ and $m_{\ell \ell}$. The usage of  $m_T^I$ also provides a good discriminating power between signal and background.
\begin{figure}[htbp]
\centering
\subfigure[True generated mass]{
\includegraphics[width=0.45\textwidth]{../AN/Figs/Distribution_higgsLHEmass_cuts_nocuts.png}
}
\subfigure[$m_T^H$]{
\includegraphics[width=0.45\textwidth]{../AN/Figs/Distribution_mth_cuts_nocuts.png}
}
\\
\subfigure[$m_{\ell \ell}$]{
\includegraphics[width=0.45\textwidth]{../AN/Figs/Distribution_mll_cuts_nocuts.png}
}
\subfigure[$m_T^I$]{
\includegraphics[width=0.45\textwidth]{../AN/Figs/Distribution_mTi_cuts_nocuts.png}
}
\caption{
    Distributions of the generated mass (no possible reconstruction), $m_T^H$, $m_{\ell \ell}$ and  $m_T^I$
    variables for different $X$ mass hypothesis. It is clear that the most discriminating variable is $m_T^I$. }
    \label{fig:mt_nocuts}
\end{figure}

%#################


\section{Signal interpretation: EW singlet, 2HDM and MSSM}
\label{sec:signalModel}
The signal is interpreted in terms of the electroweak singlet model, in 2HDM and finally in  MSSM model. The theory part of the models are described in Sec.~\ref{NSP}. 

\subsection*{Electroweak singlet model}
The EW singlet represents a scalar mixing among the high mass particle and the Higgs boson. This model relies on two parameters: the scale factor of the couplings of the high mass resonance with respect to the SM, $C'$, and the branching fraction of the electroweak singlet to non-SM decays modes, $BR_\mathrm{new}$. The electroweak singlet signal strength, $\mu'$ and the modified width, $\Gamma'$, are related with the parameters in the model by the following equations:
\begin{equation}
\mu' = C'^2 \cdot (1 - BR_\mathrm{new})
\end{equation}
\begin{equation}
\Gamma' = \Gamma_\mathrm{SM} \cdot \frac{C'^2}{1 - BR_\mathrm{new}}
\end{equation}
The high mass signal samples for different mass hypothesis have been reweighted according to this model. At the moment only the $BR_\mathrm{new} = 0$ hypothesis has been investigated while we tested different $C'$ values.
In Fig.~\ref{fig:cprime} are shown the \mll and \mt templates corresponding to a high mass boson  of 700\GeV for three different $C'$ values: $C' = 1$, corresponding to the SM Higgs decay width, $C'=0.5$, corresponding to $\Gamma' = 2.5\cdot10^{-2}\,\Gamma_\mathrm{SM}$, and $C'=0.1$, corresponding to $\Gamma' = 10^{-2}\,\Gamma_\mathrm{SM}$. A value of $BR_\mathrm{new} = 0$ is considered in all cases. We note that the signal shape is not very sensitive to different $C'$ values.
\begin{figure}[htbp]
\centering
\subfigure[Simulated LHE signals]{
\includegraphics[width=0.45\textwidth]{../AN/Figs/higgsLHEmass700_cuts_nocuts.png}
}
\subfigure[$m_T^H$]{
\includegraphics[width=0.45\textwidth]{../AN/Figs/mth700_cuts_nocuts.png}
}
\\
\subfigure[$m_{\ell \ell}$]{
\includegraphics[width=0.45\textwidth]{../AN/Figs/mll700_cuts_nocuts.png}
}
\subfigure[$m_T^I$]{
\includegraphics[width=0.45\textwidth]{../AN/Figs/mTi700_cuts_nocuts.png}
}
\caption{ 
    Distributions of the signals, the $m_T^H$, the $m_{\ell \ell}$ and the  $m_T^I$ variables at generator level for different values of $C'$, without any selection.}
    \label{fig:cprime}
\end{figure}

\subsection*{2HDM and MSSM models}
The 2HDM is a well motivated extension of the SM. It contains two Higgs doublets, from which a total of five Higgs bosons are predicted: Two CP-even bosons $h$ and $H$, a CP-odd boson $A$ and two charged bosons $H^\pm$. In most theories, $h$ exhibits the features of the SM Higgs boson, while $H$ is a CP-even Higgs boson at a higher mass. The 2HDM comprises many free parameters. Two of these are of particluar interest:
\begin{itemize}
\item $\tan\beta$: The ratio $\frac{v_u}{v_d}$ of the vacuum expectation values of the two Higgs doublets.
\item $\alpha$: The mixing angle of the two scalar Higgs bosons $h$ and $H$.
\end{itemize}
The quantity $\cos(\beta-\alpha)$ is also of interest, as the coupling of the heavy scalar Higgs boson $H$ to two vector bosons is proportional to this factor. In the decoupling limit, which occurs at $\cos(\beta-\alpha)=0$, all couplings become SM-like.
A 2HDM of type-2 is considered in this study. Here up-type quarks couple to one doublet, while down-type quarks and leptons couple to the other doublet.\\ 
\newline
The MSSM is a type-2 2HDM. On tree level only two parameters are left free. By convention, these parameters are chosen to be $\tan\beta$ and $m_{A}$, the mass of the pseudoscalar Higgs boson. The exclusion limits can be set in a two-dimensional plane as a function of these two parameters. Due to higher order diagrams additional free parameters occur. Benchmark scenarios are then used in order to constrain these parameters. Here two MSSM scenarios are used: the $m_{h}^{mod+}$ scenario and the hMSSM scenario \cite{Gori:2130983}.\\
\newline
The necessary model predictions for these scenarios are provided by the LHC Higgs Cross Section Working Group \cite{bsmhiggsxsecs}. For both MSSM scenarios the ggF cross sections have been computed with SusHi (v.1.4.1)\cite{Harlander:2012pb}. These cross sections include NLO supersymmetric QCD corrections and NNLO QCD corrections for the top quark contribution in the effective theory of a heavy top quark, as well as electroweak effects by light quarks. The masses of the Higgs bosons, their mixing, the branching fractions and the effective Yukawa couplings in the $m_{h}^{mod+}$ scenario are all calculated with FeynHiggs (v.2.10.2)\cite{Heinemeyer:1998yj, Heinemeyer:1998np, Degrassi:2002fi, Frank:2006yh, Hahn:2013ria}. For the hMSSM scenario the branching fractions are obtained from HDECAY (v.6.40)\cite{Djouadi:1997yw, Djouadi:2006bz}. The results for general 2HDM are obtained using the ggF cross sections computed with SusHi (v.1.5.0) and the branching fractions from 2HDMC (v.1.7.0)\cite{Rathsman:2011yv}. The VBF cross sections are calculated using an approximation. The BSM Higgs production cross sections for VBF, which are provided for different masses by the LHC Higgs Cross Section Working Group \cite{bsmhiggsxsecs2}, are taken and multiplied by $\cos^{2}(\beta-\alpha)$, resulting in VBF cross sections for a heavy CP-even Higgs boson.\newline
The exclusion limits obtained for the MSSM scenarios are displayed in the $m_{A}$-$\tan\beta$ plane. A fine grid is chosen in this plane, and for each point of this grid a maximum likelihood fit is performed after the $m_{A}$ and/or $\tan\beta$ dependent values of the model, such as cross sections and masses of the Higgs bosons are calculated. These fits are done using the asymptotic method. Performing a maximum likelihood fit in this manner is equivalent to a hypothesis test, where the signal hypothesis is tested against the SM-and-background hypothesis. The signal hypothesis for a combination of $m_{A}$ and $\tan\beta$ is excluded at $95\,\%$ confidence level. In the two-dimensional plane this limit is determined from interpolation between the points of the grid. The limits in the more general 2HDM are obtained in the same way, although a different parameter is chosen in place of $m_{A}$.


\section{Study of the Interference effects}
\label{sec:interference}
When a resonance $X$, with a non negligible width is considered, it is important to take into account also the interference effects both with the \WW background , with same initial and final state, and with the Higgs boson off-shell tail. 
In this analysis  the interference effects between the
new signal X produced in gluon-gluon fusion and in vector-boson-fusion is taken into account.
The effect of the various interference terms are shown in ~\ref{fig:X300} and  \ref{fig:Int_VBF_GEN} for the two different production mechanism, gluon-gluon fusion and vector-boson fusion.  The contribution of the interference of high mass resonances $X$ with the \WW background  and with the Higgs boson have opposite sign and partially cancel out. This cancellation effect is different for different resonance masses.
The interference contribution is thus non negligible and is included in the fit, Sec~\ref{StatIn}. \\
\begin{figure}[htbp]
\centering
\subfigure[Mass 300 \GeV.]{
\includegraphics[width=0.45\textwidth]{../AN/Figs/Interference_higgsLHEmass300_cuts_nocuts.png}
}
\subfigure[Mass 400 \GeV.]{
\includegraphics[width=0.45\textwidth]{../AN/Figs/Interference_higgsLHEmass400_cuts_nocuts.png}
}
\\
\subfigure[Mass 700 \GeV.]{
\includegraphics[width=0.45\textwidth]{../AN/Figs/Interference_higgsLHEmass700_cuts_nocuts.png}
}
\subfigure[Mass 1500 \GeV.]{
\includegraphics[width=0.45\textwidth]{../AN/Figs/Interference_higgsLHEmass1500_cuts_nocuts.png}
}
\caption{Distribution of for the $X$ mass resonance, produced via gluon-gluon fusion for different masses. In black the high mass signal. In red the interference between the high mass signal and the Higgs boson. In blue the interference between the high mass signal and the background. In green the total interference i.e. high 
mass signal, Higgs bison and background. }
    \label{fig:X300}
\end{figure}


\begin{figure}[htbp]
\centering
\subfigure[Mass 300 \GeV]{
\includegraphics[width=0.45\textwidth]{../AN/Figs/Inter_VFB/Interference_higgsLHEmass300_cuts_nocuts.png}
}
\subfigure[Mass 1500 \GeV]{
\includegraphics[width=0.45\textwidth]{../AN/Figs/Inter_VFB/Interference_higgsLHEmass1500_cuts_nocuts.png}
}
\caption{Distribution of for the $X$ mass resonance, produced via vector-boson-fusion fusion for different masses. In black the high mass signal. In red the interference between the high mass signal and the Higgs boson. In blue the interference between the high mass signal and the background. In green the total interference i.e. high 
mass signal, Higgs bison and background.}
    \label{fig:Int_VBF_GEN}
\end{figure}


\section{Main Background processes}
\label{anbkg}
Inside the SM that are several processes that have the same or a similar final state of the signal, called background processes. The most important background processes contributing to this final state are non resonant $q\bar{q} \to W^+ W^-$, the top production ($t\bar{t}$ and single-top) and  the Drell-Yan process. 
Other background the $W$ and the $Z$ bosons.
All these processes have been simulated with Monte Carlo generators and the simulation details have been discussed in Sec~\ref{MSsample}.
Following, the description of the main different background processes:
\begin{itemize}
\item \textit{Non-resonant WW} ($q\bar{q} \to W^+ W^-$): this background is characterized by a final state identical to the signal, however the lepton kinematics for signal and $q\bar{q} \to W^+ W^-$ processes is rather different.
For the signal process, the W bosons originate from a spin-0 particle decay
and their spins must therefore be antiparallel, implying that the charged leptons pro-
duced in their decays appear preferentially in the same hemisphere~\cite{Ellis:2012wg}. In contrast,
there is no preferential spin direction in the background case. For this reason the
azimuthal angle difference between the two leptons is on average smaller for signal
than for background, resulting in a smaller dilepton invariant mass in the former case. The Feynman diagram of the process is reported below.
\begin{figure}[h]
\centering
\vspace{0.5cm}
\includegraphics[scale= 0.9]{../Cap5/nnr_WW}
\end{figure}
\item \textit{Top} ($t\bar{t}$ and single-top): the $t\bar{t}$ events can give a signal-like signature if the decay chain is $t \to Wb$ and subsequently $W \o \ell \nu$.  In such a case, in fact, there are two leptons and two jets (from the hadronization of the $b$ quark) in the final state. This process is especially important when the signal is produced via VBS or when the signal in produced with jets coming from initial or final state radiation. The single-top instead is characterized by the presence of $W$ boson and a top quark. Following, some examples of Feynman diagrams for the top background. The $t\bar{t}$ in (a) and (b) diagrams, the single-top in (c).
\begin{figure}
\centering%
\subfigure[]%
{\includegraphics[scale= 0.3]{../Cap5/ggtt_t}} \qquad 
\subfigure[]%
{\includegraphics[scale= 0.3]{../Cap5/qqtt_s}} \qquad 
\subfigure[]%
{\includegraphics[scale= 0.5]{../Cap5/st_bg}}
\end{figure}
\newpage
\item \textit{Drell-Yan}: the Drell-Yan process is defined as the annihilation of a quark-antiquark pair into a lepton-antilepton pair. his process is described at leading order by
the two Feynman diagrams drawn. These two amplitudes are proportional to the fine structure constant $\alpha \sim 1/137$.
This kind of background is particular important for the same flavour final state of the signal having two leptons  of same flavour.
\begin{figure}[h]
\centering
\vspace{0.5cm}
\includegraphics[scale= 0.7]{../Cap5/dy}
\end{figure}

\item \textit{W+jet}: this background is charaterized by a $W$ boson, decaying in $\ell \nu$ in addition to a jet. A fake lepton arise from the misidentified jet and lead the same final state of the signal (two leptons and missing-transverse-energy). 

\item \textit{Other}: other background processes involved the Z bosons, such as $WZ/W\gamma^*$, $ZZ^*$ with a misidentified lepton and $Z\gamma$ with $\gamma$ conversion.

\end{itemize}
The main background processes the \WW production and the top production  are estimated using data. 
Instrumental backgrounds arising from non-prompt leptons in $W+$jets production and mis-measurement of $E_T^{miss}$ in Drell-Yan events are also estimated from
data. The contribution from W$\gamma^*$  is estimated partly from data. The
contribution of other sub-dominant backgrounds is obtained directly from simulated samples. The different data-driven background estimations are explained in the following sections. More precisely top and  Drell-Yan backgrounds normalizations have been extracted
directly from data-simulation comparison in specific control regions enriched in either one
or the other background separately for the different events categories, using the rateParam feature of the combine package~\cite{combine}.


\section{Data sample and Triggers}
\subsection*{Data in CMS}
Data recorded in proton proton collisions at 13 TeV during all 2016 was used in the analysis, with a total integrated luminosity of  35.9 \fbinv.
The data has been reprocessed in the reprocessing campaign characterized by the submission date \textit{03Feb2017} in CMS.
In Table~\ref{tab:data} the different data streams used are presented. All runs are taken at 25~ns and recorded in seven different periods.
\begin{table}
\begin{center}
\begin{tabular}{|l|l|}
\hline
Data Taking Era & Stream\\
\hline
\multirow{5}{*}{Run2016C} 	& SingleMuon  \\
                                & DoubleMuon \\
				& SingleElectron \\
                                & DoubleEG \\
				& MuonEG \\ \hline
\multirow{5}{*}{Run2016D}       & SingleMuon  \\
                                & DoubleMuon \\
                                & SingleElectron \\
                                & DoubleEG \\
                                & MuonEG \\ \hline
\multirow{5}{*}{Run2016E}       & SingleMuon  \\
                                & DoubleMuon \\
                                & SingleElectron \\
                                & DoubleEG \\
                                & MuonEG \\ \hline
\multirow{5}{*}{Run2016F}       & SingleMuon  \\
                                & DoubleMuon \\
                                & SingleElectron \\
                                & DoubleEG \\
                                & MuonEG \\ \hline
\multirow{5}{*}{Run2016G}       & SingleMuon  \\
                                & DoubleMuon \\
                                & SingleElectron \\
                                & DoubleEG \\
                                & MuonEG \\ \hline
\multirow{5}{*}{Run2016H}       & SingleMuon  \\
                                & DoubleMuon \\
                                & SingleElectron \\
                                & DoubleEG \\
                                & MuonEG \\ \hline
\hline 
\end{tabular}
%\vspace{0.5cm}
\caption{Data samples used in the analysis. The total integrated luminosity corresponds to 35.9\fbinv.
\label{tab:data}}
\end{center}
\end{table}

\subsection*{Triggers}
In the high mass analysis final states with 2 leptons are studied, therefore a combination of both single and double lepton triggers  are used. For the electrons A combination of triggers is necessary to increase the statistics and to adjust the $p_T$ threshold requirement. Efficiencies are
measured using the Tag and Probe method described in App.~\ref{TP}. For the trigger efficiencies
calculation, it is applied the complete identification/isolation criteria on the leptons.
The results for the trigger efficiencies for electron triggers using Tag and Probe method are more 95\%.
Also for the calculation of the muon trigger efficiency the tag-and-probe method has been used.
The triggers used in the analysis are summarized in Table~\ref{tab:triggers} 
\begin{table}
\begin{center}
\begin{tabular}{|l|l|}
   \hline
   Dataset & HLT path \\
   \hline
   
   \multirow{2}{*}{SingleElectron} & HLT\_Ele45\_WPLoose\_Gsf\_v* \\
                                   & HLT\_Ele27\_eta2p1\_WPLoose\_Gsf\_v* \\
   \hline
   
   \multirow{2}{*}{SingleMuon}   & HLT\_IsoMu22\_v* \\
                                 & HLT\_IsoTkMu22\_v* \\
   \hline
   
   \multirow{2}{*}{MuonEG}       & HLT\_Mu8\_TrkIsoVVL\_Ele17\_CaloIdL\_TrackIdL\_IsoVL\_v*  \\
                                 & HLT\_Mu17\_TrkIsoVVL\_Ele12\_CaloIdL\_TrackIdL\_IsoVL\_v*  \\
   
   \hline
   
   \multirow{2}{*}{DoubleMuon}   & HLT\_Mu17\_TrkIsoVVL\_Mu8\_TrkIsoVVL\_v*  \\
                                 & HLT\_Mu17\_TrkIsoVVL\_TkMu8\_TrkIsoVVL\_v*  \\

   
   \hline
   
   \multirow{1}{*}{DoubleEG}   &    HLT\_Ele23\_Ele12\_CaloIdL\_TrackIdL\_IsoVL\_DZ\_v* \\
   
   \hline
\end{tabular}
\caption{HLT paths used in the analysis.
\label{tab:triggers}  }
\end{center}
\end{table}
In Figure~\ref{Fig:trigger} the trigger efficiency for a gluon fusion signal
with mass 300~\GeV is shown for electrons (left) and muons (right).
An average trigger efficiency greater than 99\% is found, as shown in Figure~\ref{Fig:triggerIntegral}. The triggers used in the analysis are summarized in Table~\ref{tab:triggers}. 
\begin{figure*}[htbp]
\centering
\begin{tabular}{cc}
%  \includegraphics[width=0.45\textwidth]{Figs/Trigger/ele.png} &
%  \includegraphics[width=0.45\textwidth]{Figs/Trigger/mu.png} \\
%  (a) electron & (b) muon \\
 \includegraphics[width=0.45\textwidth]{../AN/Figs/Trigger/ele1.png} &
 \includegraphics[width=0.45\textwidth]{../AN/Figs/Trigger/mu1.png} \\
 (a) electron 1st & (b) muon 1st \\
 \includegraphics[width=0.45\textwidth]{../AN/Figs/Trigger/ele2.png} &
 \includegraphics[width=0.45\textwidth]{../AN/Figs/Trigger/mu2.png} \\
 (c) electron 2nd & (d) muon 2nd \\
\end{tabular}
\caption{
      Trigger efficiency per event
      as a function of the lepton \pt
      for electrons (a) where leading lepton is an electron, 
      and (c) where trailing lepton is an electron, 
      and muons (b) where leading lepton is a muon, 
      and (d) where trailing lepton is an muon, 
      for a gluon fusion 300~\GeV MC sample.
      In this plots the other lepton not shown is
      integrated.
      An average trigger efficiency greater than 99\% is found.      
     }
    \label{Fig:trigger}
\end{figure*}
\begin{figure*}[htbp]
\centering
 \includegraphics[width=0.45\textwidth]{../AN/Figs/Trigger/triggW.png}
\caption{
      Trigger efficiency distribution for 
      four MC samples corresponding to masses of 300, 500, 1000 and 3000~\GeV.
      An average trigger efficiency greater than 99\% is found.
     }
    \label{Fig:triggerIntegral}
\end{figure*}


