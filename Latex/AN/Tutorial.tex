% Customizable fields and text areas start with % >> below.
% Lines starting with the comment character (%) are normally removed before release outside the collaboration, but not those comments ending lines

% svn info. These are modified by svn at checkout time.
% The last version of these macros found before the maketitle will be the one on the front page,
% so only the main file is tracked.
% Do not edit by hand!
\RCS$Revision: 425573 $
\RCS$HeadURL: svn+ssh://svn.cern.ch/reps/tdr2/notes/AN-17-243/trunk/AN-17-243.tex $
\RCS$Id: AN-17-243.tex 425573 2017-09-13 18:21:01Z alverson $
%%%%%%%%%%%%% local definitions %%%%%%%%%%%%%%%%%%%%%
% This allows for switching between one column and two column (cms@external) layouts
% The widths should  be modified for your particular figures. You'll need additional copies if you have more than one standard figure size.
\newlength\cmsFigWidth
\ifthenelse{\boolean{cms@external}}{\setlength\cmsFigWidth{0.85\columnwidth}}{\setlength\cmsFigWidth{0.4\textwidth}}
\ifthenelse{\boolean{cms@external}}{\providecommand{\cmsLeft}{top\xspace}}{\providecommand{\cmsLeft}{left\xspace}}
\ifthenelse{\boolean{cms@external}}{\providecommand{\cmsRight}{bottom\xspace}}{\providecommand{\cmsRight}{right\xspace}}
%%%%%%%%%%%%%%%  Title page %%%%%%%%%%%%%%%%%%%%%%%%
\cmsNoteHeader{AN-17-243} % This is over-written in the CMS environment: useful as preprint no. for export versions
% >> Title: please make sure that the non-TeX equivalent is in PDFTitle below
\title{Search for high mass resonances using WW fully leptonic decays with 2016 data}

% >> Authors
%Author is always "The CMS Collaboration" for PAS and papers, so author, etc, below will be ignored in those cases
%For multiple affiliations, create an address entry for the combination
%To mark authors as primary, use the \author* form
\address[neu]{Northeastern University}
\address[fnal]{Fermilab}
\address[cern]{CERN}
\author[cern]{The CMS Collaboration}

% >> Date
% The date is in yyyy/mm/dd format. Today has been
% redefined to match, but if the date needs to be fixed, please write it in this fashion.
% For papers and PAS, \today is taken as the date the head file (this one) was last modified according to svn: see the RCS Id string above.
% For the final version it is best to "touch" the head file to make sure it has the latest date.
\date{\today}

% >> Abstract
% Abstract processing:
% 1. **DO NOT use \include or \input** to include the abstract: our abstract extractor will not search through other files than this one.
% 2. **DO NOT use %**                  to comment out sections of the abstract: the extractor will still grab those lines (and they won't be comments any longer!).
% 3. For PASs: **DO NOT use tex macros**         in the abstract: CDS MathJax processor used on the abstract doesn't understand them _and_ will only look within $$. The abstracts for papers are hand formatted so macros are okay.
\abstract{
\PAz   This is an example of a \textit{CMS Note} written in \LaTeX
    using the \emph{cms-tdr} document class and processed using the
    same \texttt{tdr} perl script used in generating the CMS Physics TDRs.
    Instructions for producing CMS Notes and Internal Notes are given.
}

% >> PDF Metadata
% Do not comment out the following hypersetup lines (metadata). They will disappear in NODRAFT mode and are needed by CDS.
% Also: make sure that the values of the metadata items are sensible and are in plain text:
% (1) no TeX! -- for \sqrt{s} use sqrt(s) -- this will show with extra quote marks in the draft version but is okay).
% (2) no %.
% (3) No curly braces {}.
\hypersetup{%
pdfauthor={George Alverson, Lucas Taylor, A. Cern Person},%
pdftitle={Search for high mass resonances using WW fully leptonic decays with 2016 data},%
pdfsubject={CMS},%
pdfkeywords={CMS, physics, software, computing}}

\maketitle %maketitle comes after all the front information has been supplied
% >> Text
%%%%%%%%%%%%%%%%%%%%%%%%%%%%%%%%  Begin text %%%%%%%%%%%%%%%%%%%%%%%%%%%%%
%% **DO NOT REMOVE THE BIBLIOGRAPHY** which is located before the appendix.
%% You can take the text between here and the bibiliography as an example which you should replace with the actual text of your document.
%% If you include other TeX files, be sure to use "\input{filename}" rather than "\input filename".
%% The latter works for you, but our parser looks for the braces and will break when uploading the document.
%%%%%%%%%%%%%%%
\section{CMS papers}

There are currently three kinds of CMS papers supported by this system in addition to tdrs: ``CMS Analysis Note,''
 ``CMS Physics Analysis Summary," and ``CMS Paper."
The processing for these differs only in the header of the first page,
which includes a different PDF figure  for each kind. The
appropriate header is chosen by the switch used in the
\texttt{tdr}
command.

This document only deals with papers set with Pdf\LaTeX. We found Pdf\LaTeX\ plus \texttt{cvs} to be a reliable system for the production of
large documents such as the Physics TDRs and felt it would be useful to extend it to the production of shorter documents such as CMS Notes. As of 2010 \texttt{cvs} has been replaced by subversion (\texttt{svn}).

\subsection{The mechanics of generating and typesetting papers}

To start
you will need to request a note directory in the \texttt{svn} repository from the TDR manager (currently \href{mailto:George.Alverson@cern.ch}{George Alverson}
or \href{Lucas.Taylor@cern.ch}{Lucas Taylor}). It is best to supply a list of the lxplus usernames of the co-authors who are to have write access to the repository at the time of the request.

To generate output, check out your note
directory from \texttt{svn} following the example below. The \texttt{tag} below is the identifier for your paper, typically of the form XXX-YY-NNN.
Following the sequence below will populate your local copy of the repository  with only your note and not include the other notes. If you have a note, use ``note".
For a paper, use ``paper." [Notes: (1)~When running without Kerberos authentication, use svn+ssh://username@svn.cern.ch. (2)~At Fermilab, even using \texttt{kinit user@CERN.CH} is not sufficient without specifying a specific svn server node (i.e., 137.138.229.205) instead of svn.cern.ch.]

\begin{verbatim}
> svn co -N svn+ssh://svn.cern.ch/reps/tdr2 myDir
> cd myDir
> svn update utils
> svn update -N [papers|notes]
> svn update [papers|notes]/XXX-YY-NNN
# use the following line for tcsh...
# ..use -sh for bash:
> eval `[papers|notes]/tdr runtime -csh`
> cd [papers|notes]/XXX-YY-NNN/trunk
# (edit the template, then to build the document)
> tdr  --style=[paper|pas|an|note] b XXX-YY-NNN
\end{verbatim}

The \texttt{nodraft} switch is required to turn off the ``Draft" overlay text.

If you wish to export your paper (for local work or for security), you can produce a tarball with all the necessary files with
\begin{verbatim}
> tdr --style=note --export b mynote.
\end{verbatim}
This will function on  Unix or Windows systems which have recent copies of \LaTeX\ (including \AmS-\LaTeX) and \texttt{perl} installed. We currently use the \texttt{sectsty, subfig,
fancyhdr, mathpazo, rotating, fancybox, lineno, longtable, ifthen} and \texttt{natbib} styles, which may not be included in the default distribution, plus our own versions of \texttt{pdfdraftcopy} and the \texttt{pennames} particle name macros. The latter has been modified for use with the fonts required by our standard style and also to provide for automatic switching to an italic version when necessary.

\section{\texttt{svn} commands}
\texttt{svn} is similar in many ways to \texttt{cvs}. Once a repository has been checked out, the workflow is
almost identical except for tagging. In \texttt{svn}, tagging is done by creating a new directory branch using
the \texttt{svn copy} command. Please see the \texttt{svn} \href{http://svnbook.red-bean.com/en/1.5/svn-book.pdf}{manual} for details, particularly
the chapter on \href{http://svnbook.red-bean.com/en/1.5/svn-book.pdf#svn.branchmerge}{branching and tagging} and \href{http://svnbook.red-bean.com/en/1.5/svn-book.pdf#svn.forcvs}{svn for cvs users}. Please do not change the depth of the directory structure to the top-level \TeX\ file for your document.

\emph{Please make sure to configure your svn client:} edit \texttt{\~/.subversion/config} so that it appropriately tags pdf and other commonly used file types.
\begin{verbatim}
[auto-props]
*.pdf = svn:mime-type=application/pdf
*.png = svn:mime-type=image/png
*.jpg = svn:mime-type=image/jpeg
*.tex = svn:eol-style=native
*.eps = svn:mime-type=application/postscript
\end{verbatim}
There are other useful settings as well. For example, to stop \texttt{svn} from asking to commit backup files and object files, you can set the \texttt{global-ignores} flag:
\begin{verbatim}
[miscellany]
global-ignores = *.o *.bak
\end{verbatim}


\section{Document layout}

\subsection{Standard macros}
\textit{Notes} will automatically include
\texttt{ptdr-definitions.sty}, which provides definitions
for many physics and CMS-related entities, \eg, \GeVcc. These are discussed in more detail in section~\ref{sec:CMSmacros}, and a complete list is
given the Appendix.%~\ref{app:symdef}.

All style-related parameters are set in the
class file included by the script and generally follow the article style. The chapter command is not implemented.

\subsection{Title block}
Please see the \LaTeX\ \href{https://svnweb.cern.ch/cern/wsvn/tdr2/papers/XXX-08-000/trunk/XXX-08-000.tex}{source} for this file to see how the title
page is generated. In general it follows the normal \LaTeX\ practice for title pages.

The type of note (PAS, AN, Note, etc.) is set
through the \texttt{--style} switch in the \texttt{tdr} script. When in draft mode, the string ``Draft" is displayed on the page and the title block contains (in addition to the date), information about the svn status of the document and the PDF metadata.

For ANs which need to differentiate between primary and non-primary authors, using the star form of the author macro will add a footnote to indicate a primary author: \linebreak\verb|\author*{A. Cern Person}|.

\subsection{Page size, margins and fonts}

The standard European  paper size is A4 (210\unit{mm} x 297\unit{mm} ($8.3''$ x
$11.7''$)) while American paper is US Letter (216\unit{mm} x 279\unit{mm} ($8.5''$ x
$11.0''$)), somewhat wider and shorter. In the era of straight
PostScript this led to difficulties, but PDF print drivers now
generally supply a  ``shrink and center" option. In this template
we have set the \LaTeX\ page style parameters to match the standard A4 size
(see Table~\ref{tab:page_layout}) and rely upon that option to
produce an acceptable result on US Letter paper.

Do not override the default fonts. They are currently set to be
Palatino and Helvetica. The math fonts have also been changed to
Palatino so that they do not clash with the body text,
particularly in regards to numbers and units. This means the
authors should use \verb|\text| commands to put text in subscripts
and superscripts, and most importantly \emph{do not use}
\verb|\rm| in formulas with Greek symbols, otherwise you will end up with formulae looking like the second one below.

\begin{gather}
\phi = \text{a Greek letter}\\
{\rm \phi} = \text{a mistake}
\end{gather}

Also note that the math fonts include a full set of Greek symbols in Math Italic Bold (produced with \verb|\mathbold|),
but only uppercase in Math Bold (\verb|\mathbf|). Use either \verb|\boldsymbol| or \verb|\boldmath| outside the math delimiters (\$) (but inside braces) to get bold symbols. Compare:

 \begin{tabular}{ll}
\verb|$\mathbold{\Psi \alpha}$|& $\mathbold{\Psi \alpha}$ \\
 \verb|$\mathbf{\Psi \alpha}$|& $\mathbf{\Psi \alpha}$ \\
 \verb|$\Psi \otimes \beta$|& $\Psi \otimes \beta$ \\
%  \ifthenelse{\boolean{cms@external}}{}{\verb|{\boldmath{$\Psi \otimes \beta$}}|& {\boldmath{$\Psi \otimes \beta$}} \\}
 \verb|$\boldsymbol{\Psi \otimes \beta}$|& $\boldsymbol{\Psi \otimes \beta}$.
 \end{tabular}

Note, however, that \verb|\mathbold| will not work for most journal styles.

When Greek or symbol characters are used in the title, author, keywords or section heads, please use the \verb|\texorpdfstring|
command to provide alternate versions. Acrobat cannot deal with \TeX\ characters and will ignore many of them for your PDF bookmark. See the following two subsections and check the corresponding bookmarks. (You may notice that this will produce four instances of ``Package hyperref Warning: Token not allowed in a PDFDocEncoded string" in the output log.)

\subsection{\texorpdfstring{H$_\text{2}$O-$\alpha$}%
{Water-alpha} Demo}
The title for this subsection was set with \\
\verb|\subsection{\texorpdfstring{H$_\text{2}$O-$\alpha$}|\-\verb|{Water-alpha}}|\\
The use of \verb|\text| sets the numeral 2 in the same font and weight as the rest of the title (here Helvetica bold).

\subsection{H$_2$O-$\alpha$ demo}
The title for this subsection was set with\\
\verb|\subsection{H$_2$O-$\alpha$}|.

\subsection{Tables, figures\label{sec:tab-fig}}

Place the captions above the object for tables and use \texttt{topcaption}, below for figures using \texttt{caption}. To force a full width figure or table in the two-column mode of most journal reprint formats, use \verb|\textwidth| as the unit along with the starred versions of the commands:

\begin{verbatim}
\begin{figure*}[hbtp]\begin{center}
 \includegraphics[width=0.95\textwidth]{CMS-bw-logo}
 \caption{Figures inserted using includegraphics.}
  \label{fig:ex1}\end{center}\end{figure*}
\end{verbatim}

\begin{table*}[htbH]
\begin{center}
\topcaption{An example table: Current page and paragraph layout
parameters. ( 72.27\,pt = 1\,in )\label{tab:page_layout}}
\begin{tabular}{lclc}
\hline \verb|\hoffset| & \the\hoffset & \verb|\voffset| & \the\voffset \\
\verb|\textheight| & \the\textheight & \verb|\textwidth| & \the\textwidth \\
\verb|\baselineskip| & \the\baselineskip & \verb|\marginparsep| & \the\marginparsep \\
\verb|\topmargin| & \the\topmargin &&\\
\verb|\headheight| & \the\headheight & \verb|\footskip| & \the\footskip \\
\verb|\oddsidemargin| & \the\oddsidemargin & \verb|\evensidemargin| & \the\evensidemargin \\
\verb|\columnwidth| & \the\columnwidth&\verb|\linewidth|&\the\linewidth\\
\hline
\end{tabular}
\end{center}
\end{table*}


Figures can include PDF files using the
\texttt{includegraphics} package, which is automatically installed
by our class file. A nice feature is that if a file extension is not supplied, \texttt{includegraphics} supplies
an appropriate one based on whether the file is being Pdf\LaTeX{}ed or just \LaTeX{}ed. The package also can accept
sizes to which the figures will be scaled. Specifying both width and height forces both
dimensions to be changed and causes a distortion of the figure, however, so
only use one of the two. Don't try to use scaling to correct a bad original aspect ratio. If neither width nor height is given, the size is
taken from the Crop Box size embedded in the file, which is similar to the
BoundingBox in PostScript. If there is too much white space around your figure, it may be that
the Crop Box has been mis-set during a conversion from PostScript to PDF. Recommended
translators on lxplus are \texttt{epstopdf} and \texttt{ps2pdf -dEPSCrop}. Native PostScript can not be included.

The \texttt{subfig} package is included and can be used for PASs and ANs (but not papers) to generate (a), (b), \etc labels under the subfigures through the use of the \texttt{subfloat} command. We have aliased \texttt{subfigure} to \texttt{subfloat} to avoid breaking older documents which may have depended on the \texttt{subfigure} package, but the spacing will not necessarily be the same. You may need to add line breaks by hand.


%This figure will be side-by-side in CMS mode and stacked for two-column journal layout. The labels track.
%If the figures are of equal width and fill the page across the full width (width= ~0.45 \textwidth each) the use of \cmsFigWidth is not necessary.
\begin{figure}[hbtp]
  \begin{center}
    \includegraphics[width=\cmsFigWidth]{CMS-bw-logo}
    \includegraphics[width=\cmsFigWidth]{CMScol}
 %   to generate (a) and (b) labels under the figures, you can use subfloat, but this is not recommended: takes too much space
 %   \subfloat[]{\includegraphics[width=0.2\textwidth]{CMS-bw-logo}}\subfloat[]{\includegraphics[width=0.2\textwidth]{CMScol}}
    \caption{Figures inserted using includegraphics. (\cmsLeft) Black and white. (\cmsRight) Color.}
    \label{fig:ex1}
  \end{center}
\end{figure}

When including root-generated figures, please make sure to use the standard macro to set the figure parameters,
and to first generate the output in eps format which is then converted to PDF. The macro for TDR styles, \texttt{tdrstyle.C}, is available in the \texttt{utils/general} directory. For producing standard CMS figures for publication, the additional files \texttt{CMS\_lumi.h}, and \texttt{CMS\_lumi.C} are also present, as well as an example, \texttt{myMacro.C} and \texttt{histo.root}. Instructions for their proper use are currently available at \url{https://ghm.web.cern.ch/ghm/plots}.

The non-vector file types png and jpg are also picked up if present. Vector graphics is preferred except in cases such as scatter plots with millions of points. A screen grab saved as pdf is not vector graphics. In all cases, figures intended for publication should be publication quality.

As a result of the file-tracking we use for export, please keep the length of the graphics files (including any subdirectory names and the period plus extension, which is not normally entered) shorter than 65 characters.

%------------------------------------------------------------------------------
\section{Standards\label{sec:standards}}
Please check the
\textit{\href{https://twiki.cern.ch/twiki/bin/view/CMS/Internal/PubGuidelines}{CMS
Guidelines for Authors}} and the
\textit{\href{https://svnweb.cern.ch/cern/wsvn/tdr2/utils/trunk/general/notes_for_authors.pdf}{Notes
for TDR authors}} for authoritative information on CMS standards
for publications and for tips on writing in \LaTeX. (If you find any discrepancies between those
documents and the practices in this example, please contact
\href{mailto:George.Alverson@cern.ch}{us}.)

\subsection{Math}
\textit{Notes} include the \AmS-\LaTeX\ class file which defines
many additional math symbols, including \verb|\gtrsim|
($\gtrsim$). It also allows for better control in setting
equations. Please see the \AmS-\LaTeX\
\href{ftp://ftp.ams.org/pub/tex/doc/amsmath/amsldoc.pdf}{user
guide} for complete details.

As previously mentioned, uniformity of symbol use should be enforced through  use of the definitions  in \texttt{ptdr-definitions}.\\

\subsection{Figure style}
Figures must have legible axis labels and values, symbol names, and line types when read at the final design size. For tdr-style documents, this is enforced through the use of the root macro file, tdrstyle.C, as discussed in Section~\ref{sec:tab-fig}.

\subsection{Particle names: \texorpdfstring{\PZz}{Z-0} to \texorpdfstring{\PJgy}{J/psi}}
Most standard particle names can be typeset using the  the \texttt{pennames-pazo} package, which is an implementation of the PENNAMES (Particle Entity Names) scheme adapted by us for use with Palatino/mathpazo fonts, as far as possible. The advantage is that the formatting will mostly adhere to particle naming conventions for typesetting (no, particle names are not mathematical symbols--- they're more like units).

\subsection{CMS macros\label{sec:CMSmacros}}
Macros introduced by CMS are listed in Appendix~\ref{app:symdef}. The macros for units are particularly useful, especially as the include the proper spacing between the magnitude and the unit (a thinspace), and they have an xspace at the end, which removes the necessity of ending them with a pair of braces. Thus, use \verb|a momentum of 5\TeVc was measured| to produce ``a momentum of 5\TeVc was measured."

\section{Submitting a note}

Please follow the rules and procedures defined on the
\href{http://cms.cern.ch/iCMS/jsp/iCMS.jsp?mode=single&part=publications}{iCMS
server} or on the CMS wiki page for analysis notes and other CMS note types. For PAS documents or papers intended for journals, the \href{http://cms.cern.ch/iCMS/jsp/analysis/admin/analysismanagement.jsp}{CADI} analysis management page controls submission.

\section{References example}

References (\cite{CMS_AN_2006-027,CMS_NOTE_2006-106, Brandt:1997gi, Buchmuller:1986zs,PTDR2,CMS-PAS-JME-10-003,LEPewkfits,Bertram:2000br,Khachatryan:2010mp, RooStats,nu2}) should use standard
BibTeX citations and be placed in a separate bib file. This is automatically included by the \verb+\bibliograph{auto_generated}+ command placed at the end of the note.
We recommend the use of inspirehep.net (SPIRES)
identifiers as reference keys, where possible. This allows the reference to be easily found on Spires using the \textit{find texkey}
command. It also ensures uniqueness if the references are to be combined into a larger bib file later. Note, however, that Spires tends to classify all bibliographic entities as Articles. Entities such as arXiv postings do not have an associated journal, though, and should be entered in the bib file as Unpublished. See the bib file for this note for examples, including the correct use of hyperlinks (all references should be linked when possible). Some journal styles will lowercase the titles in references, so use curly braces (\verb|{}|) to escape proper names and the like. Don't escape the entire title gratuitously.
Direct references (\eg, see Ref.~\citenum{LEPewkfits}), may use the \verb|\citenum| form of \verb|\cite|.

% >> acknowledgments (for journal papers)
% Please include the latest version from https://twiki.cern.ch/twiki/bin/viewauth/CMS/Internal/PubAcknow.
%\begin{acknowledgments}...ack-text...\end{acknowledgments}
% This will normally be updated with the text available at the time of submission,
% so please add a comment if it has been modified. Such modifications need to be
% approved.
%
\begin{acknowledgments}
\hyphenation{Bundes-ministerium Forschungs-gemeinschaft Forschungs-zentren} We congratulate our colleagues in the CERN accelerator departments for the excellent performance of the LHC and thank the technical and administrative staffs at CERN and at other CMS institutes for their contributions to the success of the CMS effort. In addition, we gratefully acknowledge the computing centers and personnel of the Worldwide LHC Computing Grid for delivering so effectively the computing infrastructure essential to our analyses. Finally, we acknowledge the enduring support for the construction and operation of the LHC and the CMS detector provided by the following funding agencies: the Austrian Federal Ministry of Science, Research and Economy and the Austrian Science Fund; the Belgian Fonds de la Recherche Scientifique, and Fonds voor Wetenschappelijk Onderzoek; the Brazilian Funding Agencies (CNPq, CAPES, FAPERJ, and FAPESP); the Bulgarian Ministry of Education and Science; CERN; the Chinese Academy of Sciences, Ministry of Science and Technology, and National Natural Science Foundation of China; the Colombian Funding Agency (COLCIENCIAS); the Croatian Ministry of Science, Education and Sport, and the Croatian Science Foundation; the Research Promotion Foundation, Cyprus; the Ministry of Education and Research, Estonian Research Council via IUT23-4 and IUT23-6 and European Regional Development Fund, Estonia; the Academy of Finland, Finnish Ministry of Education and Culture, and Helsinki Institute of Physics; the Institut National de Physique Nucl\'eaire et de Physique des Particules~/~CNRS, and Commissariat \`a l'\'Energie Atomique et aux \'Energies Alternatives~/~CEA, France; the Bundesministerium f\"ur Bildung und Forschung, Deutsche Forschungsgemeinschaft, and Helmholtz-Gemeinschaft Deutscher Forschungszentren, Germany; the General Secretariat for Research and Technology, Greece; the National Scientific Research Foundation, and National Innovation Office, Hungary; the Department of Atomic Energy and the Department of Science and Technology, India; the Institute for Studies in Theoretical Physics and Mathematics, Iran; the Science Foundation, Ireland; the Istituto Nazionale di Fisica Nucleare, Italy; the Ministry of Science, ICT and Future Planning, and National Research Foundation (NRF), Republic of Korea; the Lithuanian Academy of Sciences; the Ministry of Education, and University of Malaya (Malaysia); the Mexican Funding Agencies (CINVESTAV, CONACYT, SEP, and UASLP-FAI); the Ministry of Business, Innovation and Employment, New Zealand; the Pakistan Atomic Energy Commission; the Ministry of Science and Higher Education and the National Science Centre, Poland; the Funda\c{c}\~ao para a Ci\^encia e a Tecnologia, Portugal; JINR, Dubna; the Ministry of Education and Science of the Russian Federation, the Federal Agency of Atomic Energy of the Russian Federation, Russian Academy of Sciences, and the Russian Foundation for Basic Research; the Ministry of Education, Science and Technological Development of Serbia; the Secretar\'{\i}a de Estado de Investigaci\'on, Desarrollo e Innovaci\'on and Programa Consolider-Ingenio 2010, Spain; the Swiss Funding Agencies (ETH Board, ETH Zurich, PSI, SNF, UniZH, Canton Zurich, and SER); the Ministry of Science and Technology, Taipei; the Thailand Center of Excellence in Physics, the Institute for the Promotion of Teaching Science and Technology of Thailand, Special Task Force for Activating Research and the National Science and Technology Development Agency of Thailand; the Scientific and Technical Research Council of Turkey, and Turkish Atomic Energy Authority; the National Academy of Sciences of Ukraine, and State Fund for Fundamental Researches, Ukraine; the Science and Technology Facilities Council, UK; the US Department of Energy, and the US National Science Foundation.

Individuals have received support from the Marie-Curie program and the European Research Council and EPLANET (European Union); the Leventis Foundation; the A. P. Sloan Foundation; the Alexander von Humboldt Foundation; the Belgian Federal Science Policy Office; the Fonds pour la Formation \`a la Recherche dans l'Industrie et dans l'Agriculture (FRIA-Belgium); the Agentschap voor Innovatie door Wetenschap en Technologie (IWT-Belgium); the Ministry of Education, Youth and Sports (MEYS) of the Czech Republic; the Council of Science and Industrial Research, India; the HOMING PLUS program of Foundation for Polish Science, cofinanced from European Union, Regional Development Fund; the Compagnia di San Paolo (Torino); the Consorzio per la Fisica (Trieste); MIUR project 20108T4XTM (Italy); the Thalis and Aristeia programs cofinanced by EU-ESF and the Greek NSRF; and the National Priorities Research Program by Qatar National Research Fund.
\end{acknowledgments}
%% **DO NOT REMOVE BIBLIOGRAPHY**
\bibliography{auto_generated}   % will be created by the tdr script.

%% examples of appendices. **DO NOT PUT \end{document} at the end
%\clearpage
\appendix
\section{PTDR symbol definitions\label{app:symdef}}
If absolutely necessary, symbol definitions may
be over-ridden using the \verb|\renewcommand| command. If you don't want to over-ride the default version of a command but provide it for use outside the normal tdr system, please use the \verb|\providecommand| command.

%
% Below is a multi-column format to display the standard PTDR symbols.
% Feel free to delete the lines from here to the end and the pdefs.tex file once you start modifying the template. The actual definitions are
% pulled in by the \def\Fileversion$#1: #2 ${\gdef\fileversion{#2}}
\def\Filedate$#1: #2-#3-#4 #5 ${\gdef\filedate{#2/#3/#4}}
\Fileversion$Revision: 422586 $
\Filedate$Date: 2017-08-27 16:26:38 +0200(dom, 27 ago 2017) $
%%%%%%%%%%%%%%%%%%%%%%%%%%%%%%%%%%%%%%%%%%%%%%%%%%%%%%%%%%%%%%%%%%%%
%
%  CMS Common definitions style file
%
%  N.B. use of \newcommand rather than \newcommand means
%       that a definition is ignored if already specified
%
%                                              L. Taylor 18 Feb 2005
%%%%%%%%%%%%%%%%%%%%%%%%%%%%%%%%%%%%%%%%%%%%%%%%%%%%%%%%%%%%%%%%%%%%
\NeedsTeXFormat{LaTeX2e}
\ProvidesPackage{ptdr-definitions}[\filedate\space CMS Additional Macro Definitions (\fileversion)]
\RequirePackage{xspace}
\RequirePackage{amsmath}

% Some shorthand
% turn off italics
\newcommand {\etal}{\mbox{et al.}\xspace} %et al. - no preceding comma
\newcommand {\ie}{\mbox{i.e.}\xspace}     %i.e.
\newcommand {\eg}{\mbox{e.g.}\xspace}     %e.g.
\newcommand {\etc}{\mbox{etc.}\xspace}     %etc.
\newcommand {\vs}{\mbox{\sl vs.}\xspace}      %vs.
\newcommand {\mdash}{\ensuremath{\mathrm{-}}} % for use within formulas

% some terms whose definition we may change
\newcommand {\Lone}{Level-1\xspace} % Level-1 or L1 ?
\newcommand {\Ltwo}{Level-2\xspace}
\newcommand {\Lthree}{Level-3\xspace}

% Some software programs (alphabetized)
\newcommand{\ACERMC} {\textsc{AcerMC}\xspace}
\newcommand{\ALPGEN} {{\textsc{alpgen}}\xspace}
\newcommand{\BLACKHAT} {{\textsc{BlackHat}}\xspace}
\newcommand{\CALCHEP} {{\textsc{CalcHEP}}\xspace}
\newcommand{\CHARYBDIS} {{\textsc{charybdis}}\xspace}
\newcommand{\CMKIN} {\textsc{cmkin}\xspace}
\newcommand{\CMSIM} {{\textsc{cmsim}}\xspace}
\newcommand{\CMSSW} {{\textsc{cmssw}}\xspace}
\newcommand{\COBRA} {{\textsc{cobra}}\xspace}
\newcommand{\COCOA} {{\textsc{cocoa}}\xspace}
\newcommand{\COMPHEP} {\textsc{CompHEP}\xspace}
\newcommand{\EVTGEN} {{\textsc{evtgen}}\xspace}
\newcommand{\FAMOS} {{\textsc{famos}}\xspace}
\newcommand{\FASTJET} {{\textsc{FastJet}}\xspace}
\newcommand{\FEWZ} {{\textsc{fewz}}\xspace}
\newcommand{\GARCON} {\textsc{garcon}\xspace}
\newcommand{\GARFIELD} {{\textsc{garfield}}\xspace}
\newcommand{\GEANE} {{\textsc{geane}}\xspace}
\newcommand{\GEANTfour} {{\textsc{Geant4}}\xspace}
\newcommand{\GEANTthree} {{\textsc{geant3}}\xspace}
\newcommand{\GEANT} {{\textsc{geant}}\xspace}
\newcommand{\HDECAY} {\textsc{hdecay}\xspace}
\newcommand{\HERWIG} {{\textsc{herwig}}\xspace}
\newcommand{\HERWIGpp} {{\textsc{herwig++}}\xspace}
\newcommand{\POWHEG} {{\textsc{powheg}}\xspace}
\newcommand{\HIGLU} {{\textsc{higlu}}\xspace}
\newcommand{\HIJING} {{\textsc{hijing}}\xspace}
\newcommand{\IGUANA} {\textsc{iguana}\xspace}
\newcommand{\ISAJET} {{\textsc{isajet}}\xspace}
\newcommand{\ISAPYTHIA} {{\textsc{isapythia}}\xspace}
\newcommand{\ISASUGRA} {{\textsc{isasugra}}\xspace}
\newcommand{\ISASUSY} {{\textsc{isasusy}}\xspace}
\newcommand{\ISAWIG} {{\textsc{isawig}}\xspace}
\newcommand{\MADGRAPH} {\textsc{MadGraph}\xspace}
\newcommand{\MCATNLO} {\textsc{mc@nlo}\xspace}
\newcommand{\MCFM} {\textsc{mcfm}\xspace}
\newcommand{\MILLEPEDE} {{\textsc{millepede}}\xspace}
\newcommand{\ORCA} {{\textsc{orca}}\xspace}
\newcommand{\OSCAR} {{\textsc{oscar}}\xspace}
\newcommand{\PHOTOS} {\textsc{photos}\xspace}
\newcommand{\PROSPINO} {\textsc{prospino}\xspace}
\newcommand{\PYTHIA} {{\textsc{pythia}}\xspace}
\newcommand{\SHERPA} {{\textsc{sherpa}}\xspace}
\newcommand{\TAUOLA} {\textsc{tauola}\xspace}
\newcommand{\TOPREX} {\textsc{TopReX}\xspace}
\newcommand{\XDAQ} {{\textsc{xdaq}}\xspace}
\newcommand{\MGvATNLO}{\MADGRAPH{}5\_a\MCATNLO}


%  Experiments
\newcommand {\DZERO}{D0\xspace}     %etc.


% Measurements and units...

\newcommand{\de}{\ensuremath{^\circ}}
\newcommand{\ten}[1]{\ensuremath{\times \text{10}^\text{#1}}}
\newcommand{\unit}[1]{\ensuremath{\text{\,#1}}\xspace}
\newcommand{\mum}{\ensuremath{\,\mu\text{m}}\xspace}
\newcommand{\micron}{\ensuremath{\,\mu\text{m}}\xspace}
\newcommand{\cm}{\ensuremath{\,\text{cm}}\xspace}
\newcommand{\mm}{\ensuremath{\,\text{mm}}\xspace}
\newcommand{\mus}{\ensuremath{\,\mu\text{s}}\xspace}
\newcommand{\keV}{\ensuremath{\,\text{ke\hspace{-.08em}V}}\xspace}
\newcommand{\MeV}{\ensuremath{\,\text{Me\hspace{-.08em}V}}\xspace}
\newcommand{\MeVns}{\ensuremath{\text{Me\hspace{-.08em}V}}\xspace} % no leading thinspace
\newcommand{\GeV}{\ensuremath{\,\text{Ge\hspace{-.08em}V}}\xspace}
\newcommand{\GeVns}{\ensuremath{\text{Ge\hspace{-.08em}V}}\xspace} % no leading thinspace
\newcommand{\gev}{\GeV}
\newcommand{\TeV}{\ensuremath{\,\text{Te\hspace{-.08em}V}}\xspace}
\newcommand{\TeVns}{\ensuremath{\text{Te\hspace{-.08em}V}}\xspace} % no leading thinspace
\newcommand{\PeV}{\ensuremath{\,\text{Pe\hspace{-.08em}V}}\xspace}
\newcommand{\keVc}{\ensuremath{{\,\text{ke\hspace{-.08em}V\hspace{-0.16em}/\hspace{-0.08em}}c}}\xspace}
\newcommand{\MeVc}{\ensuremath{{\,\text{Me\hspace{-.08em}V\hspace{-0.16em}/\hspace{-0.08em}}c}}\xspace}
\newcommand{\GeVc}{\ensuremath{{\,\text{Ge\hspace{-.08em}V\hspace{-0.16em}/\hspace{-0.08em}}c}}\xspace}
\newcommand{\GeVcns}{\ensuremath{{\text{Ge\hspace{-.08em}V\hspace{-0.16em}/\hspace{-0.08em}}c}}\xspace} % no leading thinspace
\newcommand{\TeVc}{\ensuremath{{\,\text{Te\hspace{-.08em}V\hspace{-0.16em}/\hspace{-0.08em}}c}}\xspace}
\newcommand{\keVcc}{\ensuremath{{\,\text{ke\hspace{-.08em}V\hspace{-0.16em}/\hspace{-0.08em}}c^\text{2}}}\xspace}
\newcommand{\MeVcc}{\ensuremath{{\,\text{Me\hspace{-.08em}V\hspace{-0.16em}/\hspace{-0.08em}}c^\text{2}}}\xspace}
\newcommand{\GeVcc}{\ensuremath{{\,\text{Ge\hspace{-.08em}V\hspace{-0.16em}/\hspace{-0.08em}}c^\text{2}}}\xspace}
\newcommand{\GeVccns}{\ensuremath{{\text{Ge\hspace{-.08em}V\hspace{-0.16em}/\hspace{-0.08em}}c^\text{2}}}\xspace} % no leading thinspace
\newcommand{\TeVcc}{\ensuremath{{\,\text{Te\hspace{-.08em}V\hspace{-0.16em}/\hspace{-0.08em}}c^\text{2}}}\xspace}

\newcommand{\pbinv} {\mbox{\ensuremath{\,\text{pb}^\text{$-$1}}}\xspace}
\newcommand{\fbinv} {\mbox{\ensuremath{\,\text{fb}^\text{$-$1}}}\xspace}
\newcommand{\nbinv} {\mbox{\ensuremath{\,\text{nb}^\text{$-$1}}}\xspace}
\newcommand{\mubinv} {\ensuremath{\,\mu\mathrm{b}^{-1}}\xspace}
\newcommand{\mbinv} {\ensuremath{\,\mathrm{mb}^{-1}}\xspace}
\newcommand{\percms}{\ensuremath{\,\text{cm}^\text{$-$2}\,\text{s}^\text{$-$1}}\xspace}
\newcommand{\lumi}{\ensuremath{\mathcal{L}}\xspace}
\newcommand{\Lumi}{\ensuremath{\mathcal{L}}\xspace}%both upper and lower
%
% Need a convention here:
\newcommand{\LvLow}  {\ensuremath{\mathcal{L}=\text{10}^\text{32}\,\text{cm}^\text{$-$2}\,\text{s}^\text{$-$1}}\xspace}
\newcommand{\LLow}   {\ensuremath{\mathcal{L}=\text{10}^\text{33}\,\text{cm}^\text{$-$2}\,\text{s}^\text{$-$1}}\xspace}
\newcommand{\lowlumi}{\ensuremath{\mathcal{L}=\text{2}\times \text{10}^\text{33}\,\text{cm}^\text{$-$2}\,\text{s}^\text{$-$1}}\xspace}
\newcommand{\LMed}   {\ensuremath{\mathcal{L}=\text{2}\times \text{10}^\text{33}\,\text{cm}^\text{$-$2}\,\text{s}^\text{$-$1}}\xspace}
\newcommand{\LHigh}  {\ensuremath{\mathcal{L}=\text{10}^\text{34}\,\text{cm}^\text{$-$2}\,\text{s}^\text{$-$1}}\xspace}
\newcommand{\hilumi} {\ensuremath{\mathcal{L}=\text{10}^\text{34}\,\text{cm}^\text{$-$2}\,\text{s}^\text{$-$1}}\xspace}

% Physics symbols ...

\newcommand{\PT}{\ensuremath{p_{\mathrm{T}}}\xspace}
\newcommand{\pt}{\ensuremath{p_{\mathrm{T}}}\xspace}
\newcommand{\ET}{\ensuremath{E_{\mathrm{T}}}\xspace}
\newcommand{\HT}{\ensuremath{H_{\mathrm{T}}}\xspace}
\newcommand{\et}{\ensuremath{E_{\mathrm{T}}}\xspace}
\newcommand{\Em}{\ensuremath{E\hspace{-0.6em}/}\xspace}
\newcommand{\Pm}{\ensuremath{p\hspace{-0.5em}/}\xspace}
\newcommand{\PTm}{\ensuremath{{p}_\mathrm{T}\hspace{-1.02em}/\kern 0.5em}\xspace}
\newcommand{\PTslash}{\PTm}
\newcommand{\ETm}{\ensuremath{E_{\mathrm{T}}^{\text{miss}}}\xspace}
\newcommand{\MET}{\ETm}
\newcommand{\ETmiss}{\ETm}
\newcommand{\ptmiss}{\ensuremath{\pt^\text{miss}}\xspace}
\newcommand{\ETslash}{\ensuremath{E_{\mathrm{T}}\hspace{-1.1em}/\kern0.45em}\xspace}
\newcommand{\VEtmiss}{\ensuremath{{\vec E}_{\mathrm{T}}^{\text{miss}}}\xspace}
\newcommand{\ptvec}{\ensuremath{{\vec p}_{\mathrm{T}}}\xspace}
\newcommand{\ptvecmiss}{\ensuremath{{\vec p}_{\mathrm{T}}^{\kern1pt\text{miss}}}\xspace}
\newcommand{\tauh}{\ensuremath{\PGt_\mathrm{h}}\xspace}

% roman face derivative
\newcommand{\dd}[2]{\ensuremath{\frac{\cmsSymbolFace{d} #1}{\cmsSymbolFace{d} #2}}}
\newcommand{\ddinline}[2]{\ensuremath{\cmsSymbolFace{d} #1/\cmsSymbolFace{d} #2}}
\newcommand{\rd}{\ensuremath{\cmsSymbolFace{d}}}
\newcommand{\re}{\ensuremath{\cmsSymbolFace{e}}}
% absolute value
\newcommand{\abs}[1]{\ensuremath{\lvert #1 \rvert}}



\ifthenelse{\boolean{cms@italic}}{\newcommand{\cmsSymbolFace}{\relax}}{\newcommand{\cmsSymbolFace}{\mathrm}}

% Particle names which track the italic/non-italic face convention
\newcommand{\zp}{\ensuremath{\cmsSymbolFace{Z}^\prime}\xspace} % plain Z'
\newcommand{\JPsi}{\ensuremath{\cmsSymbolFace{J}\hspace{-.08em}/\hspace{-.14em}\psi}\xspace} % J/Psi (no mass)
\newcommand{\Z}{\ensuremath{\cmsSymbolFace{Z}}\xspace} % plain Z (no superscript 0)
\newcommand{\ttbar}{\ensuremath{\cmsSymbolFace{t}\overline{\cmsSymbolFace{t}}}\xspace} % t-tbar

% Extensions for missing names in PENNAMES % note no xspace, to match syntax in PENNAMES
\newcommand{\cPgn}{\ensuremath{\nu}} % generic neutrino
\providecommand{\Pgn}{\ensuremath{\nu}} % generic neutrino
\newcommand{\cPagn}{\ensuremath{\overline{\nu}}} % generic neutrino
\providecommand{\Pagn}{\ensuremath{\overline{\nu}}} % generic neutrino
\newcommand{\cPgg}{\ensuremath{\gamma}} % gamma
\newcommand{\cPJgy}{\ensuremath{\cmsSymbolFace{J}\hspace{-.08em}/\hspace{-.14em}\psi}} % J/Psi (no mass)
\newcommand{\cPZ}{\ensuremath{\cmsSymbolFace{Z}}} % plain Z (no superscript 0)
\newcommand{\cPZpr}{\ensuremath{\cmsSymbolFace{Z}'}} % plain Z'
\newcommand{\cPqt}{\ensuremath{\cmsSymbolFace{t}}} % t for t quark
\newcommand{\cPqb}{\ensuremath{\cmsSymbolFace{b}}} % b for b quark
\newcommand{\cPqc}{\ensuremath{\cmsSymbolFace{c}}} % c for c quark
\newcommand{\cPqs}{\ensuremath{\cmsSymbolFace{s}}} % s for s quark
\newcommand{\cPqu}{\ensuremath{\cmsSymbolFace{u}}} % u for u quark
\newcommand{\cPqd}{\ensuremath{\cmsSymbolFace{d}}} % d for d quark
\newcommand{\cPq}{\ensuremath{\cmsSymbolFace{q}}} % generic quark
\newcommand{\cPg}{\ensuremath{\cmsSymbolFace{g}}} % generic gluon
\newcommand{\cPG}{\ensuremath{\cmsSymbolFace{G}}} % Graviton
\newcommand{\cPaqt}{\ensuremath{\overline{\cmsSymbolFace{t}}}} % t for t anti-quark
\newcommand{\cPaqb}{\ensuremath{\overline{\cmsSymbolFace{b}}}} % b for b anti-quark
\newcommand{\cPaqc}{\ensuremath{\overline{\cmsSymbolFace{c}}}} % c for c anti-quark
\newcommand{\cPaqs}{\ensuremath{\overline{\cmsSymbolFace{s}}}} % s for s anti-quark
\newcommand{\cPaqu}{\ensuremath{\overline{\cmsSymbolFace{u}}}} % u for u anti-quark
\newcommand{\cPaqd}{\ensuremath{\overline{\cmsSymbolFace{d}}}} % d for d anti-quark
\newcommand{\cPaq}{\ensuremath{\overline{\cmsSymbolFace{q}}}} % generic anti-quark
\newcommand{\cPKstz}{\ensuremath{\cmsSymbolFace{K}^{\ast0}}\xspace} %note has xspace
% future symbols from heppennames2
\providecommand{\PGp}{\ensuremath{\pi}\xspace} % pi
\providecommand{\PGpp}{\ensuremath{\pi^+}\xspace} % pi
\providecommand{\PGpm}{\ensuremath{\pi^-}\xspace} % pi
\providecommand{\PGpz}{\ensuremath{\pi^0}\xspace} % pi
\providecommand{\PGr}{\ensuremath{\rho}\xspace} % pi
\providecommand{\PDast}{\ensuremath{\cmsSymbolFace{D}^\ast}\xspace} % D star
\providecommand{\PH}{\ensuremath{\cmsSymbolFace{H}}\xspace} % plain Higgs
\providecommand{\Ph}{\ensuremath{\cmsSymbolFace{h}}\xspace} % SUSY Higgs
\providecommand{\Pa}{\ensuremath{\cmsSymbolFace{a}}\xspace}
\providecommand{\PSA}{\ensuremath{\cmsSymbolFace{A}}\xspace} %pseudoscalar A Higgs
\providecommand{\PJGy}{\ensuremath{\cmsSymbolFace{J}\hspace{-.08em}/\hspace{-.14em}\psi}\xspace} % J/Psi (no mass)
\providecommand{\PBzs}{\ensuremath{\cmsSymbolFace{B}^0_\cmsSymbolFace{s}}\xspace} % B^0_s
\providecommand{\Pg}{\ensuremath{\cmsSymbolFace{g}}\xspace} % generic gluon
\providecommand{\PSg}{\ensuremath{\widetilde{\cmsSymbolFace{g}}}\xspace} % gluino
\providecommand{\PSQ}{\ensuremath{\widetilde{\cmsSymbolFace{q}}}\xspace} % squark
\providecommand{\PSGm}{\ensuremath{\widetilde{\mu}}\xspace} % smuon
\providecommand{\PSe}{\ensuremath{\widetilde{\cmsSymbolFace{e}}}\xspace} % selectron
\providecommand{\PASQ}{\ensuremath{\overline{\widetilde{\cmsSymbolFace{q}}}}\xspace} % anti quark
\providecommand{\PXXA}{\ensuremath{\cmsSymbolFace{A}}\xspace} % axion
\providecommand{\PXXG}{\ensuremath{\cmsSymbolFace{G}}\xspace} % graviton
\providecommand{\PXXSG}{\ensuremath{\widetilde{\PXXG}}\xspace} % gravitino
\providecommand{\PSGcp}{\ensuremath{\widetilde{\chi}^+}\xspace} % lightest positive chargino
\providecommand{\PSGcm}{\ensuremath{\widetilde{\chi}^-}\xspace} % lightest negative chargino
\providecommand{\PSGc}{\ensuremath{\widetilde{\chi}}\xspace} % neutralino
\providecommand{\PSGcz}{\ensuremath{\widetilde{\chi}^0}\xspace} % neutralino with superscript 0
\providecommand{\PSGczDo}{\ensuremath{\widetilde{\chi}^{0}_{1}}\xspace} % neutralino
\providecommand{\PSGcmDo}{\ensuremath{\widetilde{\chi}^{-}_{1}}\xspace} % neutralino
\providecommand{\PSGczDt}{\ensuremath{\widetilde{\chi}^{0}_{2}}\xspace} % neutralino
\providecommand{\PSGcpm}{\ensuremath{\widetilde{\chi}^\pm}\xspace} % neutralino
\providecommand{\PSGcpmDo}{\ensuremath{\widetilde{\chi}^\pm_{1}}\xspace} % neutralino
\providecommand{\PSGcpDo}{\ensuremath{\widetilde{\chi}^{+}_{1}}\xspace} % neutralino
\providecommand{\Pl}{\ensuremath{\cmsSymbolFace{l}}\xspace} % non-ell lepton
\providecommand{\PAl}{\ensuremath{\overline{\cmsSymbolFace{l}}}\xspace} % non-ell anti-lepton
\providecommand{\PGnl}{\ensuremath{\nu_\cmsSymbolFace{l}}\xspace} % lepton neutrino
\providecommand{\PAGnl}{\ensuremath{\overline{\nu}_\cmsSymbolFace{l}}\xspace} % anti-lepton neutrino
\providecommand{\PQtpr}{\ensuremath{\cmsSymbolFace{t}^{\prime}}\xspace} % t'
\providecommand{\PAQtpr}{\ensuremath{\bar{\cmsSymbolFace{t}}^\prime}\xspace} % t'-bar; needs to be converted to overline-requires rework a la heppennames
\providecommand{\PQbpr}{\ensuremath{\cmsSymbolFace{b}^{\prime}}\xspace} % b'
\providecommand{\PAQbpr}{\ensuremath{\bar{\cmsSymbolFace{b}}^\prime}\xspace} % b'-bar; needs same as anti-t'
\providecommand{\PGg}{\ensuremath{\gamma}\xspace} % gamma
\providecommand{\PKzS}{\ensuremath{\cmsSymbolFace{K}^0_\cmsSymbolFace{S}}\xspace} % K short
\providecommand{\PBs}{\ensuremath{\cmsSymbolFace{B}_\cmsSymbolFace{s}}\xspace} % B sub s
\providecommand{\PSQu}{\ensuremath{\widetilde{\cmsSymbolFace{u}}}\xspace}
\providecommand{\PSQd}{\ensuremath{\widetilde{\cmsSymbolFace{d}}}\xspace}
\providecommand{\PSQc}{\ensuremath{\widetilde{\cmsSymbolFace{c}}}\xspace}
\providecommand{\PSQs}{\ensuremath{\widetilde{\cmsSymbolFace{s}}}\xspace}
\providecommand{\PSQt}{\ensuremath{\widetilde{\cmsSymbolFace{t}}}\xspace} % stop
\providecommand{\PSQb}{\ensuremath{\widetilde{\cmsSymbolFace{b}}}\xspace}
\providecommand{\PASQt}{\ensuremath{\overline{\widetilde{\cmsSymbolFace{t}}}}\xspace} % anti stop
\providecommand{\PASQb}{\ensuremath{\overline{\widetilde{\cmsSymbolFace{b}}}}\xspace} % anti sbottom
\providecommand{\PSGt}{\ensuremath{\widetilde{\tau}}\xspace} % stau
\providecommand{\PZ}{\ensuremath{\cmsSymbolFace{Z}}\xspace} % may have some confusion with the \xspace...
\providecommand{\PZpr}{\ensuremath{\cmsSymbolFace{Z}'}\xspace} % plain Z' using prime
\renewcommand{\PWpr}{\ensuremath{\cmsSymbolFace{W}'}\xspace} % use prime like pennames2
\providecommand{\PGn}{\ensuremath{\nu}\xspace} % generic neutrino
\providecommand{\PAGn}{\ensuremath{\overline{\nu}}\xspace} % generic neutrino
\providecommand{\PSQtDo}{\ensuremath{\widetilde{\cmsSymbolFace{t}}_1}\xspace}
\providecommand{\PSQtDt}{\ensuremath{\widetilde{\cmsSymbolFace{t}}_2}\xspace}
\providecommand{\PQt}{\ensuremath{\cmsSymbolFace{t}}\xspace} % t
\providecommand{\PAQt}{\ensuremath{\overline{\cmsSymbolFace{t}}}\xspace} %
\providecommand{\PQb}{\ensuremath{\cmsSymbolFace{b}}\xspace} % b
\providecommand{\PAQb}{\ensuremath{\overline{\cmsSymbolFace{b}}}\xspace} %
\providecommand{\PGm}{\ensuremath{\mu}\xspace} % muon
\providecommand{\PGmm}{\ensuremath{\mu^-}\xspace} % muon
\providecommand{\PGmp}{\ensuremath{\mu^+}\xspace} % muon
\providecommand{\PGmpm}{\ensuremath{\mu^\pm}\xspace} % muon
\providecommand{\PGt}{\ensuremath{\tau}\xspace} % tau
\providecommand{\PAGt}{\ensuremath{\overline{\tau}}\xspace} % anti-tau
\providecommand{\PGpz}{\ensuremath{\pi^0}\xspace}
\providecommand{\PQq}{\ensuremath{\cmsSymbolFace{q}}\xspace} % quark (generic)
\providecommand{\PQd}{\ensuremath{\cmsSymbolFace{d}}\xspace} % down quark
\providecommand{\PQu}{\ensuremath{\cmsSymbolFace{u}}\xspace} % up quark
\providecommand{\PQs}{\ensuremath{\cmsSymbolFace{s}}\xspace} % top quark
\providecommand{\PQc}{\ensuremath{\cmsSymbolFace{c}}\xspace} % top quark
\providecommand{\PAQq}{\ensuremath{\overline{\cmsSymbolFace{q}}}\xspace} % quark (generic)
\providecommand{\PAQd}{\ensuremath{\overline{\cmsSymbolFace{d}}}\xspace} % down quark
\providecommand{\PAQu}{\ensuremath{\overline{\cmsSymbolFace{u}}}\xspace} % up quark
\providecommand{\PAQs}{\ensuremath{\overline{\cmsSymbolFace{s}}}\xspace} % top quark
\providecommand{\PAQc}{\ensuremath{\overline{\cmsSymbolFace{c}}}\xspace} % top quark
\providecommand{\PGne}{\ensuremath{\nu_\cmsSymbolFace{e}}\xspace} % electron neutrino
\providecommand{\PAGne}{\ensuremath{\overline{\nu}_\cmsSymbolFace{e}}\xspace} % anti-electron neutrino
\providecommand{\PGnGm}{\ensuremath{\nu_\PGm}\xspace} % muon neutrino
\providecommand{\PAGnGm}{\ensuremath{\overline{\nu}_\PGm}\xspace} % anti-muon neutrino
\providecommand{\PGnGt}{\ensuremath{\nu_\PGt}\xspace} % tau neutrino
\providecommand{\PAGnGt}{\ensuremath{\overline{\nu}_\PGt}\xspace} % anti-tau neutrino
\providecommand{\PAp}{\ensuremath{\overline{\cmsSymbolFace{p}}}\xspace} % anti-proton
\providecommand{\PAn}{\ensuremath{\overline{\cmsSymbolFace{n}}}\xspace} % anti-neutron
\providecommand{\PDz}{\ensuremath{\cmsSymbolFace{D}^0}\xspace} % D0 meson
\providecommand{\PADz}{\ensuremath{\overline{\cmsSymbolFace{D}}^0}\xspace} % anti-D0 meson
\providecommand{\PAD}{\ensuremath{\overline{\cmsSymbolFace{D}}}\xspace} % anti-D meson
\providecommand{\PAK}{\ensuremath{\overline{\cmsSymbolFace{K}}}\xspace} % anti-K meson
\providecommand{\PAKz}{\ensuremath{\overline{\cmsSymbolFace{K}}^0}\xspace} % anti-K0 meson
\providecommand{\PABz}{\ensuremath{\overline{\cmsSymbolFace{B}}^0}\xspace} % anti-B0 meson

% our extensions for pennames2
\providecommand{\Pepm}{\ensuremath{\cmsSymbolFace{e}^\pm}\xspace}
% for APS style tables
\ifthenelse{\boolean{cms@external}}{%
\newenvironment{scotch}[1]{\protect\centering\ruledtabular\tabular{#1}}{\endtabular\endruledtabular}
}{
\newenvironment{scotch}[1]{\protect\centering\tabular{#1}\hline\hline}{\hline\endtabular}
}

% SM (still to be classified)

\newcommand{\AFB}{\ensuremath{A_\text{FB}}\xspace}
\newcommand{\wangle}{\ensuremath{\sin^{2}\theta_{\text{eff}}^\text{lept}(M^2_{\Z})}\xspace}
\newcommand{\stat}{\ensuremath{\,\text{(stat)}}\xspace}
\newcommand{\syst}{\ensuremath{\,\text{(syst)}}\xspace}
\newcommand{\thy}{\ensuremath{\,\text{(theo)}}\xspace}
\newcommand{\lum}{\ensuremath{\,\text{(lumi)}}\xspace}
\newcommand{\kt}{\ensuremath{k_{\mathrm{T}}}\xspace}

\newcommand{\BC}{\ensuremath{\cmsSymbolFace{B_{c}}}\xspace}
\newcommand{\bbarc}{\ensuremath{\PQb\PAQc}\xspace}
\newcommand{\bbbar}{\ensuremath{\PQb\PAQb}\xspace}
\newcommand{\ccbar}{\ensuremath{\PQc\PAQc}\xspace}
\newcommand{\qqbar}{\ensuremath{\PQq\PAQq}\xspace}
\newcommand{\bspsiphi}{\ensuremath{\cmsSymbolFace{B_s} \to \JPsi\, \phi}\xspace}
\newcommand{\EE}{\ensuremath{\Pep\Pem}\xspace}
\newcommand{\MM}{\ensuremath{\Pgmp\Pgmm}\xspace}
\newcommand{\TT}{\ensuremath{\Pgt^{+}\Pgt^{-}}\xspace}

%%%  E-gamma definitions
\newcommand{\HGG}{\ensuremath{\cmsSymbolFace{H}\to\gamma\gamma}}
\newcommand{\GAMJET}{\ensuremath{\gamma + \text{jet}}}
\newcommand{\PPTOJETS}{\ensuremath{\Pp\Pp\to\text{jets}}}
\newcommand{\PPTOGG}{\ensuremath{\Pp\Pp\to\gamma\gamma}}
\newcommand{\PPTOGAMJET}{\ensuremath{\Pp\Pp\to\gamma + \text{jet}}}
\newcommand{\MH}{\ensuremath{M_{\PH}}}
\newcommand{\RNINE}{\ensuremath{R_\mathrm{9}}}
\newcommand{\DR}{\ensuremath{\Delta R}}





%%%%%%
% From Albert
%

\newcommand{\ga}{\ensuremath{\gtrsim}}
\newcommand{\la}{\ensuremath{\lesssim}}
%
\newcommand{\swsq}{\ensuremath{\sin^2\theta_\cmsSymbolFace{W}}\xspace}
\newcommand{\cwsq}{\ensuremath{\cos^2\theta_\cmsSymbolFace{W}}\xspace}
\newcommand{\tanb}{\ensuremath{\tan\beta}\xspace}
\newcommand{\tanbsq}{\ensuremath{\tan^{2}\beta}\xspace}
\newcommand{\sidb}{\ensuremath{\sin 2\beta}\xspace}
\newcommand{\alpS}{\ensuremath{\alpha_S}\xspace}
\newcommand{\alpt}{\ensuremath{\tilde{\alpha}}\xspace}

\newcommand{\QL}{\ensuremath{\cmsSymbolFace{Q}_\cmsSymbolFace{L}}\xspace}
\newcommand{\sQ}{\ensuremath{\widetilde{\cmsSymbolFace{Q}}}\xspace}
\newcommand{\sQL}{\ensuremath{\widetilde{\cmsSymbolFace{Q}}_\cmsSymbolFace{L}}\xspace}
\newcommand{\ULC}{\ensuremath{\cmsSymbolFace{U}_\cmsSymbolFace{L}^\cmsSymbolFace{C}}\xspace}
\newcommand{\sUC}{\ensuremath{\widetilde{\cmsSymbolFace{U}}^\cmsSymbolFace{C}}\xspace}
\newcommand{\sULC}{\ensuremath{\widetilde{\cmsSymbolFace{U}}_\cmsSymbolFace{L}^\cmsSymbolFace{C}}\xspace}
\newcommand{\DLC}{\ensuremath{\cmsSymbolFace{D}_\cmsSymbolFace{L}^\cmsSymbolFace{C}}\xspace}
\newcommand{\sDC}{\ensuremath{\widetilde{\cmsSymbolFace{D}}^\cmsSymbolFace{C}}\xspace}
\newcommand{\sDLC}{\ensuremath{\widetilde{\cmsSymbolFace{D}}_\cmsSymbolFace{L}^\cmsSymbolFace{C}}\xspace}
\newcommand{\LL}{\ensuremath{\cmsSymbolFace{L}_\cmsSymbolFace{L}}\xspace}
\newcommand{\sL}{\ensuremath{\widetilde{\cmsSymbolFace{L}}}\xspace}
\newcommand{\sLL}{\ensuremath{\widetilde{\cmsSymbolFace{L}}_\cmsSymbolFace{L}}\xspace}
\newcommand{\ELC}{\ensuremath{\cmsSymbolFace{E}_\cmsSymbolFace{L}^\cmsSymbolFace{C}}\xspace}
\newcommand{\sEC}{\ensuremath{\widetilde{\cmsSymbolFace{E}}^\cmsSymbolFace{C}}\xspace}
\newcommand{\sELC}{\ensuremath{\widetilde{\cmsSymbolFace{E}}_\cmsSymbolFace{L}^\cmsSymbolFace{C}}\xspace}
\newcommand{\sEL}{\ensuremath{\widetilde{\cmsSymbolFace{E}}_\cmsSymbolFace{L}}\xspace}
\newcommand{\sER}{\ensuremath{\widetilde{\cmsSymbolFace{E}}_\cmsSymbolFace{R}}\xspace}
\newcommand{\sFer}{\ensuremath{\widetilde{\cmsSymbolFace{f}}}\xspace}
\newcommand{\sQua}{\ensuremath{\widetilde{\cmsSymbolFace{q}}}\xspace}
\newcommand{\sUp}{\ensuremath{\widetilde{\cmsSymbolFace{u}}}\xspace}
\newcommand{\suL}{\ensuremath{\widetilde{\cmsSymbolFace{u}}_\cmsSymbolFace{L}}\xspace}
\newcommand{\suR}{\ensuremath{\widetilde{\cmsSymbolFace{u}}_\cmsSymbolFace{R}}\xspace}
\newcommand{\sDw}{\ensuremath{\widetilde{\cmsSymbolFace{d}}}\xspace}
\newcommand{\sdL}{\ensuremath{\widetilde{\cmsSymbolFace{d}}_\cmsSymbolFace{L}}\xspace}
\newcommand{\sdR}{\ensuremath{\widetilde{\cmsSymbolFace{d}}_\cmsSymbolFace{R}}\xspace}
\newcommand{\sTop}{\ensuremath{\widetilde{\cmsSymbolFace{t}}}\xspace}
\newcommand{\stL}{\ensuremath{\widetilde{\cmsSymbolFace{t}}_\cmsSymbolFace{L}}\xspace}
\newcommand{\stR}{\ensuremath{\widetilde{\cmsSymbolFace{t}}_\cmsSymbolFace{R}}\xspace}
\newcommand{\stone}{\ensuremath{\widetilde{\cmsSymbolFace{t}}_1}\xspace}
\newcommand{\sttwo}{\ensuremath{\widetilde{\cmsSymbolFace{t}}_2}\xspace}
\newcommand{\sBot}{\ensuremath{\widetilde{\cmsSymbolFace{b}}}\xspace}
\newcommand{\sbL}{\ensuremath{\widetilde{\cmsSymbolFace{b}}_\cmsSymbolFace{L}}\xspace}
\newcommand{\sbR}{\ensuremath{\widetilde{\cmsSymbolFace{b}}_\cmsSymbolFace{R}}\xspace}
\newcommand{\sbone}{\ensuremath{\widetilde{\cmsSymbolFace{b}}_1}\xspace}
\newcommand{\sbtwo}{\ensuremath{\widetilde{\cmsSymbolFace{b}}_2}\xspace}
\newcommand{\sLep}{\ensuremath{\widetilde{\cmsSymbolFace{l}}}\xspace}
\newcommand{\sLepC}{\ensuremath{\widetilde{\cmsSymbolFace{l}}^\cmsSymbolFace{C}}\xspace}
\newcommand{\sEl}{\ensuremath{\widetilde{\cmsSymbolFace{e}}}\xspace}
\newcommand{\sElC}{\ensuremath{\widetilde{\cmsSymbolFace{e}}^\cmsSymbolFace{C}}\xspace}
\newcommand{\seL}{\ensuremath{\widetilde{\cmsSymbolFace{e}}_\cmsSymbolFace{L}}\xspace}
\newcommand{\seR}{\ensuremath{\widetilde{\cmsSymbolFace{e}}_\cmsSymbolFace{R}}\xspace}
\newcommand{\snL}{\ensuremath{\widetilde{\nu}_L}\xspace}
\newcommand{\sMu}{\ensuremath{\widetilde{\mu}}\xspace}
\newcommand{\sNu}{\ensuremath{\widetilde{\nu}}\xspace}
\newcommand{\sTau}{\ensuremath{\widetilde{\tau}}\xspace}
\newcommand{\Glu}{\ensuremath{\cmsSymbolFace{g}}\xspace}
\newcommand{\sGlu}{\ensuremath{\widetilde{\cmsSymbolFace{g}}}\xspace}
\newcommand{\Wpm}{\ensuremath{\cmsSymbolFace{W}^{\pm}}\xspace}
\newcommand{\sWpm}{\ensuremath{\widetilde{\cmsSymbolFace{W}}^{\pm}}\xspace}
\newcommand{\Wz}{\ensuremath{\cmsSymbolFace{W}^{0}}\xspace}
\newcommand{\sWz}{\ensuremath{\widetilde{\cmsSymbolFace{W}}^{0}}\xspace}
\newcommand{\sWino}{\ensuremath{\widetilde{\cmsSymbolFace{W}}}\xspace}
\newcommand{\Bz}{\ensuremath{\cmsSymbolFace{B}^{0}}\xspace}
\newcommand{\sBz}{\ensuremath{\widetilde{\cmsSymbolFace{B}}^{0}}\xspace}
\newcommand{\sBino}{\ensuremath{\widetilde{\cmsSymbolFace{B}}}\xspace}
\newcommand{\Zz}{\ensuremath{\cmsSymbolFace{Z}^{0}}\xspace}
\newcommand{\sZino}{\ensuremath{\widetilde{\cmsSymbolFace{Z}}^{0}}\xspace}
\newcommand{\sGam}{\ensuremath{\widetilde{\gamma}}\xspace}
\newcommand{\chiz}{\ensuremath{\widetilde{\chi}^{0}}\xspace}
\newcommand{\chip}{\ensuremath{\widetilde{\chi}^{+}}\xspace}
\newcommand{\chim}{\ensuremath{\widetilde{\chi}^{-}}\xspace}
\newcommand{\chipm}{\ensuremath{\widetilde{\chi}^{\pm}}\xspace}
\newcommand{\Hone}{\ensuremath{\cmsSymbolFace{H}_\cmsSymbolFace{d}}\xspace}
\newcommand{\sHone}{\ensuremath{\widetilde{\cmsSymbolFace{H}}_\cmsSymbolFace{d}}\xspace}
\newcommand{\Htwo}{\ensuremath{\cmsSymbolFace{H}_\cmsSymbolFace{u}}\xspace}
\newcommand{\sHtwo}{\ensuremath{\widetilde{\cmsSymbolFace{H}}_\cmsSymbolFace{u}}\xspace}
\newcommand{\sHig}{\ensuremath{\widetilde{\cmsSymbolFace{H}}}\xspace}
\newcommand{\sHa}{\ensuremath{\widetilde{\cmsSymbolFace{H}}_\cmsSymbolFace{a}}\xspace}
\newcommand{\sHb}{\ensuremath{\widetilde{\cmsSymbolFace{H}}_\cmsSymbolFace{b}}\xspace}
\newcommand{\sHpm}{\ensuremath{\widetilde{\cmsSymbolFace{H}}^{\pm}}\xspace}
\newcommand{\hz}{\ensuremath{\cmsSymbolFace{h}^{0}}\xspace}
\newcommand{\Hz}{\ensuremath{\cmsSymbolFace{H}^{0}}\xspace}
\newcommand{\Az}{\ensuremath{\cmsSymbolFace{A}^{0}}\xspace}
\newcommand{\Hpm}{\ensuremath{\cmsSymbolFace{H}^{\pm}}\xspace}
\newcommand{\sGra}{\ensuremath{\widetilde{\cmsSymbolFace{G}}}\xspace}
%
\newcommand{\mtil}{\ensuremath{\widetilde{m}}\xspace}
%
\newcommand{\rpv}{\ensuremath{\rlap{\kern.2em/}R}\xspace}
\newcommand{\LLE}{\ensuremath{LL\bar{E}}\xspace}
\newcommand{\LQD}{\ensuremath{LQ\bar{D}}\xspace}
\newcommand{\UDD}{\ensuremath{\overline{UDD}}\xspace}
\newcommand{\Lam}{\ensuremath{\lambda}\xspace}
\newcommand{\Lamp}{\ensuremath{\lambda'}\xspace}
\newcommand{\Lampp}{\ensuremath{\lambda''}\xspace}
%
\newcommand{\spinbd}[2]{\ensuremath{\bar{#1}_{\dot{#2}}}\xspace}

\newcommand{\MD}{\ensuremath{{M_\mathrm{D}}}\xspace}% ED mass
\newcommand{\Mpl}{\ensuremath{{M_\mathrm{Pl}}}\xspace}% Planck mass
\newcommand{\Rinv} {\ensuremath{{R}^{-1}}\xspace}
\endinput
 at the beginning.
%
{
\small%
\providecommand{\symexamp}[2]
{\makebox[\columnwidth]{\makebox[0.160\textwidth][l]{#1:}
                          \makebox[0.025\textwidth][l]{~}
                          \parbox[t]{0.26\textwidth}{#2\vspace*{-2.5mm}\hfill}}\\}
\setlength{\columnseprule}{0.2mm}
\ifthenelse{\boolean{cms@external}}{}{\setlength{\multicolsep}{7mm}
%
\vspace*{-5mm}
\begin{multicols}{2}
%
\raggedcolumns%
}
\noindent%
\symexamp{etal}{\etal}
\symexamp{ie}{\ie}
\symexamp{eg}{\eg}
\symexamp{etc}{\etc}
\symexamp{vs}{\vs}
\symexamp{mdash}{\mdash}
\symexamp{Lone}{\Lone}
\symexamp{Ltwo}{\Ltwo}
\symexamp{Lthree}{\Lthree}
\symexamp{ACERMC}{\ACERMC}
\symexamp{ALPGEN}{\ALPGEN}
\symexamp{CALCHEP}{\CALCHEP}
\symexamp{CHARYBDIS}{\CHARYBDIS}
\symexamp{CMKIN}{\CMKIN}
\symexamp{CMSIM}{\CMSIM}
\symexamp{CMSSW}{\CMSSW}
\symexamp{COBRA}{\COBRA}
\symexamp{COCOA}{\COCOA}
\symexamp{COMPHEP}{\COMPHEP}
\symexamp{EVTGEN}{\EVTGEN}
\symexamp{FAMOS}{\FAMOS}
\symexamp{FEWZ}{\FEWZ}
\symexamp{GARCON}{\GARCON}
\symexamp{GARFIELD}{\GARFIELD}
\symexamp{GEANE}{\GEANE}
\symexamp{GEANTfour}{\GEANTfour}
\symexamp{GEANTthree}{\GEANTthree}
\symexamp{GEANT}{\GEANT}
\symexamp{HDECAY}{\HDECAY}
\symexamp{HERWIG}{\HERWIG}
\symexamp{HERWIGpp}{\HERWIGpp}
\symexamp{POWHEG}{\POWHEG}
\symexamp{HIGLU}{\HIGLU}
\symexamp{HIJING}{\HIJING}
\symexamp{IGUANA}{\IGUANA}
\symexamp{ISAJET}{\ISAJET}
\symexamp{ISAPYTHIA}{\ISAPYTHIA}
\symexamp{ISASUGRA}{\ISASUGRA}
\symexamp{ISASUSY}{\ISASUSY}
\symexamp{ISAWIG}{\ISAWIG}
\symexamp{MADGRAPH}{\MADGRAPH}
\symexamp{MCATNLO}{\MCATNLO}
\symexamp{MCFM}{\MCFM}
\symexamp{MILLEPEDE}{\MILLEPEDE}
\symexamp{ORCA}{\ORCA}
\symexamp{OSCAR}{\OSCAR}
\symexamp{PHOTOS}{\PHOTOS}
\symexamp{PROSPINO}{\PROSPINO}
\symexamp{PYTHIA}{\PYTHIA}
\symexamp{SHERPA}{\SHERPA}
\symexamp{TAUOLA}{\TAUOLA}
\symexamp{TOPREX}{\TOPREX}
\symexamp{XDAQ}{\XDAQ}
\symexamp{DZERO}{\DZERO}
\symexamp{de}{\de}
\symexamp{ten\{x\}}{\ten{x}}
\symexamp{unit\{x\}}{\unit{x}}
\symexamp{mum}{\mum {\tiny[Most units include leading thinspace]}}
\symexamp{micron}{\micron}
\symexamp{cm}{\cm}
\symexamp{mm}{\mm}
\symexamp{mus}{\mus}
\symexamp{keV}{\keV}
\symexamp{MeV}{\MeV}
\symexamp{MeVns}{\MeVns\hspace{\fill}{\tiny[\textit{no} leading thinspace with \textit{ns} suffix]}}
\symexamp{GeV}{\GeV}
\symexamp{GeVns}{\GeVns}
\symexamp{gev}{\gev}
\symexamp{TeV}{\TeV}
\symexamp{TeVns}{\TeVns}
\symexamp{PeV}{\PeV}
\symexamp{keVc}{\keVc}
\symexamp{MeVc}{\MeVc}
\symexamp{GeVc}{\GeVc}
\symexamp{GeVcns}{\GeVcns}
\symexamp{TeVc}{\TeVc}
\symexamp{keVcc}{\keVcc}
\symexamp{MeVcc}{\MeVcc}
\symexamp{GeVcc}{\GeVcc}
\symexamp{GeVccns}{\GeVccns}
\symexamp{TeVcc}{\TeVcc}
\symexamp{pbinv}{\pbinv}
\symexamp{fbinv}{\fbinv}
\symexamp{nbinv}{\nbinv}
\symexamp{mubinv}{\mubinv}
\symexamp{percms}{\percms}
\symexamp{lumi}{\lumi}
\symexamp{Lumi}{\Lumi}
\symexamp{LvLow}{\LvLow}
\symexamp{LLow}{\LLow}
\symexamp{lowlumi}{\lowlumi}
\symexamp{LMed}{\LMed}
\symexamp{LHigh}{\LHigh}
\symexamp{hilumi}{\hilumi}
\symexamp{PT}{\PT}
\symexamp{pt}{\pt}
\symexamp{ET}{\ET}
\symexamp{HT}{\HT}
\symexamp{et}{\et}
\symexamp{Em}{\Em}
\symexamp{Pm}{\Pm}
\symexamp{PTm}{\PTm}
\symexamp{PTslash}{\PTslash}
\symexamp{ETm}{\ETm}
\symexamp{MET}{\MET}
\symexamp{ETmiss}{\ETmiss}
\symexamp{ETslash}{\ETslash}
\symexamp{VEtmiss}{\VEtmiss}
\symexamp{ptvec}{\ptvec}
\symexamp{ptvecmiss}{\ptvecmiss}
\symexamp{dd\{y\}\{x\}}{\dd{y}{x}}
\symexamp{ddinline\{y\}\{x\}}{\ddinline{y}{x}}
\symexamp{rd}{\rd}
\symexamp{re}{\re}
\symexamp{abs\{x\}}{\abs{x}}
\symexamp{zp}{\zp}
\symexamp{JPsi}{\JPsi}
\symexamp{Z}{\Z}
\symexamp{ttbar}{\ttbar}
\noindent%
\textcolor{gray}{\rule{\columnwidth}{0.2pt}}
\noindent%
Extensions to PENNAMES\\
\symexamp{cPgn}{\cPgn}
\symexamp{Pgn}{\Pgn}
\symexamp{cPagn}{\cPagn}
\symexamp{Pagn}{\Pagn}
\symexamp{cPgg}{\cPgg}
\symexamp{cPJgy}{\cPJgy}
\symexamp{cPZ}{\cPZ}
\symexamp{cPZpr}{\cPZpr}
\symexamp{cPqt}{\cPqt}
\symexamp{cPqb}{\cPqb}
\symexamp{cPqc}{\cPqc}
\symexamp{cPqs}{\cPqs}
\symexamp{cPqu}{\cPqu}
\symexamp{cPqd}{\cPqd}
\symexamp{cPq}{\cPq}
\symexamp{cPg}{\cPg}
\symexamp{cPG}{\cPG}
\symexamp{cPaqt}{\cPaqt}
\symexamp{cPaqb}{\cPaqb}
\symexamp{cPaqc}{\cPaqc}
\symexamp{cPaqs}{\cPaqs}
\symexamp{cPaqu}{\cPaqu}
\symexamp{cPaqd}{\cPaqd}
\symexamp{cPaq}{\cPaq}
\symexamp{cPKstz}{\cPKstz}
\noindent%
\textcolor{gray}{\rule{\columnwidth}{0.2pt}}
\noindent%
Future PENNAMES2\hspace{\fill}{\tiny{include \verb|\xspace|}}\\
\symexamp{PH}{\PH}
\symexamp{Ph}{\Ph}
\symexamp{PJGy}{\PJGy}
\symexamp{PBzs}{\PBzs}
\symexamp{Pg}{\Pg}
\symexamp{PSg}{\PSg}
\symexamp{PSQ}{\PSQ}
\symexamp{PXXG}{\PXXG}
\symexamp{PXXSG}{\PXXSG}
\symexamp{PSGcp}{\PSGcp}
\symexamp{PSGc}{\PSGc}
\symexamp{PSGcz}{\PSGcz}
\symexamp{PSGczDo}{\PSGczDo}
\symexamp{PSGczDt}{\PSGczDt}
\symexamp{PSGcpm}{\PSGcpm}
\symexamp{PSGcpDo}{\PSGcpDo}
\symexamp{Pl}{\Pl}
\symexamp{PAl}{\PAl}
\symexamp{PGnl}{\PGnl}
\symexamp{PAGnl}{\PAGnl}
\symexamp{PQtpr}{\PQtpr}
\symexamp{PAQtpr}{\PAQtpr}
\symexamp{PQbpr}{\PQbpr}
\symexamp{PAQbpr}{\PAQbpr}
\symexamp{PGg}{\PGg}
\symexamp{PKzS}{\PKzS}
\symexamp{PBs}{\PBs}
\symexamp{PSQu}{\PSQu}
\symexamp{PSQd}{\PSQd}
\symexamp{PSQc}{\PSQc}
\symexamp{PSQs}{\PSQs}
\symexamp{PSQt}{\PSQt}
\symexamp{PSQb}{\PSQb}
\symexamp{PASQt}{\PASQt}
\symexamp{PASQb}{\PASQb}
\symexamp{PSGt}{\PSGt}
\symexamp{PZpr}{\PZpr}
\symexamp{PGn}{\PGn}
\symexamp{PAGn}{\PAGn}
\symexamp{PSQtDo}{\PSQtDo}
\symexamp{PSQtDt}{\PSQtDt}
\symexamp{PQt}{\PQt}
\symexamp{PAQt}{\PAQt}
\symexamp{PQb}{\PQb}
\symexamp{PAQb}{\PAQb}
\symexamp{PGm}{\PGm}
\symexamp{PGt}{\PGt}
\symexamp{PQq}{\PQq}
\symexamp{PQd}{\PQd}
\symexamp{PQu}{\PQu}
\symexamp{PQs}{\PQs}
\symexamp{PQc}{\PQc}
\symexamp{PAQq}{\PAQq}
\symexamp{PAQd}{\PAQd}
\symexamp{PAQu}{\PAQu}
\symexamp{PAQs}{\PAQs}
\symexamp{PAQc}{\PAQc}
\symexamp{PGne}{\PGne}
\symexamp{PAGne}{\PAGne}
\symexamp{PGnGm}{\PGnGm}
\symexamp{PAGnGm}{\PAGnGm}
\symexamp{PGnGt}{\PGnGt}
\symexamp{PAGnGt}{\PAGnGt}
\noindent%
\textcolor{gray}{\rule{\columnwidth}{0.2pt}}\\
\symexamp{AFB}{\AFB}
\symexamp{wangle}{\wangle}
\symexamp{stat}{\stat\hspace{\fill}{\tiny[Includes leading thinspace]}}
\symexamp{syst}{\syst\hspace{\fill}{\tiny[Includes leading thinspace]}}
\symexamp{thy}{\thy\hspace{\fill}{\tiny[Includes leading thinspace]}}
\symexamp{lum}{\lum\hspace{\fill}{\tiny[Includes leading thinspace]}}
\symexamp{kt}{\kt}
\symexamp{BC}{\BC}
\symexamp{bbarc}{\bbarc}
\symexamp{bbbar}{\bbbar}
\symexamp{ccbar}{\ccbar}
\symexamp{bspsiphi}{\bspsiphi}
\symexamp{EE}{\EE}
\symexamp{MM}{\MM}
\symexamp{TT}{\TT}
\symexamp{HGG}{\HGG}
\symexamp{GAMJET}{\GAMJET}
\symexamp{PPTOJETS}{\PPTOJETS}
\symexamp{PPTOGG}{\PPTOGG}
\symexamp{PPTOGAMJET}{\PPTOGAMJET}
\symexamp{MH}{\MH}
\symexamp{RNINE}{\RNINE}
\symexamp{DR}{\DR}
\symexamp{ga}{\ga}
\symexamp{la}{\la}
\symexamp{swsq}{\swsq}
\symexamp{cwsq}{\cwsq}
\symexamp{tanb}{\tanb}
\symexamp{tanbsq}{\tanbsq}
\symexamp{sidb}{\sidb}
\symexamp{alpS}{\alpS}
\symexamp{alpt}{\alpt}
\symexamp{QL}{\QL}
\symexamp{sQ}{\sQ}
\symexamp{sQL}{\sQL}
\symexamp{ULC}{\ULC}
\symexamp{sUC}{\sUC}
\symexamp{sULC}{\sULC}
\symexamp{DLC}{\DLC}
\symexamp{sDC}{\sDC}
\symexamp{sDLC}{\sDLC}
\symexamp{LL}{\LL}
\symexamp{sL}{\sL}
\symexamp{sLL}{\sLL}
\symexamp{ELC}{\ELC}
\symexamp{sEC}{\sEC}
\symexamp{sELC}{\sELC}
\symexamp{sEL}{\sEL}
\symexamp{sER}{\sER}
\symexamp{sFer}{\sFer}
\symexamp{sQua}{\sQua}
\symexamp{sUp}{\sUp}
\symexamp{suL}{\suL}
\symexamp{suR}{\suR}
\symexamp{sDw}{\sDw}
\symexamp{sdL}{\sdL}
\symexamp{sdR}{\sdR}
\symexamp{sTop}{\sTop}
\symexamp{stL}{\stL}
\symexamp{stR}{\stR}
\symexamp{stone}{\stone}
\symexamp{sttwo}{\sttwo}
\symexamp{sBot}{\sBot}
\symexamp{sbL}{\sbL}
\symexamp{sbR}{\sbR}
\symexamp{sbone}{\sbone}
\symexamp{sbtwo}{\sbtwo}
\symexamp{sLep}{\sLep}
\symexamp{sLepC}{\sLepC}
\symexamp{sEl}{\sEl}
\symexamp{sElC}{\sElC}
\symexamp{seL}{\seL}
\symexamp{seR}{\seR}
\symexamp{snL}{\snL}
\symexamp{sMu}{\sMu}
\symexamp{sNu}{\sNu}
\symexamp{sTau}{\sTau}
\symexamp{Glu}{\Glu}
\symexamp{sGlu}{\sGlu}
\symexamp{Wpm}{\Wpm}
\symexamp{sWpm}{\sWpm}
\symexamp{Wz}{\Wz}
\symexamp{sWz}{\sWz}
\symexamp{sWino}{\sWino}
\symexamp{Bz}{\Bz}
\symexamp{sBz}{\sBz}
\symexamp{sBino}{\sBino}
\symexamp{Zz}{\Zz}
\symexamp{sZino}{\sZino}
\symexamp{sGam}{\sGam}
\symexamp{chiz}{\chiz}
\symexamp{chip}{\chip}
\symexamp{chim}{\chim}
\symexamp{chipm}{\chipm}
\symexamp{Hone}{\Hone}
\symexamp{sHone}{\sHone}
\symexamp{Htwo}{\Htwo}
\symexamp{sHtwo}{\sHtwo}
\symexamp{sHig}{\sHig}
\symexamp{sHa}{\sHa}
\symexamp{sHb}{\sHb}
\symexamp{sHpm}{\sHpm}
\symexamp{hz}{\hz}
\symexamp{Hz}{\Hz}
\symexamp{Az}{\Az}
\symexamp{Hpm}{\Hpm}
\symexamp{sGra}{\sGra}
\symexamp{mtil}{\mtil}
\symexamp{rpv}{\rpv}
\symexamp{LLE}{\LLE}
\symexamp{LQD}{\LQD}
\symexamp{UDD}{\UDD}
\symexamp{Lam}{\Lam}
\symexamp{Lamp}{\Lamp}
\symexamp{Lampp}{\Lampp}
\symexamp{MD}{\MD}
\symexamp{Mpl}{\Mpl}
\symexamp{Rinv}{\Rinv}
\\
\ifthenelse{\boolean{cms@external}}{}{\end{multicols}}
\clearpage
\section{Particle symbols\label{pennames}}
\ifthenelse{\boolean{cms@external}}{}{\vspace*{-5mm}
\begin{multicols}{2}
%
\raggedcolumns%
}
%
\noindent%
\symexamp{PAz}{\PAz}
\symexamp{PBm}{\PBm}
\symexamp{PBpm}{\PBpm}
\symexamp{PBp}{\PBp}
\symexamp{PBz}{\PBz}
\symexamp{PB}{\PB}
\symexamp{PDiz}{\PDiz}
\symexamp{PDm}{\PDm}
\symexamp{PDpm}{\PDpm}
\symexamp{PDp}{\PDp}
\symexamp{PDstiiz}{\PDstiiz}
\symexamp{PDstpm}{\PDstpm}
\symexamp{PDstz}{\PDstz}
\symexamp{PDz}{\PDz}
\symexamp{PD}{\PD}
\symexamp{PEz}{\PEz}
\symexamp{PHpm}{\PHpm}
\symexamp{PHz}{\PHz}
\symexamp{PJgy}{\PJgy}
\symexamp{PKeiii}{\PKeiii}
\symexamp{PKgmiii}{\PKgmiii}
\symexamp{PKia}{\PKia}
\symexamp{PKii}{\PKii}
\symexamp{PKi}{\PKi}
\symexamp{PKm}{\PKm}
\symexamp{PKpm}{\PKpm}
\symexamp{PKp}{\PKp}
\symexamp{PKsta}{\PKsta}
\symexamp{PKstb}{\PKstb}
\symexamp{PKstiii}{\PKstiii}
\symexamp{PKstii}{\PKstii}
\symexamp{PKstiv}{\PKstiv}
\symexamp{PKstz}{\PKstz}
\symexamp{PKst}{\PKst}
\symexamp{PKzL}{\PKzL}
\symexamp{PKzS}{\PKzS}
\symexamp{PKzeiii}{\PKzeiii}
\symexamp{PKzgmiii}{\PKzgmiii}
\symexamp{PKz}{\PKz}
\symexamp{PK}{\PK}
\symexamp{PLpm}{\PLpm}
\symexamp{PLz}{\PLz}
\symexamp{PN}{\PN}
\symexamp{PNa}{\PNa}
\symexamp{PNb}{\PNb}
\symexamp{PNc}{\PNc}
\symexamp{PNd}{\PNd}
\symexamp{PNe}{\PNe}
\symexamp{PNf}{\PNf}
\symexamp{PNg}{\PNg}
\symexamp{PNh}{\PNh}
\symexamp{PNi}{\PNi}
\symexamp{PNj}{\PNj}
\symexamp{PNk}{\PNk}
\symexamp{PNl}{\PNl}
\symexamp{PNm}{\PNm}
\symexamp{PSHpm}{\PSHpm}
\symexamp{PSHz}{\PSHz}
\symexamp{PSWpm}{\PSWpm}
\symexamp{PSZz}{\PSZz}
\symexamp{PSe}{\PSe}
\symexamp{PSgg}{\PSgg}
\symexamp{PSgm}{\PSgm}
\symexamp{PSgn}{\PSgn}
\symexamp{PSgt}{\PSgt}
\symexamp{PSgxpm}{\PSgxpm}
\symexamp{PSgxz}{\PSgxz}
\symexamp{PSg}{\PSg}
\symexamp{PSq}{\PSq}
\symexamp{PWR}{\PWR}
\symexamp{PWm}{\PWm}
\symexamp{PWpr}{\PWpr}
\symexamp{PWp}{\PWp}
\symexamp{PW}{\PW}
\symexamp{PZLR}{\PZLR}
\symexamp{PZgc}{\PZgc}
\symexamp{PZge}{\PZge}
\symexamp{PZgy}{\PZgy}
\symexamp{PZi}{\PZi}
\symexamp{PZz}{\PZz}
\symexamp{PaBz}{\PaBz}
\symexamp{PaB}{\PaB}
\symexamp{PaDz}{\PaDz}
\symexamp{PaD}{\PaD}
\symexamp{PaKz}{\PaKz}
\symexamp{PaSq}{\PaSq}
\symexamp{PagL}{\PagL}
\symexamp{PagOp}{\PagOp}
\symexamp{PagSm}{\PagSm}
\symexamp{PagSp}{\PagSp}
\symexamp{PagSz}{\PagSz}
\symexamp{PagXp}{\PagXp}
\symexamp{PagXz}{\PagXz}
\symexamp{Pagne}{\Pagne}
\symexamp{Pagngm}{\Pagngm}
\symexamp{Pagngt}{\Pagngt}
\symexamp{Paii}{\Paii}
\symexamp{Pai}{\Pai}
\symexamp{Pap}{\Pap}
\symexamp{Paqb}{\Paqb}
\symexamp{Paqc}{\Paqc}
\symexamp{Paqd}{\Paqd}
\symexamp{Paqs}{\Paqs}
\symexamp{Paqt}{\Paqt}
\symexamp{Paqu}{\Paqu}
\symexamp{Paq}{\Paq}
\symexamp{Paz}{\Paz}
\symexamp{Pbgcia}{\Pbgcia}
\symexamp{Pbgciia}{\Pbgciia}
\symexamp{Pbgcii}{\Pbgcii}
\symexamp{Pbgci}{\Pbgci}
\symexamp{Pbgcza}{\Pbgcza}
\symexamp{Pbgcz}{\Pbgcz}
\symexamp{Pbi}{\Pbi}
\symexamp{PcgLp}{\PcgLp}
\symexamp{PcgS}{\PcgS}
\symexamp{PcgXp}{\PcgXp}
\symexamp{PcgXz}{\PcgXz}
\symexamp{Pcgcii}{\Pcgcii}
\symexamp{Pcgci}{\Pcgci}
\symexamp{Pcgcz}{\Pcgcz}
\symexamp{Pcgh}{\Pcgh}
\symexamp{Pem}{\Pem}
\symexamp{Pep}{\Pep}
\symexamp{Pe}{\Pe}
\symexamp{Pfia}{\Pfia}
\symexamp{Pfib}{\Pfib}
\symexamp{Pfiia}{\Pfiia}
\symexamp{Pfiib}{\Pfiib}
\symexamp{Pfiic}{\Pfiic}
\symexamp{Pfiid}{\Pfiid}
\symexamp{Pfiipr}{\Pfiipr}
\symexamp{Pfii}{\Pfii}
\symexamp{Pfiv}{\Pfiv}
\symexamp{Pfi}{\Pfi}
\symexamp{Pfza}{\Pfza}
\symexamp{Pfzb}{\Pfzb}
\symexamp{Pfz}{\Pfz}
\symexamp{PgD}{\PgD}
\symexamp{PgDa}{\PgDa}
\symexamp{PgDb}{\PgDb}
\symexamp{PgDc}{\PgDc}
\symexamp{PgDd}{\PgDd}
\symexamp{PgDe}{\PgDe}
\symexamp{PgDf}{\PgDf}
\symexamp{PgDh}{\PgDh}
\symexamp{PgDi}{\PgDi}
\symexamp{PgDj}{\PgDj}
\symexamp{PgDk}{\PgDk}
\symexamp{PgL}{\PgL}
\symexamp{PgLa}{\PgLa}
\symexamp{PgLb}{\PgLb}
\symexamp{PgLc}{\PgLc}
\symexamp{PgLd}{\PgLd}
\symexamp{PgLe}{\PgLe}
\symexamp{PgLf}{\PgLf}
\symexamp{PgLg}{\PgLg}
\symexamp{PgLh}{\PgLh}
\symexamp{PgLi}{\PgLi}
\symexamp{PgLj}{\PgLj}
\symexamp{PgLk}{\PgLk}
\symexamp{PgLl}{\PgLl}
\symexamp{PgLm}{\PgLm}
\symexamp{PgO}{\PgO}
\symexamp{PgOm}{\PgOm}
\symexamp{PgOma}{\PgOma}
\symexamp{PgS}{\PgS}
\symexamp{PgSa}{\PgSa}
\symexamp{PgSb}{\PgSb}
\symexamp{PgSc}{\PgSc}
\symexamp{PgSd}{\PgSd}
\symexamp{PgSe}{\PgSe}
\symexamp{PgSf}{\PgSf}
\symexamp{PgSg}{\PgSg}
\symexamp{PgSh}{\PgSh}
\symexamp{PgSi}{\PgSi}
\symexamp{PgSm}{\PgSm}
\symexamp{PgSp}{\PgSp}
\symexamp{PgSz}{\PgSz}
\symexamp{PgU}{\PgU}
\symexamp{PgUa}{\PgUa}
\symexamp{PgUb}{\PgUb}
\symexamp{PgUc}{\PgUc}
\symexamp{PgUd}{\PgUd}
\symexamp{PgUe}{\PgUe}
\symexamp{PgUf}{\PgUf}
\symexamp{PgX}{\PgX}
\symexamp{PgXa}{\PgXa}
\symexamp{PgXb}{\PgXb}
\symexamp{PgXc}{\PgXc}
\symexamp{PgXd}{\PgXd}
\symexamp{PgXe}{\PgXe}
\symexamp{PgXm}{\PgXm}
\symexamp{PgXz}{\PgXz}
\symexamp{Pgfa}{\Pgfa}
\symexamp{Pgfiii}{\Pgfiii}
\symexamp{Pgf}{\Pgf}
\symexamp{Pgg}{\Pgg}
\symexamp{Pgha}{\Pgha}
\symexamp{Pghb}{\Pghb}
\symexamp{Pghpr}{\Pghpr}
\symexamp{Pgh}{\Pgh}
\symexamp{Pgmm}{\Pgmm}
\symexamp{Pgmp}{\Pgmp}
\symexamp{Pgm}{\Pgm}
\symexamp{Pgne}{\Pgne}
\symexamp{Pgngm}{\Pgngm}
\symexamp{Pgngt}{\Pgngt}
\symexamp{Pgoa}{\Pgoa}
\symexamp{Pgob}{\Pgob}
\symexamp{Pgoiii}{\Pgoiii}
\symexamp{Pgo}{\Pgo}
\symexamp{Pgpa}{\Pgpa}
\symexamp{Pgpii}{\Pgpii}
\symexamp{Pgpm}{\Pgpm}
\symexamp{Pgppm}{\Pgppm}
\symexamp{Pgpp}{\Pgpp}
\symexamp{Pgpz}{\Pgpz}
\symexamp{Pgp}{\Pgp}
\symexamp{Pgra}{\Pgra}
\symexamp{Pgrb}{\Pgrb}
\symexamp{Pgriii}{\Pgriii}
\symexamp{Pgr}{\Pgr}
\symexamp{Pgt}{\Pgt}
\symexamp{Pgya}{\Pgya}
\symexamp{Pgyb}{\Pgyb}
\symexamp{Pgyc}{\Pgyc}
\symexamp{Pgyd}{\Pgyd}
\symexamp{Pgy}{\Pgy}
\symexamp{Phia}{\Phia}
\symexamp{Pn}{\Pn}
\symexamp{Pp}{\Pp}
\symexamp{Pqb}{\Pqb}
\symexamp{Pqc}{\Pqc}
\symexamp{Pqd}{\Pqd}
\symexamp{Pqs}{\Pqs}
\symexamp{Pqt}{\Pqt}
\symexamp{Pqu}{\Pqu}
\symexamp{Pq}{\Pq}
\symexamp{PsDipm}{\PsDipm}
\symexamp{PsDm}{\PsDm}
\symexamp{PsDp}{\PsDp}
\symexamp{PsDst}{\PsDst}
\noindent%
\textcolor{gray}{\rule{\columnwidth}{0.2pt}}\\
\noindent%
Future PENNAMES\hspace{\fill}{\tiny{include \verb|\xspace|}}\\
\symexamp{PH}{\PH}
\symexamp{PJGy}{\PJGy}
\symexamp{PBzs}{\PBzs}
\symexamp{Pg}{\Pg}
\symexamp{PSg}{\PSg}
\symexamp{PSQ}{\PSQ}
\symexamp{PXXG}{\PXXG}
\symexamp{PXXSG}{\PXXSG}
\symexamp{PSGcp}{\PSGcp}
\symexamp{PSGc}{\PSGc}
\symexamp{PSGcz}{\PSGcz}
\symexamp{PSGczDo}{\PSGczDo}
\symexamp{PSGczDt}{\PSGczDt}
\symexamp{PSGcpm}{\PSGcpm}
\symexamp{Pl}{\Pl}
\symexamp{PAl}{\PAl}
\symexamp{PGnl}{\PGnl}
\symexamp{PAGnl}{\PAGnl}
\symexamp{PQtpr}{\PQtpr}
\symexamp{PAQtpr}{\PAQtpr}
\symexamp{PQbpr}{\PQbpr}
\symexamp{PAQbpr}{\PAQbpr}
\symexamp{PGg}{\PGg}
\symexamp{PKzS}{\PKzS}
\symexamp{PBs}{\PBs}
\symexamp{PSQt}{\PSQt}
\symexamp{PZpr}{\PZpr}
\symexamp{PGn}{\PGn}
\symexamp{PAGn}{\PAGn}
\ifthenelse{\boolean{cms@external}}{}{\end{multicols}}
}
\section{OS X specific instructions}
These instructions are based on a clean installation of Mac OS X 10.7.3 (Lion). This release has current versions of both perl and svn.

Download the TeXLive 2011 installation, \url{http://mirror.ctan.org/systems/mac/mactex/MacTeX.mpkg.zip}, and install (if not already done). This is a relatively large installation.

If a simple \textit{kinit Your\_CERN\_Username@CERN.CH} doesn't allow you to access the svn repository in the standard fashion, you can follow the instructions at \url{http://svn.web.cern.ch/svn/howto.php#accessing-sshlinux} to set up an ssh key pair. I tried using the keychain, but it isn't supported in the included version of svn. There are commercial versions available with GUIs, and maybe even free versions---I didn't look very hard---but they are not necessary.

Then follow the general instructions in \url{https://svnweb.cern.ch/cern/wsvn/tdr2/papers/XXX-08-000/trunk/XXX-08-000_temp.pdf} (this document) and \url{https://svnweb.cern.ch/cern/wsvn/tdr2/utils/trunk/general/notes_for_authors.pdf}.

Additional style files are required to generate documents in the journal formats, and many of these need to be installed individually.
%%% DO NOT ADD \end{document}!

