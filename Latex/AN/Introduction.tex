\section{Introduction}\label{sec:Introduction}

The major breakthrough in modern experimental particle physics was the Higgs boson discovery by the LHC experiments ALTAS and CMS in 2012.
The discovered particle is well compatible with Standard Model (SM) Higgs
mechanisms prediction: the only unknown parameter, the boson's mass, has been
measured to be  close to 125 \GeV. Nevertheless,  in order to identify whether the SM Higgs sector is complete, precise measurements on the Higgs boson properties, as the coupling strengths, the $CP$ structure and the transverse momentum, are needed. 
A complementary and also important strategy is the search of additional heavy
scalars, that would prove the presence of  beyond-the-SM (BSM) physics in
the form of a non-minimal Higgs sector. The existence of sibling  Higgs boson,
denoted $X$, is motivated in many BSM scenarios, so the research in the full
mass range accessible at LHC remains one of the main objectives of the experimental community.\\
\newline
The search of such sibling Higgs boson has been performed using Run-I and
early Run-II data in many different decay channels and an upper limit on its
cross section has been provided. With the full 2016 data collected by CMS
experiment at 13 \TeV, approximately  36 \fbinv, it is now possible to set a very tight upper limit on the sibling Higgs boson cross section.
One of the most sensitive decay channel for the $X$ boson is in a couple of W bosons for masses above 200 \GeV. Only the fully leptonic final state, 2$\ell$2$\nu$
  (with $\ell =$ e or \PGm), is considered in this analysis: indeed this channel  has a clear signature due to the presence of the two isolated leptons and a moderate missing-transverse-energy (MET) that is a indirect evidence of the neutrinos presence.\\
\newline
The search is similar to the previous analysis \cite{CMS-PAS-HIG-16-023} with several
improvements. The mass range is extended to 3000 \GeV (was 1000 \GeV)  and new
events categories have been introduced, optimised for the gluon-gluon fusion (ggF) and for the vector-boson fusion (VBF) production mechanisms.
The signal is interpreted in terms of the electroweak (EW) singlet model
including a detailed simulation of the on interference between $X$ signal, SM Higgs boson $H_{125}$ and WW backgrounds. 
