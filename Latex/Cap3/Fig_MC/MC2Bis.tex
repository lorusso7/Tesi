
\chapter{Generatori Monte Carlo}
\thispagestyle{empty}

Come abbiamo visto nelle collisioni fra protoni di alta energia,  sono generalmente presenti, nello stato finale, centinaia di particelle. 
Data la complessità degli eventi è necessario l'uso di generatori Monte Carlo, ovvero di  programmi che permettono di simulare il risultato realistico delle collisioni assumendo un certo modello per i processi coinvolti e che sono oggetto dello  studio.
L'uso dei generatori Monte Carlo si rende necessario perché è impossibile predire  \textit{a priori} che cosa accade evento-per-evento: infatti, come noto, in meccanica quantistica   si può calcolare solo la probabilità di avere un certo risultato.
La  simulazione di un evento viene effettuata a passi successivi \cite{Sjostrand:2006su, Buckley:2011ms}, come schematizzato in Fig. \ref{general}, suddividendo in tal modo il problema in più parti di complessità inferiore.
\begin{figure}
\centering
\includegraphics[scale= 1]{generalMC}
\caption{\textit{Rappresentazione schematica di un evento generato all'interno di un generatore di eventi. I partoni provenienti dai protoni indicenti partecipano sia all'hard process sia alle interazioni multiple. Successivamente si ha l'adronizzazione.}}
\label{general}
\end{figure}
I vari passaggi sono qui riassunti:
\begin{itemize}
\item \textit{Hard process}: i protoni incidenti sono costituiti da partoni (\textit{quark} e gluoni) e l'\textit{hard process} consiste in una  collisione fra due partoni, provenienti da adroni differenti.
%, come nel caso di produzione del bosone di Higgs prodotto dalla fusione di due gluoni: $gg \rightarrow h^0$. 
L'elemento di matrice del processo è  calcolabile in modo perturbativo e spesso  viene calcolato solo l'ordine perturbativo più basso detto \textit{leading order }(LO).
\item \textit{Parton shower}: i partoni, entranti o uscenti, che partecipano all'\textit{hard process} possono emettere gluoni:  infatti in analogia con l'interazione elettromagnetica,  una particella con carica di colore accelerata   può irradiare per \textit{bremsstrahlung}.
I gluoni a loro volta possono produrre coppie \textit{quark-antiquark} generando così  sciami di partoni o \textit{parton shower}. 
%Nell'evoluzione del \textit{parton shower} si ha dunque l'emissione di partoni aggiuntivi che possono essere emessi sia i modo collineare, rispetto al partone uscente, sia in modo \textit{soft}. 
L'emissione di partoni aggiuntivi avviene principalmente in modo collineare rispetto al partone iniziale e ad energie progressivamente minori.
Nello stato finale si avrà  un insieme di partoni, detto \textit{jet}, localizzato in zona collineare al partone iniziale.
%, che trasporta la gran parte dell'energia del    \textit{parton shower}. 
Questo processo probabilistico può essere simulato  come  un processo di Markov \cite{Markov1907eot} ed è implementato negli algoritmi di  \textit{parton shower} dei quali parleremo in seguito. 
\item Interazioni multiple: i protoni incidenti sono stati legati di partoni fortemente interagenti fra loro. In una singola collisione, può dunque capitare di avere più coppie di partoni che interagiscono fra loro. Si dice in questo caso che si hanno  interazioni multiple in aggiunta all'\textit{hard process}.
\item Adronizzazione: Nell'evoluzione dell'evento i partoni sono via via  generati con impulsi relativi sempre più bassi. Per valori dell'impulso dell'ordine di 1 GeV le forze di confinamento prevalgono. A queste scale di energia, come abbiamo spiegato nella sezione 1.1, la teoria perturbativa fallisce nella descrizione, quindi si ricorre a modelli non-perturbativi
che descrivono la formazione degli adroni reali. Questo processo di \textit{adronizzazione}  preserva la struttura del \textit{jet} che può perciò essere osservato sperimentalmente.
 \item Decadimento delle particelle instabili: molte delle particelle prodotte nel processo primario o nell'adronizzazione sono  instabili  e vengono fatte decadere a meno che non possano interagire direttamente con il rivelatore.
\end{itemize}
I metodi di simulazione Monte Carlo permettono di considerare questi passi in modo sequenziale:  il risultato di ogni passo è il punto di partenza del passo successivo.
%definendo  un insieme di regole ben definite e che sono utilizzate in modo iterativo, per costruire differenti stati finali. 
Alla fine in un singolo evento sono presenti centinaia di particelle ognuna delle quali ha una decina di gradi di libertà (massa, sapore, impulso, vita media, spin, vertice di produzione, ecc.), dunque si intuisce l'elevato numero di parametri che entrano in gioco e che devono essere simulati per ogni  evento.
Lo scopo finale è fornire una descrizione  realistica di quello che accade nelle collisioni ad alta energia, per poter così confrontare il modello implementato   Monte Carlo con i dati sperimentali e vedere se siamo  di fronte a eventi inaspettati, che potrebbero indicare nuova fisica.
Schematicamente la sezione d'urto dello stato finale è data da
\begin{equation}
 \sigma_{final \: state}=\sigma_{hard \: process} \: \mathcal{P}_{tot, \:hard \: process \rightarrow final \:state} \: \mbox{,}\end{equation}
integrata su tutto lo spazio delle fasi e sommata su tutti i possibili stati finali (si pensi per esempio alla produzione di due o più \textit{jet}). Questa è la quantità misurabile associata all'{\textit{hard process}}. \\
Lo spettro dei generatori Monte Carlo è molto ampio: si va dai generatori \textit{general-purpose} come P{\footnotesize YTHIA} \cite{bib:pythia}, Herwig++ \cite{bib:herwig} e S{\footnotesize HERPA} \cite{bib:sherpa}
a quelli specializzati  nel calcolo dell'elemento di matrice, come Madgraph\_aM{\footnotesize C@NLO} \cite{bib:madgraph} e  P{\footnotesize OWHEG} \cite{Oleari:2010nx}.




\section{\textit{Hard process}}

In molti processi di interesse a LHC entrano in gioco alti impulsi, per produrre particelle con elevata massa o \textit{jet} molto energetici.  La simulazione di questi eventi è la parte centrale dei generatori Monte Carlo.
La sezione d'urto per un processo di \textit{scattering} $ab \rightarrow n$ è data \cite{Buckley:2011ms} da,

\begin{eqnarray}
 \sigma & = & \sum_{a,b}  \int_{0}^{1} \, dx_{a} dx_b \int f_{a}^{h_1} (x_a , \mu_F) f_{b}^{h_2} (x_b , \mu_F) \: d \hat{\sigma}_{ab \rightarrow n}(\mu_F , \mu_R)  \nonumber \\
& = & \sum_{a,b}  \int_{0}^{1} \, dx_{a} dx_b \int d \Phi_n  f_{a}^{h_1} (x_a , \mu_F) f_{b}^{h_2} (x_b , \mu_F) \nonumber \\
 & \times& \frac{1}{2\hat{s}} 
 | \mathcal{M}_{ab \rightarrow n} 
(\Phi_n , \mu_F , \mu_R)|^2  \: \mbox{,} \end{eqnarray}
dove




\begin{itemize}
\item $f_{a}^{h} (x , \mu)$ sono le funzioni di distribuzione dei partoni (PDF) che dipendono dalla frazione $x$ dell'energia del partone $a$ (variabile di Bjorken)  rispetto all'adrone $h$, e dalla scala di fattorizzazione $\mu_F$, che avevamo introdotto già nella Eq. 1.5.
\item  $\hat{\sigma}_{ab \rightarrow n}$ è la sezione d'urto partonica del processo $ab \rightarrow n$. 
La sezione d'urto differenziale totale è data dal prodotto del corrispondente elemento di matrice al quadrato, $| \mathcal{M}_{ab \rightarrow n} |^2 $, e dal flusso di partoni incidenti $1/(2 \hat{s})= 1/(2 x_a x_b s)$, dove $\sqrt{s}$ è l'energia del centro di massa del sistema.
\item L'elemento di matrice $| \mathcal{M}_{ab \rightarrow n}  (\Phi_n , \mu_F , \mu_R) |^2 $ può essere scritto come la somma su tutti i diagrammi di  Feynman,

\begin{equation}
\mathcal{M}_{ab \rightarrow n}= \sum_{i} \mathcal{F}_{ab \rightarrow n}^{(i)} \: \mbox{.} \end{equation}

\item $d\Phi_n$ è il differenziale dello spazio delle fasi per  $n$ particelle nello stato finale.
\end{itemize}
 Lo spazio delle fasi non sarà tutto lo spazio fisico possibile ma conterrà dei tagli per due motivi: da una parte i tagli  rifletteranno la geometria e  l'accettanza del rivelatore; dall'altra parte è quasi sempre necessario un taglio sull'impulso trasverso delle particelle prodotte nel processo  per evitare divergenze nel calcolo della sezione d'urto\footnote{Si può immaginare di avere una singolarità analoga a quella che si ha nello \textit{scattering} coulombiano classico.}.
In generale il calcolo dell'elemento di matrice richiederebbe il calcolo di tutti i diagrammi di Feynamn che infatti  crescono in modo fattoriale (Fig. \ref{fatt})  con il numero di particelle nello stato finale.
\begin{figure}[h]
\centering
\includegraphics[scale= 2.5]{fattoriale}
\caption{ \textit{Andamento del numero di diagrammi di Feynman al crescere del numero $n$ di gluoni nel processo $e^+ e^- \rightarrow q \bar{q} + ng$}.}
\label{fatt}
\end{figure}
La maggior parte dei generatori di eventi è in grado di calcolare il \textit{leading order} dell'elemento di matrice  dei processi noti all'interno del Modello Standard del tipo $2 \rightarrow 1$,  $2 \rightarrow 2$ ed anche  $2 \rightarrow 3$ \cite{bib:madgraph}.  \\
Tuttavia se ci fermassimo al primo ordine perturbativo, si avrebbe una descrizione solo approssimativa del processo: infatti gli ordini successivi comportano delle correzioni importanti sia alla forma delle distribuzioni che alla sezione d’urto totale. Il LO risulta  utile per un primo studio ma, dove è possibile, è importante   valutare il \textit{next-to-leading-order} (NLO)\footnote{Per alcuni processi particolarmente importanti, ad esempio $gg \rightarrow H$, sono  addirittura disponibili i calcoli al  \textit{next-next-to-leading-order} (NNLO).}.\\
La sezione d'urto calcolata al NLO  è composta da tre parti: dal LO o parte di Born, e dalle correzioni reali e virtuali all'emissione (Fig. \ref{nlofig}),

\begin{equation}
 \label{xsecNLO}
 d\sigma^{NLO} =  d \tilde{\Phi}_n [\mathcal{B} (  \tilde{\Phi}_n  ) + \alpha_s \mathcal{V}(  \tilde{\Phi}_n  ) ] +  d \tilde{\Phi}_{n+1} \alpha_s \mathcal{R}(\tilde{\Phi}_{n+1}  ) \: \mbox{,}   \end{equation}
 dove i termini $\mathcal{B}$, $\mathcal{R}$ e $\mathcal{V}$ denotano rispettivamente la parte di Born, la parte reale e quella  virtuale. L'integrale dovrà essere effettuato su tutte le $n$ o $n+1$  particelle dello stato finale e sulle variabili di Bjorken relative ai partoni incidenti.
%Le difficoltà in questo caso sono dovute alle divergenze ``ultraviolette'' e ``infrarosse''. 
Supponiamo che nell'approssimazione di  Born  il processo sia  del tipo $2 \rightarrow 2$. Se si volesse andare all'ordine successivo, NLO, si deve tenere del grafico con un partone aggiuntivo nello stato finale, processo $2 \rightarrow 3$, e della correzione virtuale con un  \textit{loop} nel processo $2 \rightarrow 2$ .
E' da notare che la sezione d'urto per processi del tipo $2 \rightarrow 3$ è divergente quando l'energia di uno dei  partoni tende a zero (divergenza \textit{soft}) oppure quando due  partoni sono collineari (divergenza collineare). 


\begin{figure}
\centering
\includegraphics[scale=0.22]{nlo2}
\caption{ \textit{Esempi di diagrammi di Feynman (a) di Born, (b) reale, (c) virtuale. }}
\label{nlofig}
\end{figure}

 
\section{\textit{Parton shower}}
In una collisione fra partoni  una carica di colore viene accelerata, quindi sarà presente emissione di \textit{bremsstrahlung}. Quando si studia un processo del tipo $2\rightarrow n$, dove $n$ rappresenta il numero di partoni nello stato finale, l'elemento di matrice al LO (anche detto \textit{tree-level}) avrà delle divergenze nel caso collineare e \textit{soft}. In particolare i processi che soffrono di questo tipo di divergenza sono $q \rightarrow qg$, $\bar{q} \rightarrow \bar{q}g$, $g \rightarrow gg$: i primi sono i processi analoghi a  $e \rightarrow e\gamma$ in QED, mentre il terzo è dovuto al fatto che la QCD non è una teoria abeliana. Il processo  $g \rightarrow q \bar{q}$ invece non ha questo tipo  divergenze.
Le divergenze dell'elemento di matrice \textit{tree-level} possono essere rimosse introducendo nel calcolo le correzioni virtuali, che però saranno all'ordine successivo; questi calcoli dunque risultano particolarmente complessi e sono possibili solo per un  numero limitato  di processi.  Gli algoritmi di \textit{parton shower}  \cite{Sjostrand:2006su} offrono un modo alternativo e abbastanza semplice per eliminare le divergenze collineari e \textit{soft} attraverso:
\begin{itemize}
\item una struttura iterativa che combina in un unico stato finale multi-partonico i tre stati che soffrono delle divergenze,
\item l'introduzione del fattore di forma di Sudakov.
\end{itemize}
I partoni entranti o uscenti, che sono lontani (temporalmente) dall'\textit{hard process}, sono detti \textit{on-shell} perchè il modulo del loro quadrimpulso è uguale alla massa a riposo. Tuttavia più ci si avvicina  all'interazione, a causa del principio di indeterminazione ($\Delta E \Delta t \sim \hbar$),  i partoni possono essere in uno stato detto  \textit{off-shell}, cioè il modulo del loro quadrimpulso non corrisponde alla massa a riposo. Per questo motivo sono in grado di emettere altri partoni e in particolare quanto più sono vicini allo scattering quanto maggiore può essere l’energia dei partoni emessi. Se l’emissione avviene prima dello scattering si parla di radiazione di stato iniziale (ISR), mentre dopo l’interazione di parla di radiazione di stato finale (FSR). \\
Ogni partone è caratterizzato da una scala di ``virtualità'' $Q^2$ che corrisponde in modo approssimato a una scala di  ordine temporale dello sciame. E' importante sottolineare che sono disponibili differenti definizioni per $Q^2$; tuttavia indipendentemente dalla convenzione scelta,  la scala $Q^2$ aumenta  avvicinandosi all'\textit{hard process}, quindi nell'ISR, e diminuisce allontanandosi, nell'FSR. Se prendiamo come esempio la FSR, l’evoluzione inizia ad una scala $Q^2_{max}$ che è legata all’hard process e termina quando si raggiunge una scala limite, $Q_0$, che sarà dell'ordine di 1 GeV.\\
La scelta più comune utilizzata è porre $Q^2=p^2=E^2- |\vec{p}\,|^2$. Con questa convenzione in un processo   di tipo $a \rightarrow bc$, nel caso di FSR, $Q^2 >0$, ovvero di tipo \textit{time-like}, e diminuirà fino a che si raggiunge la scala limite $Q_0$. 
Le cose sono più complicate nel caso ISR: in questo caso $a$ e $b$, supposti qui \textit{off-shell}, hanno  $p^2$ di tipo \textit{space-like}, quindi si ridefinisce $Q_i^2=-m_i^2$ in modo da garantire l'ordine crescente di $Q^2$, i.e. $Q_b ^2 > Q_a ^2$. 
Di contro $c$ non parteciperà all'\textit{hard process} e  avrà $p^2>0$ e quindi il suo sciame evolverà come quello del FSR.
\paragraph{Radiazione di stato finale}


Nell'approccio col \textit{parton shower}  lo stato di radiazione finale è modellizzato attraverso una serie di processi divisionali  del tipo $a \rightarrow bc$.   
Ciò è evidente dal processo   $q \bar{q}g$, Fig. \ref{nlofig} (b), dove le correzioni reali dell'elemento di matrice al primo ordine corrispondono all'emissione di un gluone. L'evoluzione dello sciame è descritta da due parametri: la frazione di energia portata da uno dei due partoni uscenti, $z=E_b/E_a$, e la variabile di ordine $t$. Come abbiamo detto una  possibile scelta per $t$ è la virtualità $Q_a^2$ del partone incidente.
%Per il processo con stato finale   $q \bar{q}g$, si ha:
%\begin{equation}
%1-x_{\bar{q}}= \frac{ m_{qg}^2}{s}= \frac{ Q^2}{s}  \: \mbox{,}   \end{equation}
%dove $x_i=2E_i/ \sqrt{s}$ ($i=q$, $\bar{q}$, $g$) e $ m_{qg}$ la massa invariante della coppia $qg$. Dunque per $x_{\bar{q}} \rightarrow 1$ (ovvero $E_{\bar{q}}\rightarrow %\sqrt{s}/2$), la virtualità del \textit{quark} intermedio $Q^2\rightarrow 0$, ciò corrisponde alla regione collineare dove la separazione fra $q$ e $g$ si annulla.
%In questo limite $x_q$ corrisponde a $z\approx \frac{E_q}{E_{qg}}$.\\
Nel limite collineare la probabilità di divisione $d \mathcal{P}_{a \rightarrow bc}$, espressa in termini di $z$ e $t=\ln(Q^2/\Lambda^2)$ è data da:
\begin{equation}
 d \mathcal{P}_{a \rightarrow bc}= \sum_{bc} \frac{\alpha_{abc}}{2 \pi}\: {P}_{a \rightarrow bc} \:dt dz  \: \mbox{,} \label{prob}  \end{equation}
dove $dt=\frac{d Q^2}{Q^2}$, $\alpha_{abc}$ è la costante di accoppiamento che regola la il processo di divisione e ${P}_{a \rightarrow bc}$ è detto \textit{kernel splitting}; queste sono funzioni  universali e valgono nel limite collineare:

\begin{eqnarray}
P_{q \rightarrow qg    }&=& \frac{4}{3} \frac{1+z^2}{1-z} \mbox{,} \nonumber \\ 
 P_{g \rightarrow gg }&=& 3 \frac{(1-z(1-z))^2}{z(1-z)}    \mbox{,} \\ 
P_{g \rightarrow q\bar{q} }&=& \frac{n_f}{2} (z^2+ (1-z)^2)   \mbox{,} \nonumber \end{eqnarray}
dove $n_f$ è il numero di sapori dei \textit{quark}.
Tuttavia la probabilità così valutata  è superiore all'unità perché soffre  delle stesse divergenze dell'elemento di matrice al LO. Infatti l'espressione \ref{prob} è valutata in approssimazione collineare. In particolare si hanno due tipi di divergenze: collineari, dovute alla dipendenza di tipo $1/Q^2$, e \textit{soft} che corrisponde al limite $z=1$.\\
Per ovviare a ciò, nell'approccio del \textit{parton shower},  come prima cosa si valuta la probabilità di divisione fra $t$ e $t +dt$; questa si ottiene integrando l'Eq \ref{prob} su tutti i possibili  $z$ compresi nell'intervallo $[z_{min}(t), \: z_{max}(t)]$:
\begin{equation}
 d \mathcal{P}_{a \rightarrow bc}= \left( \sum_{bc} \int_{z_{min}(t^{'})}^{{z_{max}(t^{'})}}  \frac{\alpha_{abc}}{2 \pi}\: {P}_{a \rightarrow bc} \:dt dz \right) dt  \: \mbox{.}   \end{equation}
Come in altre situazioni fisiche\footnote{Per esempio il decadimento radioattivo.} e non solo, la probabilità che accada qualcosa a $t$ è data dalla probabilità che ciò avvenga fra  $t$ e $t +dt$, moltiplicata per la probabilità che ciò non sia già avvenuto fra l'istante iniziale $t_0$ e $t$.
In questo caso allora la probabilità di avere una divisione a $t$ è:
\begin{equation}
 d \mathcal{P}_{a}^{\mbox{\footnotesize{FSR}}}(t)=   d \mathcal{P}_{a} \cdot \mbox{exp} \left(   -\sum_{bc} \int_{t_0}^t dt^{'}  \int_{z_{min}(t^{'})}^{{z_{max}(t^{'})}} \frac{\alpha_{abc}}{2 \pi} P_{a \rightarrow bc}(z) dz \right) \mbox{,}\end{equation}
dove $t_{0}$ è la scala di partenza dello sciame. Il termine esponenziale è detto fattore di forma di Sudakov e rappresenta, come intuibile, la probabilità di non divisione.  Se lo si vuole interpretare in termini di diagrammi di Feynman questo rappresenta le correzioni virtuali dell'elemento di matrice LO.\\
Tutto questo procedimento può essere combinato insieme  per avere più emissioni  a differenti passi successivi: si avrà così uno sciame di partoni che sarà ordinato in $Q$ decrescente. Infine è importante sottolineare che la descrizione fornita dal \textit{parton shower} è corretta nel caso si abbiano \textit{jet} collineari e fallisce in configurazioni in cui sono presenti partoni ben separati.
   

\paragraph{Radiazione di stato iniziale}
L'evoluzione della radiazione di stato iniziale è molto più complicata rispetto a quella di stato finale. Infatti \textit{quark} e gluoni sono continuamente emessi e riassorbiti all'interno dei protoni incidenti. Ciò significa che quando avviene l'\textit{hard scattering} la radiazione di stato iniziale è già presente.
Si potrebbe semplicemente pensare di simulare ISR partendo dai partoni \textit{on-shell} prima dell'interazione e facendoli evolvere a scale $Q^2$  sempre più elevate fino a raggiungere  un \textit{hard process}. Questo approccio però è molto inefficiente perché risulta particolarmente raro simulare il processo di interesse dato che  avrebbe la stessa probabilità che ha in natura. Si utilizza allora nei generatori di eventi un approccio differente: come prima cosa viene prodotto l'\textit{hard process} e successivamente si prova a ricostruire all'indietro cosa può essere avvenuto. Questo procedimento prende il nome di ``evoluzione all'indietro'', Fig. \ref{isr} .
Consideriamo, come nel caso FSR, il processo di tipo $a \rightarrow bc$ e valutiamo in questo caso la probabilità che un partone $b$ possa essere stato prodotto dal partone $a$. E' necessario introdurre la funzione di densità partonica; questa  evolve in accordo con l'equazione di DGLAP \cite{Altarelli:1977zs}, 


\begin{figure}
\centering
\includegraphics[scale= 0.8]{isr}
\caption{ \textit{Evoluzione dello stato iniziale. La linea in grassetto corrisponde al partone che subirà l'hard process (rappresentato da una croce). Le linee sottili rappresentano i partoni che non possono ricombinarsi, mentre quelle tratteggiate sono fluttuazioni che possono o non possono ricombinarsi.   }}
\label{isr}
\end{figure}

\begin{equation}
  \frac{d f_b(x, \:t)}{dt}= \sum_{ac} \int_x ^1 \frac{d x^{'}}{x^{'}} \: f_a(x^{'},t) \:\frac{\alpha_{abc}}{2\pi} \:P_{a \rightarrow bc} \:(\frac{x}{x^{'}}) \mbox{,}\end{equation}
dove $f_{a,b}(x, \:t)$ sono le PDF del partone $a$, $b$, che ha ha frazione $x$ dell'impulso del protone incidente e scala $t=\mbox{ln}(Q^2/ \Lambda^2) $, mentre $P_{a \rightarrow bc}$ è  la funzione di \textit{kernel splitting}.\\
Nell'evoluzione all'indietro la probabilità che il partone $b$ sia stato generato da $a$  nell'intervallo di scala fra $t$ e $t-dt$ è data da:
\begin{equation}
d\mathcal{P}_{b}(t)=\frac{d f_b(x, \:t) }{ f_b(x, \:t)}= |dt| \sum_{ac}  \int  \frac{d x^{'}}{x^{'}} \frac{d f_a(x^{'}, \:t) }{ f_b(x, \:t)} \frac{\alpha_{abc}}{2\pi}          \:P_{a \rightarrow bc} \:(\frac{x}{x^{'}})      \mbox{,}\end{equation}
mentre la probabilità di non divisione fra la scala $t_{max}$ e $t<t_{max}$:
\begin{equation}
S_b (x,t,t_{max})=   \mbox{exp} \left( - \int_t ^{t_{max}} dt^{'} \sum_{ac}  \int  \frac{d x^{'}}{x^{'}} \frac{d f_a(x^{'}, \:t^{'}) }{ f_b(x, \:t^{'})} \frac{\alpha_{abc}}{2\pi}          \:P_{a \rightarrow bc} \:(\frac{x}{x^{'}}) \right)     \mbox{,}\end{equation} 
Infine allora la probabilità di ricombinare $b$  in $a$ è data nell'intervallo compreso fra $t$ e $(t-dt)$ da:

\begin{eqnarray}
d \mathcal{P}_{b}^{\mbox{\footnotesize{ISR}}}(t) &=& - \frac{d S_b (x,t,t_{max})}{dt} dt \nonumber \\
&=&  \sum_{ac}  \int  \frac{d x^{'}}{x^{'}} \frac{d f_a(x^{'}, \:t) }{ f_b(x, \:t)} \frac{\alpha_{abc}}{2\pi}          \:P_{a \rightarrow bc} \:(\frac{x}{x^{'}})  \cdot S_b (x,t,t_{max}) dt \end{eqnarray}
In questo caso il fattore di forma si Sudakov è differente rispetto a quello del FSR dato che contiene le PDF.
Questo fa sì che i risultati del \textit{parton shower} non dipendono solo dall’algoritmo ma anche dalle PDF usate.

\paragraph{Risommazione} Quando si calcola un'osservabile della QCD in modo perturbativo, l'espansione in termini in potenze di $\alpha_S$ contiene termini del tipo $\alpha_S^n L^k$ ($k<2n$), dove $L=\ln (q_{cut}/s)$, essendo $q_{cut}$ il taglio sull'emissioni risolvibili. Quando si considerano ``piccoli'' valori di $q_{cut}$ il logaritmo dell'espansione perturbativa diventa grande e può  far divergere la serie perturbativa. 
Il termine di ordine $n$ dell'espansione perturbativa è il più significativo solo se i termini successivi della serie sono trascurabili; tuttavia ciò non è garantito nel caso in cui siano presenti elevati valori di $L$. E' dunque necessario cosiderare i termini che hanno un elevato valore del logaritmo. Lo studio di questi termini è detto risommazione e viene effettuato mettendo insieme i termini nella serie perturbativa in base al loro grado di divergenza: $\alpha_S^n L^{2n}$ sono i termini \textit{leading log}, LL; $\alpha_S^n L^{2n-1}$ sono i termini \textit{next-to-leading log}, NLL, e così via. Alla fine viene effettuata la loro somma a tutti gli ordini di $\alpha_S$. Per  molti processi sono disponibili i calcoli al NLL.
Il \textit{parton shower} riproduce gli effetti della risommazione approssimativamente al NLL.

\paragraph{Combinazione fra ME e PS}
I due differenti approcci del calcolo dell'elemento di matrice e del \textit{parton shower} hanno dei vantaggi e degli svantaggi. Per quanto riguarda il ME si ha:
\begin{itemize}
\item i calcoli dell'elemento di matrice al LO possono essere effettuati esattamente fino a casi in cui sono presente molti \textit{jet} (dell'ordine di sei) nello stato finale.
\item si ha una buona descrizione di partoni separati
\item i calcoli perturbativi sono esatti 
\item tuttavia, la sezione d'urto diverge  nel caso collineare e \textit{soft}, quindi non è possibile una descrizione esaustiva della struttura interna del \textit{jet}.
\end{itemize}
D'altro lato il PS:
\begin{itemize}
\item è un approccio universale che produce una configurazione realistica dei partoni
\item le divergenze, nel limite collineare, sono trattate con l'introduzione del fattore di forma di Sudakov. Dunque si ha un'appropriata descrizione dell'evoluzione del \textit{jet} 
\item tuttavia il metodo fallisce quando si descrivono partoni separati, dato che l'approssimazione collineare in tal caso non può essere valida.
\end{itemize}
Chiaramente i due metodi sono complementari  ed una loro combinazione (o \textit{merge}) è auspicabile. Esistono differenti approcci che combinano il ME con il PS. La difficoltà principale è che non è facile coprire l'intero spazio delle fasi senza sovrapposizioni e buchi: si vuole descrivere  un processo in cui sono presenti $n$ partoni ben separati nello stato finale,  utilizzando sia l'elemento di matrice al LO ma  volendo anche includere la risommazione dei grandi logaritmi (LL, NLL) che è tipica del PS. Una descrizione schematica della combinazione per quattro \textit{jet}  è riportata in Fig \ref{merge} \cite{bib:lenzi}.
Sull'asse orizzontale sono riportati gli ordini di accoppiamento in $\alpha_S$, mentre su quello verticale la potenza del logaritmo.
Il PS descrive il LL ($m=2n$) e il NLL ($m=2n-1$) ovvero le sfere in verde (p.e. nel caso di $n=2$, $m=$4, 3 le due sfere colorate in verde e segnate come ``4'').
Le sfere che  descrivono l'evento con 4 \textit{jet}, combinando il ME con il PS, sono tutte le verdi, quella azzurra, e le tre rosse segnate con il  ``4''.
La difficoltà sorge perché il ME descrive esattamente tutte le sfere segnate con il ``4'': quindi se semplicemente  sommassimo i due approcci avremmo doppi conteggi delle sfere verdi denominate ``4''.    
\begin{figure}
\centering
\includegraphics[scale= 0.7]{merge}
\caption{ \textit{Rappresentazione schematica della combinazione fra ME e PS.   }}
\label{merge}
\end{figure}
I  principali approcci che combinano ME e PS sono:
\begin{itemize}
\item Ripesamento del \textit{parton shower}: l'idea di base  è partire  dal processo all'ordine più basso e successivamente ripesare l'emissione del PS come se fosse stata prodotta dal ME. Questo approccio non cambia la sezione d'urto, che rimane all'ordine più basso, ma migliora il popolamento dello spazio delle fasi \cite{ripesamento, ripesamento2}.  
\item Prescrizione CKKW: lo spazio delle fasi viene suddiviso in due zone utilizzando $k_{\perp}$ che è una misura del taglio $Q_0 ^2$: la regione in cui è prodotto il \textit{jet} è riempita con il ME, quella di evoluzione con il PS \cite{ckkw, ckkw2}. 
\item La prescrizione  MLM, che è pure molto diffusa, si basa sullo stesso principio, ma è  implementata in un modo diverso. 
\end{itemize}


\section{Interazioni multiple}
I protoni incidenti che partecipano all'interazione sono composti da un gran numero di partoni (\textit{quark} e gluoni) che possono interagire in modo indipendente gli uni con gli altri in aggiunta all'\textit{hard-process}.
La sezione d'urto totale per il processo di QDC $2\rightarrow2$ è dominata dal processo $t$, quindi la sezione d'urto diverge come $d p_{\perp}^2/p_{\perp}^4$ per $p_{\perp} \rightarrow 0$  \cite{Sjostrand:2006su}.
Dunque quando si simula un evento reale, in aggiunta all'evento \textit{hard}, caratterizzato dall'avere grandi impulsi trasversi trasferiti, si deve tener conto anche delle collisioni aggiuntive a piccolo  $p_{\perp}$. Se queste avvengono in modo indipendente allora ci si aspetta un distribuzione di Poisson, $P_n= \langle n \rangle^n \mbox{exp}(-\langle n \rangle)/n! $. Tuttavia la conservazione dell'energia e dell'impulso fa si che le interazioni non siano effettivamente indipendenti  sopprimendo così la possibilità, per $p_{\perp} \rightarrow 0$, di avere un elevato numero di interazioni. 
Va inoltre osservato che per eliminare la divergenza è necessario introdurre un valore di \textit{cut-off} dell’impulso trasverso, al di sotto del quale non si generano  collisioni.



\section{Adronizzazione}
Il processo di adronizzazione, in questo contesto, è un particolare modello, utilizzato nei generatori di eventi, che descrive il passaggio dallo stato partonico finale allo stato adronico finale, che  è un'osservabile sperimentale.  E' importante sottolineare che questa transizione è trattata in modo fenomenologico e  non mediante un approccio rigoroso.  Le due più importanti classi per l'adronizzazione sono il modello a stringhe e quello a \textit{cluster}. La differenza è che il primo trasforma i sistemi partonici direttamente in adroni, mentre il secondo compie un passo intermedio dove raggruppa gli oggetti ad una scala dell'ordine di $\sim 1$ GeV.

\paragraph*{Modello a stringhe}
Il più completo ed esauriente ``modello a stringhe'' è quello di Lund: sappiamo dalla  QCD  che fra partoni è presente una forza di confinamento lineare che aumenta con la distanza. Consideriamo, come esempio, lo stato finale in cui sono presenti due \textit{quark}, $ q \bar{q}$. Come i partoni si allontanano il tubo di flusso di colore viene ``stirato'' fra $q$ e $\bar{q}$, Fig. \ref{tubo} (a). Le dimensioni trasverse del tubo sono quelle tipiche adroniche, quindi di circa 1 fm.
Se il tubo è assunto essere uniforme, il potenziale cresce linearmente, $V(r)=\kappa r$, con $\kappa \approx$ 1 GeV/fm, costante della stringa.
\begin{figure}
\centering
\includegraphics[scale= 0.5]{stringone}
\caption{(a) \textit{Il tubo di flusso presente fra un quark e un antiquark che si allontanano.} (b) \textit{Moto e rottura di una stringa del sistema.}}
\label{tubo}
\end{figure}
A corte distanze sarebbe necessario introdurre un termine di Coulomb aggiuntivo, $\sim \frac{\alpha_s}{r}$, tuttavia nel modello di Lund si assume questo termine trascurabile.
Come il \textit{quark} e l'\textit{antiquark} si allontanano del vertice di creazione, l'energia potenziale accumulata nella stringa aumenta fino a che non si rompe dando origine ad una coppia $q' \bar{q}'$. Così il sistema si divide in due nuovi singoletti di colore $q \bar{q} '$ e $q' \bar{q}$. Questi due sistemi si allontaneranno a loro volta ripetendo  il processo appena descritto. L'evoluzione del sistema nello spazio-tempo è rappresentata in \ref{tubo} (b).
Alla fine del processo si avrà una seri di coppie $q_i \bar{q_i}$, ognuna delle quali formerà un adrone.
Per ora  è stato considerato solamente il caso $q \bar{q}$. Tuttavia se più partoni provengono dall'interazione il modello a stringhe diventa più complicato. Per un evento in cui è presente un gluone  aggiuntivo, $q \bar{q} g$, la stringa è tesa fra $q$ e $g$ e fra $g$ e $\bar{q}$, Fig. \ref{tubo3}. 
\begin{figure}
\centering%
{\includegraphics[scale= 0.5]{stringtwo22}}
\caption{ \textit{Moto della stringa nel caso $q \bar{q}g$.}}
\label{tubo3}
\end{figure}

\begin{figure}
\centering%
\subfigure[]%
{\includegraphics[scale= 1.5]{split2}}
\subfigure[]%
{\includegraphics[scale= 1.4]{split}}
\caption{(a)  \textit{Struttura del parton shower nel modello cluster.} (b) \textit{Distribuzione di massa invariante per singoletti.}}
\label{tubo2}
\end{figure}


\paragraph*{Modello \textit{cluster}}  Questo modello di adronizzazione è basato sulla proprietà di preconfinamento del \textit{parton shower}: la massa invariante di una singola coppia di partoni con colore opposto è la stessa  a qualsiasi scala $Q^2$. Questa distribuzione ha il suo massimo ad una massa  che è circa il \textit{cutoff} del \textit{parton shower} e  decresce  rapidamente verso lo zero, Fig \ref{tubo2} (a).\\
Nel modello, i gluoni del \textit{parton shower}, sono rappresentati da coppie di linee colore-anticolore connesse al vertice. Ogni linea di colore, in prossimità del  \textit{cutoff}, è collegata con un'altra linea di anticolore presente alla  stessa scala. A questo punto le linee contigue di colore/anticolore sono interpretate, nel limite non perturbativo, come  coppie \textit{quark-antiquark} che danno origine a  mesoni, i quali sono gli oggetti osservabili nello stato finale.
Questo  meccanismo è rappresentato in \ref{tubo2} (b).

\section{Decadimenti adronici e radiazione elettromagnetica.}
Nella fase di adronizzazione possono essere prodotti  adroni instabili che decadono  in altre particelle. Dunque lo stato finale dell'evento è il risultato della convuluzione fra l'adronizzazione e il decadimento. Le informazione necessari per la simulazione delle particelle instabili del decadimento sono generalmente presa dal ``Particle Data Book'' (PDG) \cite{bib:pdg} che fornisce le proprietà (p.e. vita media) di un gran numero di particelle.
In generale, in un generatore di eventi, è necessario scegliere quali adroni includere nella simulazione e successivamente scegliere i possibili canali di decadimento. Oltre ai decadimenti adronici, risulta necessario simulare anche l'emissione di radiazione elettromagnetica. L'approccio più comune adottato è quello di utilizzare algoritmi analoghi a quelli utilizzati per simulare l'emissione di QCD nel \textit{parton shower}.

\section{Ricostruzione dei \textit{jet}}
\label{rico_jet}
Successivamente all'adronizzazione e al decadimento delle particelle instabili è possibile stimare il quadrimpulso dei partoni generati nell'\textit{hard process}  dalla direzione e dall'energia dei \textit{jet} che vengono ricostruiti a partire dalle particelle nello stato finale \cite{bib:run2jet, mass:in:dijet}.
La ricostruzione dei \textit{jet} è affidata ad appositi algoritmi; questi introducono la variabile  distanza, $d_{ij}$, fra due oggetti (particelle o pseudo-\textit{jet}) definita da,
\begin{equation}
d_{ij}=\mbox{min}( k_{ti}^{2p}, k_{tj}^{2p})  \frac{\Delta_{ij}^2}{R^2} \mbox{,}\end{equation}
dove $\Delta_{ij}^2=(y_i - y_j)^2+ (\phi_i - \phi_j)^2$ mentre $k_{ti}$, $y_i$ e $\phi_i$ sono rispettivamente l'impulso trasverso, la rapidità e l'angolo azimutale di $i$. Invece $R$ è il parametro radiale. Si introduce inoltre la distanza fra un oggetto $i$ e il fascio (\textit{beem}), $d_{iB}=k_{ti}^{2p}$ \\

Gli algoritmi procedono calcolando la distanza $d_{ij}$ e fra tutte le coppie di particelle $i$,  $j$ identificando  la minore. Per le due particelle con distanza più piccola si sommano i quadripulsi. Si valuta inoltre   $d_{iB}$ per ogni $i$ e se è minore della distanza  $d_{ij}$ con tutte le altre particelle $j$,  $i$ è considerato un \textit{jet} e viene rimosso   dalla lista degli oggetti presenti nell'evento. 
Infine le distanze vengono ricalcolate e tutta questa procedura è ripetuta finché non si trovano più oggetti da sommare.
Il valore di $p=-1$ definisce l'algoritmo   \textit{anti-}$k_t$ \cite{Cacciari:2008gp}, che è quello utilizzato, mentre   il parametro libero $R$ è stato posto  uguale a 0.5.

\section{Validazione Monte Carlo}
Come già accennato, le simulazioni Monte Carlo sono utilizzate in fisica delle particelle per confrontare le predizioni teoriche con i dati. Inoltre dall'analisi dei dati dei processi previsti nel Modello Standard è possibile ricavare i parametri liberi della teoria che possono essere così inseriti in \textit{input} all'interno dei  generatori di eventi. 
Tuttavia quando si simula un evento con un generatore è importante distinguere la ``verità Monte Carlo'' dalle ``osservabili fisiche''. 
Infatti può essere utile dividere il processo di interesse in vari sotto-processi intermedi come, per esempio, lo stato iniziale   o la produzione di una risonanza. 
Questi oggetti intermedi non sono osservabili fisiche ed in pratica non è possibile effettuare misure dirette o indirette. Contrariamente, nel mondo simulato, è possibile accedere a questi oggetti, e di conseguenza anche alle variabili che li caratterizzano (p.e. $p_{\perp}$, $\eta$, $\dots$).
In aggiunta, nella modellizzazione, è possibile (e conveniente) produrre, in modo distinto,  eventi di solo segnale (S) e di solo fondo\footnote{Per fondo si intende tutti gli eventi che producono  stati finali analoghi a quelli di segnale.} (B).\\
In questo contesto, la validazione dei generatori Monte Carlo consiste nel confrontare i risultati con quelli di altri generatori, con le predizioni dei calcoli analitici e dove possibile con i dati. È superfluo aggiungere che la validazione del generatore Monte Carlo permette molto spesso anche di trovare  errori (\textit{bug}) nel codice di analisi. 
\paragraph*{Rivet} 
Uno dei principali strumenti per la validazione dei Monte Carlo è Rivet (\textit{Robust Independed Validation of Experiment and Theory}) \cite{Buckley:2010ar}; questo programma offre un insieme di analisi \textit{standard} con le quali è possibile verificare l'accuratezza di un dato generatore. In aggiunta l'utilizzatore può scrivere una sua propria analisi utilizzando tutti i componenti di Rivet e questa analisi diventa poi un \textit{plugin} che può essere aggiunto alle librerie in C++ del programma..
Rivet permette di visualizzare i risultati dell'analisi in  istogrammi fornendo in aggiunta, nella parte inferiore del grafico, anche il rapporto fra il numero di eventi presenti nei differenti campioni utilizzati (se più di uno).


\section{Generatori principali}
Per la fisica delle alte energie sono disponibili differenti generatori Monte Carlo.  Ognuno di questi ha metodi differenti per combinare il  ME con il  PS.
Qui ci concentreremo in particolare su  \aMC  interfacciato con  P{\footnotesize YTHIA},   P{\footnotesize OWHEG} -anch'esso interfacciato con   P{\footnotesize YTHIA}- e S{\footnotesize HERPA}. 


\paragraph{Madgraph\_aM{\footnotesize C@NLO}}
L'approccio di \aMC \cite{bib:madgraph} è molto ambizioso, infatti lo scopo di questo  generatore è  calcolare la sezione d'urto al NLO includendo nel calcolo sia  i contributi reali che virtuali. L'\textit{hard process} è prodotto col metodo del ME mentre l'emissioni \textit{soft} e collineare col PS.
Il primo passo è calcolare le correzioni al NLO del ME per un processo a $n$ partoni, includendo   $n+1$ partoni provenienti dalle correzioni reali ed  $n$ provenienti da quelle virtuali. Successivamente si valuta come il \textit{parton shower} popola lo spazio delle fasi a $n+1$ partoni  escludendo in questa fase il fattore di forma di Sudakov. Per ottenere il ``vero'' stato in cui  sono presenti $n+1$ partoni \aMC sottrae l'espressione del PS dallo stato $n+1$ del ME. L'espressioni del PS senza fattore di Sudakov e del  ME sono in accordo nel limite \textit{soft} e collineare, quindi le singolarità sono cancellate ottenendo così un valore finito per la sezione d'urto nel caso di $n$ e $n+1$ partoni.  Un problema tecnico è che nel limite collineare non si ha la certezza che il ME sovrasti sempre il PS. Questo problema è risolto introducendo una frazione di eventi con peso negativo, Fig. \ref{weight}.  Infine viene applicato il \textit{parton shower}, che  include il fattore di Sudakov e dunque permette di ottenere un risultato finito e corretto al NLL. 
 




\begin{figure}
\centering
\includegraphics[scale= 0.7]{weight}
\caption{\textit{Distribuzione dei pesi, per differenti generatori Monte Carlo,con  normalizzazione alla sezione d'urto di 1 fb$^{-1}$.}}
\label{weight}
\end{figure}

%Nell'approccio ai Monte Carlo col \textit{parton shower}, si parte dalle sezione d'urto di Born e si aggiungono le correzioni all'ordine successivo dovute al  \textit{parton shower}. Tuttavia nel caso si parta dalla sezione d'urto NLO, sono presenti, aggiungendo il \textit{parton shower}, doppi conteggi degli stessi diagrammi dovuti in particolare all'emissione reale ed al termine negativo proveniente dall'espansione al primo ordine del fattore di forma si Sudakov. 
%Lo scopo di  aM{\footnotesize C@NLO} è di rimuovere questi termini aggiuntivi dall'espressione al NLO. 
%L'eliminazione delle divergenze rende possibile la produzione di due differenti campioni, uno per la parte di Born ($\mathbb{S}$) ed uno per la parte reale ($\mathbb{H}$) ciascuno con un suo proprio peso. $\mathbb{S}$ e $\mathbb{H}$ devono essere accettati in modo proporzionale al loro peso prima di applicare il  \textit{parton shower}.  In generale non si ha la garanzia che i pesi siano positivi (Fig. \ref{weight}), infatti in alcune configurazioni è necessario sottrarre (o in altre aggiungere) eventi.

\paragraph{P{\footnotesize OWHEG}} L'idea alla  base di   P{\footnotesize OWHEG} \cite{Oleari:2010nx} è generare per prima cosa la radiazione più dura, e successivamente  passare l'evento al generatore del \textit{parton shower}. Nei generatori di \textit{parton shower} la produzione, ordinata in impulso trasverso, della radiazione più dura è sempre la prima; quindi  P{\footnotesize OWHEG} sostituisce semplicemente questa con l'emissione al NLO. 
In   P{\footnotesize OWHEG} gli eventi sono prodotti con un peso positivo e costante (Fig. \ref{weight}).


\paragraph{P{\footnotesize YTHIA}8 }  P{\footnotesize YTHIA}8 \cite{bib:pythia} è un generatore che può calcolare il ME per processi con due particelle o partoni nello stato finale, ma soprattutto genera il \textit{parton shower} e la successiva adronizzazione. Il \textit{parton shower} è ordinato in impulso trasverso, $p_T$, e la prima emissione è corretta con il metodo del ripesamento. Per l’adronizzazione utilizza il modello di Lund.  


\paragraph{S{\footnotesize HERPA}}  S{\footnotesize HERPA}  \cite{bib:sherpa} è un generatore Monte Carlo che come PYTHIA8  fornisce una descrizione completa della collisioni adroniche, dal calcolo dell’elemento di matrice, fino all’adronizzazione.  Il \textit{parton shower} include sia le emissioni QCD che quelle dovute alla QED, ovvero i fotoni. Può calcolare il ME per i processi principali (p.e. $gg \rightarrow H$) al NLO e combinare il ME con il  PS. Il codice è scritto completamente in linguaggio C$++$.  

